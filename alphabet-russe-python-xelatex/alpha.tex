% Try this code on https://www.overleaf.com/project/5df27a25916aef0001c87ccd
\documentclass[12pt,a4paper]{article}
\usepackage{fontspec}
\usepackage{xunicode}
\newfontfamily\cyrillicfont{Times New Roman}
\usepackage{polyglossia}
\usepackage{xltxtra}
\setmainlanguage{french}
\setotherlanguages{russian, english}
\usepackage{hyperref}
\usepackage{graphics}
\hypersetup{colorlinks=true, citecolor=electricblue,filecolor=electricblue,linkcolor=red,urlcolor=blue}
\author{Laurent Garnier}
\date{}
\title{Alphabet cyrillique Russe \textrussian{русский}\\ avec Python 3 et \XeLaTeX}

\begin{document}

\begin{titlepage}

\begin{center}

{\Large {\sc Alphabet cyrillique Russe\par avec Python 3\par et \XeLaTeX} \par}
\vspace{2cm}
\end{center}
\newpage

\vspace{2cm}

\begin{center}

BY

{\Large Laurent \textsc{Garnier} \par}

\vspace{2cm}
{\sc Visit my blog on languages:}
 \url{www.polyglothuman.fr}\par
{\sc Visit my blog on Russian:}
 \url{www.govoritparoussky.fr}\par

\end{center}
\newpage

\begin{center}

 {\sc Visit my YouTube channel:} 
 \url{http://bit.ly/LanguageScience}\par

 {\sc Visit my playlist on Russian:} 
 \url{http://bit.ly/LearnRussianWithMe}\par

\end{center}

\end{titlepage}

\tableofcontents
\section{Alphabet cyrillique russe}
\begin{center}
 \begin{tabular}{c|c|c}
    Indice & Lettre Majuscule & lettre minuscule \\
    \hline
    0 & \textrussian{А} & \textrussian{а} \\
    1 & \textrussian{Б} & \textrussian{б} \\
    2 & \textrussian{В} & \textrussian{в} \\
    3 & \textrussian{Г} & \textrussian{г} \\
    4 & \textrussian{Д} & \textrussian{д} \\
    5 & \textrussian{Е} & \textrussian{е} \\
    6 & \textrussian{Ж} & \textrussian{ж} \\
    7 & \textrussian{З} & \textrussian{з} \\
    8 & \textrussian{И} & \textrussian{и} \\
    9 & \textrussian{Й} & \textrussian{й} \\
    10 & \textrussian{К} & \textrussian{к} \\
    11 & \textrussian{Л} & \textrussian{л} \\
    12 & \textrussian{М} & \textrussian{м} \\
    13 & \textrussian{Н} & \textrussian{н} \\
    14 & \textrussian{О} & \textrussian{о} \\
    15 & \textrussian{П} & \textrussian{п} \\
    16 & \textrussian{Р} & \textrussian{р} \\
    17 & \textrussian{С} & \textrussian{с} \\
    18 & \textrussian{Т} & \textrussian{т} \\
    19 & \textrussian{У} & \textrussian{у} \\
    20 & \textrussian{Ф} & \textrussian{ф} \\
    21 & \textrussian{Х} & \textrussian{х} \\
    22 & \textrussian{Ц} & \textrussian{ц} \\
    23 & \textrussian{Ч} & \textrussian{ч} \\
    24 & \textrussian{Щ} & \textrussian{щ} \\
    25 & \textrussian{Ъ} & \textrussian{ъ} \\
    26 & \textrussian{Ь} & \textrussian{ь} \\
    27 & \textrussian{Э} & \textrussian{э} \\
    28 & \textrussian{Ю} & \textrussian{ю} \\
    29 & \textrussian{Я} & \textrussian{я} \\
  \end{tabular}
\end{center}
\section{Retrouvez les détails et les sons sur mon blog}
{\Large Allez consulter cet article de blog dans lequel je détaille la logique de l'alphabet cyrillique russe :\par}
\begin{center}
\url{https://govoritparoussky.fr/comment-apprendre-lalphabet-russe-meme-quand-on-est-debutant/}
\end{center}
\section{Quelques exemples}
\begin{description}
\item[russe]\textrussian{а ты говоришь по русски?}
\item[français]Parles-tu Russe ?
\item[russe]\textrussian{Я немного говорю по русски}
\item[français] Je parle un peu russe
\end{description}
\end{document}
