% Created 2018-01-15 Mon 17:42
% Intended LaTeX compiler: pdflatex
\documentclass[11pt]{article}
\usepackage[utf8]{inputenc}
\usepackage[T1]{fontenc}
\usepackage{graphicx}
\usepackage{grffile}
\usepackage{longtable}
\usepackage{wrapfig}
\usepackage{rotating}
\usepackage[normalem]{ulem}
\usepackage{amsmath}
\usepackage{textcomp}
\usepackage{amssymb}
\usepackage{capt-of}
\usepackage{hyperref}
\author{Laurent Garnier}
\date{\today}
\title{Papier bleu de Steem (\href{https://steem.io/steem-bluepaper.pdf}{Steem bluepaper})}
\hypersetup{
 pdfauthor={Laurent Garnier},
 pdftitle={Papier bleu de Steem (\href{https://steem.io/steem-bluepaper.pdf}{Steem bluepaper})},
 pdfkeywords={},
 pdfsubject={},
 pdfcreator={Emacs 25.3.1 (Org mode 9.1.3)}, 
 pdflang={English}}
\begin{document}

\maketitle
\tableofcontents

Un protocole pour permettre une monnaie sociale intelligente pour les
éditeurs et les entreprises de contenu à travers l'Internet.

\section{Introduction}
\label{sec:orgd80601b}
Steem fournit un protocole\footnote{Delegated Proof of Stake Position
Paper. Grigg, 2017. \url{https://steemit.com/eos/@iang/seeking-consensus-on-consensus-dpos-or-delegated-proof-of-stake-and-the-\%0Atwo-generals-problem}} de chaîne de blocs (blockchain)
évolutif pour un contenu accessible publiquement et immuable, ainsi
qu'un jeton numérique rapide et sans frais appelé STEEM\footnote{Pour le différentier de sa chaîne de blocs, l'orthographe
correcte du jeton numérique natif de Steem est STEEM.}) ce
qui permet aux gens de gagner de la monnaie (STEEM) en utilisant
leur cerveau (ce qu'on peut appeler "Preuve de cerveau"
[Proof-of-Brain]). Les deux blocs de construction de ce protocole, à
la fois la chaîne de blocs et le jeton, dépendent l'un de l'autre
pour la sécurité, l'immuabilité et la longévité, et sont donc partie
intégrante de l'existence de chacun. Steem a fonctionné avec succès
pendant plus d'un an, et a maintenant dépassé Bitcoin et Ethereum en
nombre de transactions traitées.\footnote{Volumes de transaction: transactions par seconde. Steem Témoin
et utilisateur
"@roadscape". \url{https://steemit.com/blockchain/@roadscape/tps-report-2-the-flippening}}

Comparé à d'autres chaînes de blocs, Steem se distingue comme la première
base de données accessible au public pour du contenu stocké de façon
immuable sous la forme de texte brut, avec un mécanisme d'incitation
intégré. Cela fait de Steem une plate-forme de publication publique
à partir de laquelle toute application Internet peut tirer et
partager des données tout en récompensant ceux qui contribuent au
contenu le plus précieux. 

Dans le domaine des crypto-monnaies, les propriétés uniques de STEEM
le rendent à la fois «intelligente» et «sociale» par rapport à
d'autres, comme le bitcoin et l'éther. Cela provient de deux
nouvelles fonctionnalités du jeton. La première est une piscine de
jetons (pool of tokens) dédiés à la création de contenu et à la
conservation (appelé «piscine de récompenses» ["rewards pool"]). La
seconde est un système de vote qui tire parti de la sagesse de la
foule pour évaluer la valeur du contenu et distribuer des jetons à
celui-ci. Ces deux propriétés uniques lorsqu'elles sont combinées
sont appelées Preuve de Cerveau, ce qui est un principe basé sur la
Preuve de Travail (Proof-of-Work),\footnote{Proof-of-Work. Wikipedia. \url{https://en.wikipedia.org/wiki/Proof-of-work\_system}
et je me permets d'ajouter cette vidéo dont j'ai fait les sous-titres
\href{https://youtu.be/bBC-nXj3Ng4}{Bitcoin and Proof-of-Work in 26 minutes}} déstiné à insister sur le
travail humain requis pour distribuer des jetons aux participants de
la communauté. La preuve de cerveau positionne STEEM comme un outil
pour construire des communautés en perpétuelle croissance, qui
encouragent leurs membres à ajouter de la valeur à la communauté
grâce à la structure de récompenses intégrée. 

En plus de ces progrès dans la technologie de la chaîne de bloc et des jetons,
Steem en tant que système fournit des fonctionnalités avancées
supplémentaires pour améliorer l'expérience de l'utilisateur, comme
la récupération de compte volé\footnote{Stolen Account Recovery initiation for Steemit.com users:
07-13-2017 \url{https://steemit.com/recover\_account\_step\_1} (en français
s'il vous plaît)}, les services d'entiercement\footnote{NdT : Ne connaissant pas ce mot français en voici une définition \href{http://www.larousse.fr/dictionnaires/francais/entiercement/29936}{Larousse}.},
le contenu favorisé par l'utilisateur et les comptes d'épargne. Tout
cela est fait tout en fournissant aux utilisateurs trois fois le
temps de confirmation et zéro frais sur toutes les
transactions. Tout cela lui permet de soutenir la mission d'apporter
une monnaie intelligente et sociale aux éditeurs et bâtisseurs de
communautés sur Internet. 










\section{Preuve de Cerveau : Jetons Intelligents et Sociaux}
\label{sec:org10d3240}
Les systèmes de jetons qui récompensent les utilisateurs lorsqu'ils
contribuent à un système de communauté basé sur des jetons
nécessitent des mécanismes pour établir et évaluer la valeur sociale
du contenu: nous appelons cela «preuve de cerveau».
\subsection{La piscine de récompenses ("D'où les jetons viennent-ils ?")}
\label{sec:org9946c94}
L'un des aspects les plus innovants (et des plus mal compris) de la
chaîne de blocs Steem est la "piscine de récompenses" à partir de
laquelle les jetons sont distribués aux créateurs de contenu de
valeur. Afin de comprendre ce qu'est la piscine de récompenses, il
faut d'abord comprendre que les jetons sont produits différemment
dans les chaînes de blocs DPoS\footnote{NdT : DPoS: \href{https://www.liskafrica.com/guides/la-preuve-de-participation-d\%25C3\%25A9l\%25C3\%25A9gu\%25C3\%25A9-dpos}{Delegateed Proof-of-Stake} (expliqué en français) et
\href{https://hackernoon.com/explain-delegated-proof-of-stake-like-im-5-888b2a74897d}{Hackernoon} (expliqué en anglais)} qu'ils ne le sont dans les
chaînes de blocs PoW (Preuve de Travail). Dans les chaînes de blocs PoW
traditionnelles, les jetons sont produits régulièrement mais
distribués aléatoirement aux personnes dont les machines effectuent
un travail ("les mineurs"). 

Différent des crypto-monnaies de type PoW, les jetons de Steem sont
générés à un taux fixe d'un bloc toutes les trois secondes. Ces
jetons sont distribués aux différents acteurs du système selon les
règles définies par la chaîne de blocs. Ces acteurs, tels que les
créateurs de contenu, les témoins et les conservateurs
(organisateurs ?\footnote{NdT : Les mots \emph{\href{http://www.wordreference.com/enfr/curators}{curator}} ou \emph{curation} sont particulièrement
chiants à traduire.}), rivalisent de manière spécialisée pour les
jetons. Contrairement aux moyens traditionnels de distribution de
PoW, où les mineurs rivalisent sur la puissance informatique brute,
les acteurs du réseau Steem sont incités à être compétitifs de
manière à ajouter de la valeur au réseau.

Le taux de génération de nouveaux jetons a été fixé à 9,5\% par an à
compter de décembre 2016, et diminue à raison de 0,01\% tous les 250
000 blocs, soit environ 0,5\% par an. L'inflation continuera à
décroître à ce rythme jusqu'à atteindre 0,95\%, après une période
d'environ 20,5 ans.

De l'offre de nouveaux jetons créés par la chaîne de blocs Steem chaque
année, 75\% de ces jetons composent la «piscine de récompenses» qui sont
distribués aux créateurs de contenu et aux conservateurs de
contenu. 15\% sont distribués aux détenteurs de jetons acquis, et
10\% sont distribués aux Témoins, les producteurs de blocs coopérant
au protocole de consensus DPoS de Steem.

\subsection{Récompenses pour les créateurs de contenu et les conservateurs}
\label{sec:org8388cad}
Les utilisateurs qui produisent du contenu ajoutent de la valeur au
réseau en créant du matériel qui attirera de nouveaux utilisateurs
vers la plateforme, tout en maintenant l'engagement et le
divertissement des utilisateurs existants. Cela facilite la
distribution de la monnaie à un plus grand nombre d'utilisateurs et
augmente l'effet de réseau. Les utilisateurs qui prennent le temps
d'évaluer et de voter pour le contenu jouent un rôle important dans
la distribution de la devise aux utilisateurs qui ajoutent le plus
de valeur. La chaîne de blocs récompense ces deux activités par rapport
à leur valeur sur la base de la sagesse collective de la foule
rassemblée à travers le système de vote pondéré par les parts.
\subsection{Voter avec des jetons jalonnés pour déterminer l'attribution des récompenses}
\label{sec:orgb29a2f3}
Steem opère sur la base de un-STEEM, un vote. Selon ce modèle, les
personnes qui ont le plus contribué à la plateforme, selon le solde
de leur compte, ont le plus d'influence sur la façon dont les
contributions sont évaluées. L'enjeu peut être acheté ou gagné. Les
utilisateurs ne peuvent pas acquérir une influence supplémentaire
en possédant plusieurs comptes, car un même compte avec un montant
de participation aura la même influence que deux comptes différents
partageant le même montant de participation. Le seul moyen pour les
utilisateurs d'accroître leur influence sur la plateforme est
d'augmenter leur participation.

En outre, Steem ne permet pas aux membres de voter avec STEEM quand il
est engagé à un calendrier d'acquisition de 13 semaines appelé
Steem Power. Selon ce modèle, les membres ont une incitation
financière à voter de manière à maximiser la valeur à long terme de
leur STEEM.
\section{Vitesse et échelle sur la Blockchain Steem}
\label{sec:org882bf6f}
La chaîne de blocs Steem est conçue pour être l'une des chaînes de blocs
les plus rapides et les plus efficaces, ce qui est nécessaire pour
supporter la quantité de trafic attendue sur une plate-forme de
médias sociaux plus grande que la taille de Reddit. Steem a déjà
dépassé Bitcoin en nombre de transactions, et est capable d'évoluer
pour prendre en charge 10 000 transactions ou plus par seconde.
\subsection{La preuve de participation déléguée (DPoS)}
\label{sec:org8738ddc}
Souvent engorgées par la preuve de travail (PoW)\footnote{Bitcoin Scalability Problem
\url{https://en.wikipedia.org/wiki/Bitcoin\_scalability\_problem}}, de
nombreuses chaînes de blocs ne peuvent pas s'étendre au-delà de
trois transactions par secondes, ce qui est une fraction du trafic
financier mondial. Steem avait besoin d'une bien plus grande
échelle et vitesse que celle offerte par PoW, et ainsi un
algorithme moins connu appelé Preuve de Participation Déléguée
(DPoS)\footnote{DPoS Whitepaper
\url{https://steemit.com/dpos/@dantheman/dpos-consensus-algorithm-this-missing-white-paper}} a été mis à profit pour jeter les bases d'une
chaîne de blocs adaptée à des milliards d'utilisateurs. 

Grâce au DPoS, la chaîne de blocs Steem est capable de générer un nouveau
bloc toutes les 3 secondes avec une charge de calcul minimale. Cela
signifie que la chaîne de blocs peut traiter plus de transactions et
contenir plus d'informations, y compris du contenu.

En définissant les règles pour le cas où un Hardfork se produit,
les témoins élus dans le cadre DPoS peuvent rapidement et
efficacement décider d'aller de l'avant avec une proposition
de hardfork, permettant au protocole chaîne de blocs Steem d'évoluer plus
rapidement que la plupart des autres. La chaîne de blocs Steem a déjà
bifurquée 18 fois\footnote{\url{https://steemit.com/steemit/@steemitblog/proposing-hardfork-0-20-0-velocity}}, et chaque fois qu'un Hardfork s'est
produit, une seule chaîne a persisté après la bifurcation.
\subsection{Chaîne de base (ChainBase)}
\label{sec:org4ec419b}
ChainBase\footnote{ChainBase Release
\url{https://steemit.com/steem/@steemitblog/announcing-steem-0-14-4-shared-db-preview-release}} est la partie de la base de données de la pile
de chaîne de blocs et qui a remplacé Graphene\footnote{Graphene Documentation \url{http://docs.bitshares.org/}}
en 2016. ChainBase a des temps de chargement et de sortie plus
rapides, prend en charge l'accès parallèle à la base de données et
est plus robuste contre les plantages que son prédécesseur.La
corruption de la base de données est également moins fréquente,
elle permet un "instantané" instantané de l'état de la base de
données et peut servir plus de requêtes RPC à partir de la même
mémoire.  
\subsection{AppBase}
\label{sec:org0b8a16d}
AppBase est la première étape dans la création d'une FABRIC
multi-chaîne. AppBase permet à de nombreux composants de la chaîne
de blocs Steem de devenir modulaire en créant des chaînes de blocs
supplémentaires sans consensus comme greffons (plugins) dédiés. Ces
greffons peuvent être mis à jour beaucoup plus rapidement parce
qu'ils n'ont pas besoin de rejouer la chaîne de blocs en
entier. Cela rend steemd\footnote{Le composant du cadre (framework) de la chaîne de blocs Steem
responsable du traitement des transactions et de la distribution des
récompenses.} beaucoup plus efficace et facile à
maintenir et à étendre.

En pratique, AppBase permet à différents cœurs, ou même à
différents ordinateurs, de maintenir différentes parties de la
chaîne de blocs Steem. Cela est significativement plus efficace que
de nécessiter que chaque coeur, et chaque ordinateur du réseau
maintienne l'entière chaîne de blocs. Modulariser la chaîne de
blocs permet de tirer pleinement parti de la nature modulaire des
ordinateurs. Cela est un pas nécessaire dans le long processus de
création d'une chaîne de blocs entièrement parallèle, entièrement
optimisée. 
\section{Fonctionnalités de la plate-forme Steem}
\label{sec:org28b77d4}
La chaîne de blocs Steem sert un double but d'être à la fois un système de
traitement de jeton numérique, et une plate-forme de média social de
masse. Les fonctionnalités offertes par la chaîne de blocs
nécessitent de soutenir les deux buts, et de fournir aux
utilisateurs une experience de classe mondiale lorsqu'ils utilisent
chacun des aspects de la plate-forme.
\subsection{Primitives conçues pour les applications de contenu}
\label{sec:org7c77a4c}
Steem offre aux utilisateurs la possibilité unique de publier et
stocker différents types de contenus directement et de façon
permanente à l'intérieur du registre immuable de la chaîne de blocs
en tant que texte brute. Une fois stockées dans la chaîne de blocs,
les données deviennent disponibles publiquement pour les
développeurs pour construire avec. Les développeurs sont capables
d'intéragir avec le contenu directement dans la chaîne de blocs en
utilisant les APIs\footnote{NdT : l'acronyme API est un peu partout présent sur internet
et comme souvent jamais traduit alors j'en profite ici : \href{https://fr.wikipedia.org/wiki/Interface\_de\_programmation}{interface de programmation}} disponibles. Plusieurs développeurs de
primitives de la chaîne de blocs peuvent construire à partir de noms de
compte, messages, commentaires, votes et solde de compte.
\subsection{Système de nom natif}
\label{sec:org9c571df}
Les adresses de portefeuille utilisées par de nombreuses
technologies de la chaîne de blocs, telles que Bitcoin et Ethereum, se
composaient historiquement de longues chaînes de lettres et de
chiffres aléatoires. Cependant, ces adresses de portefeuille
peuvent rendre difficile la transaction avec d'autres utilisateurs
dans un contexte de média social en ligne parce que les
utilisateurs ne peuvent pas se rappeler les adresses de longue chaîne
de mémoire. La chaîne de blocs Steem utilise chaque nom
d'utilisateur comme adresse de portefeuille, ce qui renforce
l'expérience utilisateur pour les participants qui tentent
d'envoyer des jetons car ils peuvent vérifier les adresses à partir
de leur propre mémoire. 
\subsection{Steem Blockchain Dollars (SBD)}
\label{sec:orge2b05e1}
De nombreux utilisateurs initiés à la crypto-monnaie peinent à
comprendre comment les «jetons internet magiques» attribués par la
plateforme peuvent réellement avoir une valeur réelle. Afin de
combler le fossé entre les systèmes de monnaie fiduciaire plus
traditionnels auxquels les utilisateurs traditionnels sont habitués
et les jetons de crypto-monnaie qui leur sont attribués via la
plate-forme, une nouvelle monnaie appelée Steem Blockchain Dollars
(SBD) a été créée. Les jetons SBD sont conçus pour être étroitement
liés à un dollar, de sorte que les utilisateurs qui les reçoivent
peuvent savoir à peu près combien ils valent en termes de «dollars
réels». Les jetons SBD offrent également une monnaie relativement
stable aux utilisateurs s'ils veulent préserver la valeur de leur
compte par rapport à l'USD. Une explication technique plus
détaillée peut être trouvée dans le livre blanc technique de
Steem\footnote{Steem Whitepaper \url{https://steem.io/SteemWhitePaper.pdf}}.
\subsection{Echange décentralisé}
\label{sec:org6d94add}
La chaîne de blocs Steem offre un échange de jeton décentralisé,
similaire aux échanges de Bitshare.\footnote{Bitshares Decentralized Exchange \url{http://docs.bitshares.org/\_downloads/bitshares-general.pdf}} L'échange permet aux
utilisateurs d'échanger leurs jetons STEEM et SBD via un marché
pair à pair public décentralisé. Les utilisateurs peuvent placer
des ordres d'achat et de vente, et l'appariement des ordres est
effectué automatiquement par la chaîne de blocs. Il existe
également un carnet de commandes accessible au public et un
historique des commandes que les utilisateurs peuvent utiliser pour
analyser le marché. Les utilisateurs peuvent interagir directement
avec l'échange en utilisant l'API de la chaîne de blocs, ou utiliser une
interface graphique telle que celle sur Steemit.com.\footnote{Steemit.com Currency Market \url{https://steemit.com/market}} 
\subsection{Paiements par l'intermédiaire de l'entiercement}
\label{sec:org15abc7b}
La nature irréversible des transactions de la chaîne de blocs est un élément
de sécurité important, bien qu'il y ait de nombreux cas où les
utilisateurs ne soient pas à l'aise pour envoyer leurs jetons à un
autre individu sans moyen de les récupérer si l'autre utilisateur
ne respecte pas leur contrat. La chaîne de blocs Steem fournit un
moyen pour les utilisateurs de s'envoyer des pièces de monnaie avec
un tiers désigné comme un service d'entiercement. L'utilisateur
agissant comme le service d'entiercement est capable de déterminer
si les termes de l'accord ont été respectés, et soit autoriser les
fonds à être remis au récepteur ou renvoyé à l'expéditeur.
\subsection{Structure de clé privée hiérarchique}
\label{sec:org85c60fc}
Steem utilise un système de clé privée hiérarchique unique en son
genre pour faciliter les transactions à faible sécurité et haute
sécurité. Les transactions à faible sécurité ont tendance à être
sociales, comme l'affichage ou les commentaires. Les transactions
de haute sécurité ont tendance à être des transferts et des
changements de clés. Cela permet aux utilisateurs d'implémenter
différents niveaux de sécurité pour leurs clés, en fonction de
l'accès que les clés permettent. 

Ces clés privées sont la Publication, Active et Propriétaire. La
clé de publication permet aux comptes de publier, commenter,
modifier, voter, resteem\footnote{"Resteem" est le terme utilisé dans la chaîne de blocs Steem
lorsqu'un utilisateur partage le contenu avec ses abonnés.}, et suivre/rendre muet d'autres
comptes. La clé active est destinée à des tâches plus délicates
telles que le transfert de fonds, les transactions de montée /
descente, la conversion de Steem Dollars, le vote de témoins, la
passation d'ordres au marché et la réinitialisation de la clé de
publication. La clé du propriétaire est uniquement destinée à
être utilisée lorsque cela est nécessaire. C'est la clé la plus
puissante car elle peut changer n'importe quelle clé d'un compte, y
compris la clé du propriétaire, et prouver la propriété pendant une
récupération de compte. Idéalement, elle est destinée à être stockée
hors ligne et utilisée uniquement lorsque les clés du compte doivent
être modifiées ou pour récupérer un compte compromis.

Steem facilite également l'utilisation d'un mot de passe principal
qui crypte les trois clés. Des services Web peuvent utiliser un mot de
passe principal qui décrypte et signe avec la clé privée
nécessaire. Les mots de passe principaux peuvent permettre aux
utilisateurs de faire confiance à certains services pour empêcher
le transfert de clés inappropriées sur tous les serveurs,
augmentant ainsi l'expérience utilisateur tout en maintenant un
environnement de signature côté client sécurisé.

\subsection{Autorités Multi Sig}
\label{sec:org77f9e26}
La chaîne de blocs Steem permet à une autorité d'être divisée entre
plusieurs entités, de sorte que plusieurs utilisateurs peuvent
partager la même autorité, ou plusieurs entités sont nécessaires
pour autoriser une transaction afin qu'elle soit valide. Ceci est
fait de la même manière que Bitshares où chaque paire de clés
publique / privée reçoit un poids, et un seuil est défini pour
l'autorité. Pour qu'une transaction soit valide, suffisamment
d'entités doivent signer pour que la somme de leurs poids atteigne
ou dépasse le seuil.

\subsection{Bénéficiaires de récompenses multiples}
\label{sec:org50fc79c}
Pour un poste donné, il peut y avoir un certain nombre de personnes
différentes qui ont un intérêt financier dans la récompense. Cela
inclut l'auteur, les co-auteurs possibles, les référents, les
fournisseurs d'hébergement, les blogs qui ont intégré les
commentaires de la chaîne de blocs et les développeurs
d'outils. Quel que soit le site ou l'outil utilisé pour construire
un article ou un commentaire, vous pourrez définir comment les
récompenses de ce commentaire sont réparties entre les différentes
parties. Cela permet diverses formes de collaboration, ainsi qu'un
moyen pour les plates-formes qui sont construites au-dessus de la
chaîne de blocs Steem pour collecter une partie des récompenses de leurs
utilisateurs.

\subsection{Jetons de média intelligents (Smart Media Tokens)}
\label{sec:org6675685}
Cette couche de protocole est en cours de développement. Son livre
blanc sera posté ici.

\subsection{Récupération de compte volé}
\label{sec:orgf7a2156}
Si le compte d'un utilisateur est compromis, il peut changer de clé
en utilisant sa clé privée. Dans le cas où l'attaquant est capable
de compromettre la clé privée du propriétaire et de changer le mot
de passe, l'utilisateur dispose de 30 jours pour soumettre une clé
privée auparavant fonctionnelle via le processus de récupération de
compte volé de Steem et reprendre le contrôle de son compte. Cela
peut être offert par une personne ou une entreprise qui fournit des
services d'enregistrement à Steem. Il n'est pas obligatoire pour le
registraire de fournir ce service à ses utilisateurs, mais il est
disponible pour augmenter la valeur de l'expérience des
utilisateurs d'un registraire.
\subsection{Sécurité à travers les serrures de temps}
\label{sec:org6319cc4}
Si la clé active ou propriétaire d'un utilisateur est compromise,
l'attaquant aura un accès complet à tous les fonds de son
compte. Parce que les transactions chaîne de blocs sont irréversibles,
les utilisateurs n'ont aucun moyen de récupérer leurs fonds après
avoir été volés. 

La chaîne de blocs Steem permet aux utilisateurs de stocker leurs jetons
STEEM et SBD dans un compte d'épargne, de sorte que les fonds ne
peuvent être retirés qu'après une période d'attente de trois
jours. En outre, STEEM qui est détenu dans le calendrier
d'acquisition de 13 semaines ne peut être retiré à un taux de 1/13
par semaine, après une période d'attente initiale de sept
jours. Ces verrous temporels empêchent un attaquant d'accéder
immédiatement à la totalité des fonds de l'utilisateur, de sorte
que le propriétaire légitime ait le temps de reprendre le contrôle
de son compte avant que tous ses fonds puissent être retirés.
\subsection{Limitation du débit de bande passante pour les opérations sans frais}
\label{sec:org909016a}
Parce que les témoins sont entièrement payés grâce à la génération
de nouveaux jetons, il n'est pas nécessaire de facturer aux
utilisateurs des frais pour alimenter la chaîne de blocs. La seule
raison de facturer des frais aurait pour effet dissuasif d'empêcher
les utilisateurs d'effectuer un nombre déraisonnable de
transactions, ce qui pourrait avoir un impact sur la performance de
la chaîne de blocs.

Afin de placer des limites raisonnables sur l'utilisation du
système, chaque utilisateur dispose d'une bande passante
limitée. Lorsque les utilisateurs effectuent des opérations de
chaîne de blocs telles que les transferts de jetons, l'envoi de
contenu et le vote, ils utilisent une partie de leur bande
passante. Si un utilisateur dépasse sa capacité de bande passante,
il doit attendre d'effectuer des actions supplémentaires jusqu'à ce
que sa bande passante se recharge.

Les limites de bande passante s'ajustent en fonction de
l'utilisation du réseau, de sorte que les utilisateurs disposent
d'une bande passante plus importante lorsque l'utilisation du
réseau est faible. La quantité de bande passante autorisée par un
compte est directement proportionnelle à la quantité de Steem Power
dont dispose un utilisateur, de sorte que les utilisateurs peuvent
toujours augmenter leur bande passante en obtenant une puissance
Steem supplémentaire.
\section{Conclusion}
\label{sec:org8f11644}
Le programme unique de récompenses et d'incitations offert par la
chaîne de blocs et le token de Steem est conçu pour faire de Steem
l'ultime passerelle de crypto-monnaie pour les utilisateurs
ordinaires. La performance de la chaîne de blocs est conçue avec
l'adoption massive de la devise et de la plate-forme en
tête. Lorsqu'il est combiné avec des temps de traitement
ultra-rapides et des transactions sans frais, Steem est positionné
pour devenir l'une des principales technologies de chaîne de blocs
utilisées par les personnes autour du monde.
\end{document}