%%%%%%%%%%%%%%%%%%%%%%%%%%%%%%%%%%%%%%%%%%%%%%%%%%%%%%%%%%%%%%%%%%%%%%%%%%%
%%%% ENGLISH 
%%%%%%%%%%%%%%%%%
% Phrases en anglais
\newcommand{\exEN}[1]{%
  \textcolor{blue}{\textenglish{#1}}%
}

\newcommand{\uks}[1]{% uk sound
  \href{#1}{\includegraphics[width=0.5cm, height=0.3cm]{../img/union-jack-mini}}%
}

\newcommand{\uss}[1]{% us sound
  \href{#1}{\includegraphics[width=0.5cm, height=0.3cm]{../img/us-flag-mini}}%
}

\newcommand{\aus}[1]{% australian sound
  \href{#1}{\includegraphics[width=0.5cm, height=0.3cm]{../img/aussie-flag-mini}}%
}

% Liens vers les bonnes prononciations phonétiques
\newcommand{\properukus}[2]{%
  \begin{center}
    (\uks{#1}\quad \uss{#2})
  \end{center}
}

% Liens pour les drapeaux cliquables à droite au dessus du titre
% de la section
% RCLF: Right Corner Linked Flags
% #1: sound 1, #2: link 1, #3: sound 2, #4: link 2
\newcommand{\RCLF}[4]{%
  \begin{tikzpicture}[remember picture,overlay]
    \node[anchor=east,inner sep=0pt] at (current page text
    area.east|-0,3cm) {#1$\Rightarrow$ \uks{#2}\quad #3$\Rightarrow$ \uss{#4}};
  \end{tikzpicture}
}


% Insertion des liens YouGlish
\newcommand{\youglish}[1]{%
\exEN{For more examples with the word ``#1'' click on the \href{https://youtu.be/sYmJLFEowaY}{flags} below:}%
  \begin{center}%
    (\uks{https://youglish.com/search/#1/uk}\quad\uss{https://youglish.com/search/#1/us}\quad\aus{https://youglish.com/search/#1/aus})%
  \end{center}%
}

% Liens vers mon blog www.doyouspeakenglish.fr
\newcommand{\dyse}[1]{%
  \footnote{Consultez mon
    article complet sur ce son en cliquant sur \url{http://doyouspeakenglish.fr/#1}}%
}

% Liens vers les sons de l'alphabet
% #1 : lettre, #2 : lien interne, #3 : phonétique
\newcommand{\letr}[3]{%
  \oxford{#1} \hyperlink{#2}{\wordref{#1}{#3}}
}


% Liens vers le livre How The Ell Brain Learns
\newcommand{\HTEBL}{%
    ~\href{https://amzn.to/2pSEAsm}{\exEN{How The Ell Brain Learns}}\xspace%
}

% Liens vers le livre a critical introduction to phonetics
\newcommand{\lodge}{%
  \href{https://amzn.to/2K45eHn}{\exEN{A Critical Introduction to Phonetics}}\xspace
}

% Liens vers le livre a manual of english phonetics and phonology
\newcommand{\bs}{%
  \href{https://amzn.to/2HV8BzG}{\exEN{A Manual of English Phonetics and
    Phonology}}\xspace %de \textsc{Paul Burleigh} et \textsc{Peter Skandera}
}

% Liens vers le livre Collins Practical Phonology and Phonetics
\newcommand{\cpp}{%
  \href{https://amzn.to/2HsmuUR}{\exEN{Pratical Phonetics and
      Phonology}}\xspace
  %de \textsc{Beverley S. Collins}
}

% Liens vers le livre Ogden Phonetics
\newcommand{\ogden}{%
  \href{https://amzn.to/2Kc0NKE}{\exEN{An Introduction to English Phonetics}}\xspace
  %de \textsc{Richard Ogden}
}

% Epigraphes maison
\newcommand\juxtaposed[2]%
  {\par\noindent
   \parbox[t]{3cm}{\raggedright#1}%
   \hfill
   \parbox[t]{2cm}{\raggedleft#2}%
   \bigskip\par
  }

%%%%%%%%%%%%%%%%%%%%%%%%%%%%%%%%%%%%%%%%%%%%%%%%%%%%%%%%%%%%%%%%%%%%%%%%%%%%
%%% FRANÇAIS
%%%%%%%%%%%%%%%%%%%%%%%%%%
% Traductions en français
\newcommand{\exFR}[1]{%
  \textcolor{orange}{\textcursive{#1}}%
}

% Liens vers le livre Français Lycée
\newcommand{\FL}{%
  ~\href{https://amzn.to/2vWXuUu}{\exFR{français lycée}}\xspace%
  %de Pierre Brunel\xspace%
}

% Liens vers le livre Grévisse de l'Enseignant
\newcommand{\GE}{%
  ~\href{https://amzn.to/2I0vT9J}{\exFR{Grévisse de l'enseignant}}\xspace%
}

% Liens vers le livre Honni soit qui mal y pense
\newcommand{\HSQMYP}{%
  ~\href{https://amzn.to/2Fp3KUq}{\exFR{Honni soit qui mal y pense}}%
}

% Liens vers le livre Anglais débutant
\newcommand{\ad}{%
  ~\href{https://amzn.to/2F5OhZa}{\exFR{Anglais débutant 1 leçon par
      jour pendant 3 mois}}\xspace%
}

% Liens vers le livre Bescherelle l'anglais pour tous
\newcommand{\besch}{%
  ~\href{https://amzn.to/2HiDBfI}{\exFR{Bescherelle l'anglais pour tous}}\xspace%
}


% \newcommand{\blocFR}[2]{%
%   \begin{bclogo}[couleur=blue!30, arrond=0.1, logo=\bcdfrance]{#1}
%     \exFR{#2}
%   \end{bclogo}
% }

%%%   BLOCS
\newmdenv[linecolor=red, frametitle=Exemple]{exemple}
\newmdenv[linecolor=red, frametitle=Citation]{citations}

% \newmdenv[%
%           linecolor=red,%
%           roundcorner=5pt,%
%           backgroundcolor=green!20,%
%           frametitle=Traduction,%
%           frametitlerule=true,%
%           frametitlebackgroundcolor=blue!20,%
%           ]{traduction}

\global\mdfdefinestyle{tradstyle}{%
  backgroundcolor=blue!20,%
  linecolor=red,%
  middlelinewidth=2pt,%
  frametitlerule=true,%
  apptotikzsetting={%
    \tikzset{mdfframetitlebackground/.append style={%
        shade,left color=blue!20, right color=blue}%
    }%
  },%
  frametitlerulecolor=green!60,%
  frametitlerulewidth=1pt,%
  innertopmargin=\topskip,%
}


\newcommand{\tradbloc}[4]{%
  {\flushleft
    \begin{mdframed}[%
      style=tradstyle,%
      frametitle={%
        \exFR{%
          \href{#1}{#2},%
          #3%
        }%
      }%
      ]
      \exFR{<<~#4~>>} 
    \end{mdframed}%
  }
}
              
\global\mdfdefinestyle{citestyle}{%
  backgroundcolor=orange!20,%
  linecolor=red,%
  middlelinewidth=2pt,%
  frametitlerule=true,%
  apptotikzsetting={%
    \tikzset{mdfframetitlebackground/.append style={%
        shade,left color=orange, right color=orange!20}%
    }%
  },%
  frametitlerulecolor=green!60,%
  frametitlerulewidth=1pt,%
  innertopmargin=\topskip,%
}

\newcommand{\citebloc}[4]{%
  {\flushleft
    \begin{mdframed}[%
      style=citestyle,%
      frametitle={%
        \exEN{%
          \href{#1}{#2},%
          #3%
        }%
      }%
      ]
      \exEN{``#4''} 
    \end{mdframed}%
  }
}

\mdfsetup{%
  middlelinecolor=red,%
  middlelinewidth=2pt,%
  backgroundcolor=green!20,%
  roundcorner=10pt%
}


% Longues citations : #1 largeur #2 source #3 punct 
% \newenvironment{\bigquote}[1]
%  {\begin{center}%
%     \shadowbox{%
%       \begin{minipage}{#1\textwidth}\end{minipage}}%
%  }%
%  {\end{center}}
    


%%%%%%%%%%%%%%%%%%%%%%%%%%%%%%%%%%%%%%%%%%%%%%%%%%%%%%%%%%%%%
% Phonétique
%%%%%%%%%%%%%%%%%%%%%%%%%%%%%%%%%%%%%%%%%%%%%%%%%%%%%%%%%%%%%
% Mots écrits en transcription phonétiques (donc PHONèMes)
\newcommand{\phonm}[1]{%
  \textcolor{teal}{/#1/}
}

% Liens vers wordreference
\newcommand{\wordref}[2]{%
  \href{http://www.wordreference.com/enfr/#1}{\phonm{#2}}
}

% Liens vers oxforddictionnaries
\newcommand{\oxford}[1]{%
  \href{https://en.oxforddictionaries.com/definition/#1}{\exEN{#1}}
}

% Liens vers cambridge
\newcommand{\cambridge}[1]{%
  \href{https://dictionary.cambridge.org/fr/dictionnaire/anglais/#1}%
  {\exEN{#1}}
}

% Exemple de mot avec Oxford
\newcommand{\exMotOx}[2]{%
  Le mot \exEN{\href{http://www.wordreference.com/enfr/#1}{#1}} qui
  s'écrit phonétiquement
  \href{https://en.oxforddictionaries.com/definition/#1}{\phonm{#2}}
}

% Exemple de mot avec Cambridge
\newcommand{\exMotCam}[2]{%
  Le mot \exEN{\href{http://www.wordreference.com/enfr/#1}{#1}} qui
  s'écrit phonétiquement
  \href{https://dictionary.cambridge.org/fr/dictionnaire/anglais/#1}{\phonm{#2}}
}

% Son isolé ou phone
\newcommand{\phon}[1]{%
  \textcolor{teal}{[#1]}
}

% Sons écrits en transcription phonétiques
\newcommand{\son}{%
  \textcolor{teal}{son}
}

% Notations
\newcommand{\notation}{%
  \begin{center}
    {\Large Attention aux \hyperlink{notation}{notations} !}
  \end{center}
}
% Présentation du nom du son avec les liens vers le blog et vers
% wikipédia
% \newcommand[7]{\sn}{% sound name
%   Ce \textcolor{red}{son} a pour nom technique\dyse{#1} :% #1: lien
%                                 % vers le blog
%   %
%   \begin{itemize}%
%   \item \exEN{#2\CW{#3}.}% #2: sound name, #3: wiki EN
%   \item \exFR{#4\CW{#5}.}% #4: nom du son, #5: wiki FR sinon blog voir
%                          % package ifthen pour gérer ça
%   \end{itemize}%
%   %
%   \indicsound%
%   %
%   \properukus{#6}{#7}% #6: UK YT, #7: US YT
% }

% Références à des sons
% ancre
\newcommand{\sonancre}[1]{%
  \hypertarget{#1}{C'est le \textcolor{red}{son}~}%
}
% lien
\newcommand{\sonref}[2]{%
  \hyperlink{#1}{\son[~]{#2}}%
}

% Mini-table des matières ciblées
\newcommand{\tdm}[9]{%
  % #1 Type colonnes : lcr (left center right)
  % #2 ref ligne 1, colonne 1 ici sone
  % #3 ref ligne 1, colonne 2 ici sonae
  % #4 ref ligne 2, colonne 1 ici sonenv
  % #5 ref ligne 2, colonne 2 ici sonenvlong
  % #6 ref ligne 3, colonne 1 ici sonalong
  % #7 ref ligne 3, colonne 2 ici enenvomegaenv
  % #8 ref ligne 4, colonne 1 ici omegaenvenenv
  % #9 ref ligne 4, colonne 2 ici eeteenv
  \begin{center}
    \begin{table}[h]
      \centering
      \begin{tabular}{#1||#1}
        \ref{sec:#2} page~\pageref{sec:#2}
        & \ref{sec:#3} page~\pageref{sec:#3} \\
        \\
        \ref{sec:#4} page~\pageref{sec:#4}
        & \ref{sec:#5} page~\pageref{sec:#5} \\
        \\
        \ref{sec:#6} page~\pageref{sec:#6}
        & \ref{sec:#7} page~\pageref{sec:#7} \\
        \\
        \ref{sec:#8} page~\pageref{sec:#8}
        & \ref{sec:#9} page~\pageref{sec:#9} 
      \end{tabular}
      \caption{Liste des symboles problématiques}
      \label{tab:notations}
    \end{table}
\end{center}
}

% Citations en latin
\newcommand{\exLT}[1]{%
  \textcolor{olive}{\textlatin{#1}}%
}

% Consultation Wikipédia
\newcommand{\CW}[1]{%
  \footnote{Consulter \href{#1}{Wikipédia} pour plus de détails.}%
}

% Son nom technique est
\newcommand{\tn}[2]{% technic name
  \par Son nom technique est~:%
  \begin{center}%
    \shadowbox{\textcolor{green}{\underline{\textcolor{magenta}{#1}}}}\CW{#2}.%
  \end{center}%
  
}

% Flags
\newcommand{\flags}{%
  \begin{itemize}%
  \item \exEN{Click on the flags below to check how to produce the sounds.}%
  \item \exFR{Cliquez sur les drapeaux ci-dessous afin de vérifier la façon de produire les sons correctement.}%
  \end{itemize}%
}

% Bla bla pour annoncer le nom du son
\newcommand{\indicsound}{%
  Son nom indique la façon de le produire.%
  \flags%
}

% Diphthongs
% #1 : voyelle de gauche #2 : voyelle de droite
% #3 : diphtongue résultante
% #4 : lien url vers l'article de blog
% #5 : liens éventuels vers les sons élémentaires
% #6 : l'ancre
\newcommand{\diph}[6]{
  \hypertarget{#6}{Cette \gls{diph}} est la combinaison\footnote{#5} de la voyelle
  \phon{#1} et de la voyelle \phon{#2}. D'une certaine manière on peut
  écrire <<~l'équation~>>~:%
  \begin{center}%
    \shadowbox{\phon{#1} + \phon{#2} = \phon{#3}}%
  \end{center}%
  Consultez mon article de
  \href{http://doyouspeakenglish.fr/#4}{blog}\footnote{Consulter
    l'article de blog \url{http://doyouspeakenglish.fr/#4}} sur le sujet pour en savoir plus.%
}

\renewcommand{\mkcitation}[1]{\footnote{#1}}
\renewcommand{\mktextquote}[6]{#1#2#6#4#3#5}

\newcommand{\up}[1]{\textsuperscript{#1}}

%%%%%%%%%%%%%%%
% Speech avant une section de voyelles anglaises
%%%%%%%%%%%%%%%
\newcommand{\speech}[2]{%
  Il y a \textcolor{green}{#1} \textcolor{teal}{#2} dans la langue anglaise et pour chaque \textcolor{teal}{son}, je vous~%
  proposerai au moins \textcolor{green}{4}~%
  exemples. Pour chaque \textcolor{teal}{son} je commencerai par vous donner sa définition~%
  technique\footnote{Ainsi que sa traduction, néanmoins il est important~%
    de prendre l'habitude d'éviter de traduire afin de~%
    véritablement s'imprégner de la langue (ce conseil est valable pour~%
    l'apprentissage de n'importe quelle langue).} parce que cette dernière~%
  explique la façon de produire le \textcolor{teal}{son}. Vous aurez également deux liens~%
  externes :%
  %
  \begin{enumerate}%
  \item Le premier lien\footnote{Accessible en cliquant sur le drapeau du~%
      Royaume Uni.} vous permettra de consulter un article de     \href{http://doyouspeakenglish.fr/}{mon blog}
    avec (au moins) une vidéo qui~%
    propose des exemples supplémentaires avec une prononciation~%
    typiquement anglaise qu'on appelle l'accent \acrfull{rp}\footnote{La
      prononciation appelée BBC accent ou
      \href{https://en.wikipedia.org/wiki/Received_Pronunciation}{\acrlong{rp}}.}.%
  \item Le second lien\footnote{Accessible en cliquant sur le drapeau~américain.}~%
    vous permettra de consulter un article de \href{http://doyouspeakenglish.fr/}{mon blog} avec (au moins)
    une vidéo qui propose des~%
    exemples supplémentaires avec une prononciation typiquement
    américaine qu'on appelle l'accent \acrfull{ga}\footnote{Aussi appelée \href{https://en.wikipedia.org/wiki/General_American}{\acrlong{ga}}.}.%
  \end{enumerate}%
  %
  La structure de chaque section sera toujours la même pour chaque \textcolor{teal}{son} :%
  \begin{enumerate}%
  \item Des exemples de mots afin d'illustrer le \textcolor{teal}{son}. Pour chaque mot~%
    vous disposerez d'un lien externe vers un dictionnaire~%
    \exEN{english}-\exFR{français} qui vous proposera plusieurs prononciations, la~%
    \gls{phon}, la traduction\footnote{En fait
      plusieurs traductions,~avec~parfois~plusieurs registres de langue (y compris l'argot).}, et enfin~%
    plusieurs exemples (en \exEN{english} et leurs traductions en \exFR{français}).%
  \item Une transcription du mot selon la norme de l'\acrshort{api} ou \acrshort{ipa} en~%
    anglais\footnote{À partir de maintenant on utilisera la terminologie~%
      anglaise.} avec un lien externe vers un dictionnaire~%
    \exEN{english}-\exEN{english} qui vous fournira une définition\footnote{En fait~%
      plusieurs selon les sens du mot.}, des exemples d'utilisation du~%
    mot, l'origine du mot, et enfin sa \gls{phon} anglaise.%
  \item Des exemples de phrases utilisant le mot. Dans chaque phrase~%
    vous trouverez au moins deux liens externes\footnote{Parfois plusieurs,~%
      souvent vers des vidéos mais parfois vers des textes de chansons.}~%
    qui vous permettront de voir le mot ou groupe de mots utilisé(s) par~%
    des locuteurs natifs\footnote{Souvent américains mais pas toujours.}.%
  \item Des exemples complémentaires\footnote{Avec cette fois en bonus des~%
      liens vers des prononciations typiquement australiennes en plus des~%
      prononciations anglaises et américaines.} avec cette fois la~%
    possibilité de voir le texte qui s'affiche à l'écran en temps réel\dots  %
  \end{enumerate}%
}