\chapter{Pourquoi la phonétique ?}\label{chap:phonetic}

\href{https://fr.wikipedia.org/wiki/Simon_Sinek}{Simon Sinek} qui est
un très bon motivateur explique très bien dans son livre
\href{https://amzn.to/2qY8uMD}{\exEN{Start with why}} que le
<<~Pourquoi ?~>> est la question fondamentale qu'il faut se poser en
premier. Je vous recommande aussi de regarder cette \href{https://youtu.be/hER0Qp6QJNU}{vidéo} dans
laquelle il explique le problème des \exEN{millenials}\footnote{Et ce problème ne
  concerne pas uniquement les gens nés à partir des années 80, il concerne
  l'ensemble de la société.} et dans un registre légèrement différent je vous
recommande également mon \href{http://doyouspeakenglish.fr/the-millenials/}{article}\footnote{\url{http://doyouspeakenglish.fr/the-millenials/}} de blog sur ces sujets.  

\exEN{Why Phonetics?} est précisément le premier chapitre du livre
\lodge de Ken Lodge. Livre dont je vais vous proposer quelques extraits dans ce
chapitre. Extraits que nous analyserons et traduirons.

En voici un premier extrait :


\begin{center}
\begin{mdframed}[style=citestyle, frametitle={Extrait de~\cite{lodge}}]
  \exEN{``The reasons for the study of phonetics should be made clear
      at the outset. This chapter is intended to set out the
      reasons why linguists (and any other people interested in
      spoken language of any kind) need phonetics as a tool of
      investigation.''}
\end{mdframed}
\end{center}
  

Ce qui donne en français :

\begin{center}
\begin{mdframed}[style=tradstyle, frametitle={\exFR{Traduction} de l'extrait ci-dessus}]
  \exFR{<<~Les raisons de l'étude de la phonétique devraient être
    précisées dès le départ. Ce chapitre vise à exposer les raisons
    pour lesquelles les linguistes (et toute autre personne intéressée
    par la langue parlée, quelle qu'elle soit) ont besoin de la
    phonétique comme outil d'investigation.~>>} 
\end{mdframed}  
\end{center}

Le ton est donné ! La phonétique est un outil nécessaire. Nous allons
explorer les explications que notre ami~\cite{lodge} propose. Et je vous
recommande déjà vivement l'acquisition de son ouvrage.

\newpage
\minitoc
\newpage

\section{Comment décrivons-nous la parole ?}

Le titre de cette section est la traduction du titre de section de son
livre. Il est à noter que l'anglais n'a pas de mot véritablement
équivalent au mot parole.

En effet, le titre original utilisait le mot \exEN{speech} qui sert
aussi à traduire le mot discours. En français aussi le mot discours
peut servir aussi bien pour l'écrit que pour l'oral.

Néanmoins le mot parole est plus précis commme nous le reverrons un
peu plus tard.

Revenons à notre section dont je vous propose un premier extrait ci-dessous pour commencer :

\begin{center}
\begin{mdframed}[style=citestyle, frametitle={Extrait de~\cite{lodge}}]
  \exEN{''Traditional education largely ignores spoken language; even
    in drama and foreign language learning, little attention is paid
    to the details of speech in an objective way. We, therefore, need
    a method of describing speech in objective, verifiable terms, as
    opposed to the lay approaches which typically describe sounds as
    'hard', 'soft', 'sharp' and so on, which can only be properly
    understood by the person using such descriptions. Such an approach
    to any subject of study is totally subjective: since only the
    person carrying out the descriptions can understand them, other
    people are expected to be 'on the same wavelength' and clever
    enough to follow them. So, if we are to observe and describe
    speech in any meaningful way, we need some kind of objectively
    verifiable way of doing so. In fact, there are three ways of
    approaching the task.''}
\end{mdframed}  
\end{center}


La première partie de la première phrase est déjà assez éloquente avant même
que je ne vous livre la traduction (je pense que) vous avez déjà compris de quoi il
s'agit\dots{} l'oralité est ignorée\footnote{D'ailleurs,
  n'appelle-t-on pas <<~peuples primitifs~>> précisément ceux qui
  n'ont pas développé de système d'écriture ?}, niée alors même qu'elle est le
fondement du langage !

Observons désormais une traduction complète de cet extrait :

\begin{center}
\begin{mdframed}[style=tradstyle, frametitle={\exFR{Traduction} de l'extrait ci-dessus}]
    \exFR{L'éducation traditionnelle ignore en grande
    partie la langue parlée; même dans l'art dramatique et
    l'apprentissage des langues étrangères, peu d'attention est
    accordée aux détails de la parole d'une manière objective. Nous
    avons donc besoin d'une méthode de description de la parole en
    termes objectifs et vérifiables, opposés aux approches profanes qui
    décrivent généralement les sons comme ``durs'', ``doux'', ``forts'' et
    ainsi de suite, qui ne peuvent être correctement compris que par la
    personne utilisant de telles descriptions. Une telle approche pour
    n'importe quel sujet d'étude est totalement subjective : puisque
    seul la personne qui exécute les descriptions peut les comprendre,
    les autres sont censés être sur la même longueur d'onde et
    assez intelligents pour les suivre. Donc, si nous devons observer
    et décrire la parole de quelque manière significative, nous avons
    besoin d'un type de manière objectivement vérifiable de le
    faire. En fait, il y a trois façons d'approcher la tâche.} 
\end{mdframed}  
\end{center}


Que ceux qui trouvent que les mots sont un peu trop pompeux ou les
formules trop alambiquées se rassurent. Ce que notre ami~\cite{lodge} nous
dit ici c'est tout simplement :
\begin{enumerate}
\item Que l'oralité a majoritairement toujours été considéré comme
  inférieure à l'écrit. Et c'est toujours le cas malheureusement. Il
  faut savoir que l'écriture n'est véritablement répandue que depuis
  l'invention de l'imprimerie. Et il faut attendre le 19ème siècle
  pour que le <<~peuple~>> accède à l'école et donc à la lecture et
  l'écriture. Par conséquent, l'oralité est ultra prépomdérante dans
  l'histoire de l'humanité y compris dans les pays où certains
  essaient d'imposer la supériorité de l'écrit. J'ajouterais que
  l'oralité est une prédisposition biologique ce qui n'est pas le cas
  de l'écrit. Et comme le dit très bien notre ami \href{https://youtu.be/1wsYbQxppRc}{Michel Serres}\footnote{\url{https://youtu.be/1wsYbQxppRc}},
  l'invention de l'écriture est une réduction de la mémoire ou plutôt
  une délégation de mémoire. Que ça soit la Bible ou les récits
  mythologiques Grecs\footnote{Ou d'ailleurs.}, la plupart des textes
  dits <<~sacrés~>> ont d'abord été des mémorisations
  orales\footnote{Ils n'avaient ni la télévision ni les réseaux
    sociaux du coup ils utilisaient davantages leurs réseaux
    neuronaux.}.
\item Que si l'on veut étudier le discours oral on a besoin d'outils
  objectifs. En effet, on ne peut pas se contenter d'expressions
  vagues comme les adjectifs cités dans son extrait. Qu'est-ce qu'un
  \son <<~dur~>> ? Dur à dire si j'ose le petit jeu de mot. Bien sûr
  c'est pratique lorsqu'on veut faire de l'art qui part essence est
  subjectif mais si l'on veut étudier scientifiquement la chose\dots{}
  et bien on a besoin d'une terminologie précise. En clair, notre ami
 ~\cite{lodge} nous dit que l'étude du discours oral doit être traitée comme
  toutes les autres sciences et pour ce faire il lui faut en quelques
  sorte une mathématisation ou si vous préférez une modélisation à
  l'aide de symboles précis.
\end{enumerate}

Voyons à présent quelles sont les trois façons qu'il décrit pour
s'atteler à cette tâche.

\newpage

\section{Première branche de la phonétique}\label{sec:phon-t1}

Commençons par la première branche de la phonétique.

\begin{center}
\begin{mdframed}[style=citestyle, frametitle={Extrait de~\cite{lodge}}]
    \exEN{``What is speech exactly? The expression 'a lot of hot air' is
    rather a good starting point. Speech is made by modulating air in
    various ways inside our bodies. The organs of speech -- the lungs,
    throat, tongue, nose, lips and so on, which we shall discuss in
    detail in Chapter Two -- can be moved into many different
    configurations to produce the different sounds we perceive when
    listening to spoken language. A study of the ways in which these
    \textbf{articulators} of speech behave is called
    \textbf{articulatory phonetics}. In this book the detailed
    investigation of articulation will take up in eight out of the
    nine chapters.''}
\end{mdframed}  
\end{center}


Ce qui donne en français :

\begin{center}
\begin{mdframed}[style=tradstyle, frametitle={\exFR{Traduction} de l'extrait ci-dessus}]
  \exFR{Qu'est-ce que la parole ? L'expression ``beaucoup
    d'air chaud'' est plutôt un bon point de départ. La parole est
    faite en modulant l'air de diverses façons à l'intérieur de notre
    corps. Les organes de la parole -- les poumons, la gorge, la langue,
    le nez, les lèvres etc\dots{}, dont nous allons discuter
    en détail dans le chapitre deux -- peuvent être déplacés dans de
    nombreuses configurations différentes pour produire les différents
    sons que nous percevons quand nous écoutons la langue parlée. Une étude
    des façons dont ces \textbf{articulateurs} du discours se
    comportent est appelée \textbf{phonétique articulatoire}. Dans ce
    livre, l'enquête détaillée sur l'articulation occupera huit des
    neuf chapitres.}
\end{mdframed}  
\end{center}

De la même façon que notre ami~\cite{lodge} l'a fait dans son livre, ici
aussi nous nous occuperons principalement de la \gls{phon} articulatoire.

En bref, cela signifie que l'on va se focaliser sur la
façon de produire les sons. D'ailleurs, même si j'ai donné tous les
noms techniques, il n'est nullement question de théorie pour pratiquer
la \gls{phon} et améliorer sa prononciation, tout est question
d'écoute et de répétition. Mais comme l'a très justement souligné
notre ami~\cite{lodge}, l'utilisation d'une terminologie précise a pour but
et avantage de nous permettre de nous y retrouver et de voir les
différences entre les sons.

Juste pour la culture poursuivons la description de notre ami~\cite{lodge}. 

\newpage

\section{Deuxième branche de la phonétique}\label{sec:phon-t2}

Passons à la deuxième branche de la \gls{phon}.

\begin{center}
\begin{mdframed}[style=citestyle, frametitle={Extrait de~\cite{lodge}}]
  \exEN{``Basically, air is pushed out of the body and disturbs the outside
  air between the speaker and anyone in the vicinity who can hear
  him/her. These disturbances are known as \textbf{pressure
    fluctuations}, which in turn cause the hearer's eardrum to
  move. The molecules of the air move together and then apart in
  various ways, producing a \textbf{sound wave}. The study of the
  physical nature of sound waves is \textbf{acoustic
    phonetics}. We shall look at this aspect of speech and the
  relationship of articulation to acoustic effects in Chapter Nine.''}
\end{mdframed}  
\end{center}

    
Ce qui donne en français :

\begin{center}
\begin{mdframed}[style=tradstyle, frametitle={\exFR{Traduction} de l'extrait ci-dessus}]
  \exFR{Fondamentalement, l'air est expulsé du corps et perturbe
    l'air extérieur entre l'orateur et toute personne dans le
    voisinage qui peut l'entendre. Ces perturbations sont
    appelées \textbf{fluctuations de pression}, qui à leur tour
    provoquent le mouvement du tympan de l'auditeur. Les molécules de
    l'air se déplacent ensemble puis se séparent de diverses manières,
    en produisant une \textbf{onde sonore}. L'étude de la nature
    physique des ondes sonores est l'\textbf{acoustique
      \gls{phon}}. Nous allons regarder cet aspect du discours et la 
    relation d'articulation aux effets acoustiques dans le chapitre neuf.}
\end{mdframed}  
\end{center}


Ceci est très intéressant et nous conduirait à une analyse physique du
\son qui serait également valable pour la musique. D'ailleurs, en
général, lorsqu'on parle d'acoustique la première chose qui vient à
l'esprit c'est la musique\footnote{À moins que je ne sois le seul à
  penser comme ça.}. Néanmoins, pour ce premier livre je me suis
restreint à la \gls{phon} articulatoire. Et puis je vais vous faire
une petite confidence cette partie sur la \gls{linguistic} est celle que
j'ai écrit en dernier parce que je me suis dit que faire une
introduction à la \gls{phon} sans même la définir
sérieusement\footnote{Ben oui la petite définition Wikipédia c'est
  bien gentil mais c'est un peu léger.} ça serait un peu abuser. Du
coup ça m'a mis l'eau à la bouche pour un livre plus conséquent sur le
sujet. Mais pour l'instant mon approche est purement pragmatique,
comme utiliser ces outils pour améliorer et accélérer l'acquisition
d'une langue (ici l'anglais).

\newpage

\section{Troisième branche de la phonétique}\label{sec:phon-t3}

Observons maintenant la dernière branche de la \gls{phon}.

\begin{center}
\begin{mdframed}[style=citestyle, frametitle={Extrait de~\cite{lodge}}]
  \exEN{``The third way of considering speech, \textbf{auditory
      phonetics}, deals with the ways in which speech affects and is
    interpreted by the hearer(s). This aspect of the investigation of
    speech will not be considered in this book.''}
\end{mdframed}  
\end{center}

Ce qui donne en français :

\begin{center}
\begin{mdframed}[style=tradstyle, frametitle={\exFR{Traduction} de l'extrait ci-dessus}]
  \exFR{La troisième façon de considérer la parole, la
    \textbf{\gls{phon} auditive}, traite des façons dont la
    parole affecte et est interprétée par le ou les auditeur(s). Cet
    aspect de l'étude de la parole ne sera pas considéré dans ce livre.}
\end{mdframed}  
\end{center}


Maintenant je pense que vous avez un aperçu assez intéressant de ce
qu'est la \gls{phon}. En particulier, même si nous nous restreindrons
à la \gls{phon} articulatoire, il me semblait important de montrer la
rigueur scientifique qu'il y a derrière cette discipline qui ne se
résume pas aux pauvres petites transcriptions que l'on voit dans les
dictionnaires et dont les professeurs de langues ne nous ont jamais
parlé ni au collège ni au lycée.

Bien entendu notre cher ami~\cite{lodge} ne s'adresse pas au
débutant\footnote{Encore qu'avec un peu de patience et de persévérance
tout le monde peut lire et comprendre son livre.}, mais je pense que sa
description est assez claire.

Nous en avons fini avec la description de la théorie \gls{phon}. Avant
de rentrer dans le vif du sujet je vais tout de même faire un autre
petit chapitre similaire. Pourquoi ? Parce que l'autre source que j'ai
sélectionné est un manuel à l'usage de locuteurs non anglophones donc
son approche pourra nous être utile. Et de plus cette fois on brosse
un portrait général de la \gls{linguistic}.

\newpage
\minitoc
\newpage

