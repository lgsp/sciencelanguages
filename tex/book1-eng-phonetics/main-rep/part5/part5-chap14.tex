\chapter{Mes autres projets}\label{chap:projets}

\section{Pourquoi \TeX{} et d'où ça me vient ?}\label{sec:tex}

L'écriture de cet ouvrage m'a demandé un nombre incalculable d'heures
de travail. Et cela en un temps relativement réduit puisque j'ai dû y
passer environ 3 mois. C'est une véritable aventure qui m'a replongé
dans l'univers passionnant de la programmation \TeX{} que j'avais
découvert lorsque j'étais étudiant en mathématiques. À l'époque je
devais rédiger mon mémoire de travail d'étude et de recherche sur les
structures non euclidiennes des fonctions holomorphes à plusieurs
variables complexes\footnote{Quelque chose de ce genre là
  \url{https://www.math.u-psud.fr/~merker/Enseignement/Master-2-2017-2018/hartogs-separe.pdf}
pour ceux que ça intéressent ou encore \url{https://www.math.u-psud.fr/~perrin/Conferences/Romilly.pdf}.}\dots{} oui il s'agit bien de mots qui font tous partis de la
langue française mais qui sont inintelligibles pour le commun des
mortels. Preuve que l'on peut parler une autre langue tout en
utilisant officiellement la même langue. Mais je m'égare, ce n'est pas
l'objet de ce livre que de traiter de mathématiques qui sont selon moi
un langage tout aussi passionnant que le langage humain\footnote{Elles
feront l'objet de prochains livres à venir une fois que j'en aurais
fini avec ce premier cycle sur la \gls{phon} articulatoire.}. \TeX{}
est un langage qui a été inventé par un enseignant-chercheur en
mathématiques et informatique et il est désormais utilisé par de
nombreux individus qui n'ont pas spécialement d'affinité avec les
mathématiques comme par exemple des chercheurs en sciences
humaines. En particulier, la communauté des linguistes s'est plutôt
bien investie\footnote{Comme on peut le voir ici
  \url{http://cl.indiana.edu/~md7/08/latex/slides.pdf} par exemple ou
  encore là \url{https://www.tug.org/TUGboat/tb25-1/peter.pdf} et
  enfin de façon plus succinte sur ce \href{https://allthingslinguistic.com/post/50042310246/what-is-latex-and-why-do-linguists-love-it}{blog}.} dans l'appropriation de cet \href{http://stefanocoretta.altervista.org/xelatex-linguistics/}{outil}.

\section{Mon projet Babel}\label{sec:babel}

D'ailleurs je suis d'ores et déjà motivé pour poursuivre mon
exploration de l'univers merveilleux de la \gls{phon}\footnote{Et
  aussi de la \gls{linguistic} en général parce que je trouve ça très
  intéressant et séduisant pour l'esprit et la pratique qui est
  beaucoup plus accessible au commun des mortels que les
  mathématiques, ce qui en fait un outil plus agréable pour échanger
  avec ses semblables bipèdes qu'on appelle encore \href{https://youtu.be/dGiQaabX3_o}{Homo} Sapiens
  \href{https://youtu.be/npfShBTNp3Q}{Sapiens}.}. Comme je l'ai déjà dit plus haut dans l'ouvrage l'\acrshort{api} ne
concerne pas que l'anglais, il concerne toutes les langues
humaines. Bien qu'il me serait difficile de traiter toutes les langues
du monde, je suis déjà en cours de rédaction d'une introduction à la
\gls{phon} française. Et une bonne partie des sons et symboles
utilisés pour l'anglais sont valables également pour le
français (donc si ça vous intéresse vous ne serez pas dépaysé). Parmi les \href{https://www.youtube.com/playlist?list=PLfKvL-VUSKAnkBk88BAb3oq1MlGVnhwcY}{autres
  langues} que je compte explorer comme je l'ai déjà fait à plusieurs
reprises il y a :
\begin{itemize}
  \item  l'\href{https://www.youtube.com/playlist?list=PLfKvL-VUSKAnM9MWJT9F1z1QZTdb73i7r}{allemand}
  \item
    l'\href{https://www.youtube.com/playlist?list=PLfKvL-VUSKAkXu2x3Fp74QxxYUVP43haA}{arabe}
  \item le
    \href{https://www.youtube.com/playlist?list=PLfKvL-VUSKAl4R0Mh7sKvQjqCsiEEa6D9}{chinois}
  \item
    l'\href{https://www.youtube.com/playlist?list=PLfKvL-VUSKAm_p6ikI_pTbxNuHco73REt}{espagnol}
  \item
    l'\href{https://www.youtube.com/playlist?list=PLfKvL-VUSKAkbDhpbtXc7RdroMBBeTJx0}{hébreu}
  \item le
    \href{https://www.youtube.com/playlist?list=PLfKvL-VUSKAn0zUUPYsMDd8_1J_UtfRxh}{portugais}
  \item le \href{https://www.youtube.com/playlist?list=PLfKvL-VUSKAk0YrJ3rV6cBj-w6rNCeOJB}{russe}
  \end{itemize}

Pour chaque langue je proposerai des connexions avec la \gls{phon}
anglaise et la \gls{phon} française. C'est pour cette raison que j'ai
commencé par l'anglais et que j'enchaîne avec le français.

