\chapter{Trucs et astuces}

Dans cette partie je vais vous livrer un ensemble de 10 trucs et astuces
pour améliorer votre niveau en anglais. C'est parti pour le top 10 !

\begin{enumerate}
  \item \'Ecoutez la bonne prononciation ! Normalement si vous avez
  cliqué sur les nombreux liens que je vous ai fourni vous avez dû au
  moins écouter les prononciations des dictionnaires Cambridge, Oxford
  et Wordreference. Ajouté à cela vous avez églament dû écouter des
  vidéos YouTube avec des américains et des britanniques. Et enfin des
  australiens grâce à YouGlish. Et bien je vous en donne encore deux autres qui
  s'appellent \href{http://www.howjsay.com/}{howjsay}\footnote{\url{http://www.howjsay.com/}}
  et \href{https://youtu.be/K4BMVam-qKs}{forvo}\footnote{\url{https://youtu.be/K4BMVam-qKs}}.
  \item Enregistrez-vous ! Que ça soit avec la webcam de votre
  ordinateur ou simplement le micro, avec votre smartphone, avec des
  apps comme \href{https://itunes.apple.com/fr/app/parlez-traduisez-traducteur/id804641004?mt=8}{parlez et traduisez} ou avec
  \href{https://vocaroo.com/}{Vocaroo}\footnote{\url{https://vocaroo.com/}}.
  Enregistrez votre voix et/ou idéalement votre visage également car
  les sons se construisent avec les muscles du visage.
  \item Parmi les vidéos que j'ai partagé avec vous, vous avez
    probablement dû remarquer que certaines revenez régulièrement
    comme par exemple Rachel. Il se trouve qu'elle a fait une
    excellente
    \href{https://www.youtube.com/playlist?list=PL27A5D7DE7D02373A}{playlist}
    que je vous recommande dans laquelle elle présente les exercices
    de Benjamin Franklin.
  \item Mettez en application les 30 astuces que je partage dans la
    playlist \href{https://www.youtube.com/playlist?list=PLfKvL-VUSKAnf4oZzkI3q24X4FJrGzcGr}{30 façons d'apprendre l'anglais}\footnote{\url{https://www.youtube.com/playlist?list=PLfKvL-VUSKAnf4oZzkI3q24X4FJrGzcGr}}.
  \item Testez vos connaissances de la culture anglo-saxonne à l'aide
    de ce petit \href{https://www.quizz.biz/quizz-436703.html}{quizz}\footnote{\url{https://www.quizz.biz/quizz-436703.html}}.
  \item Testez vos connaissances des expressions idiomatiques très
    présentes en anglais avec ce    \href{https://www.quizz.biz/quizz-918091.html}{quizz}\footnote{\url{https://www.quizz.biz/quizz-918091.html}}.
  \item Amusez-vous ! Par exemple avec ce site de jeux conçus
    spécialement pour apprendre l'anglais
    \url{https://www.gamestolearnenglish.com/} ou si vous voulez
    monter en puissance
    \href{https://www.boatloadpuzzles.com/playcrossword}{Boatload
      Puzzles}\footnote{Attention ils sont très durs à finir !
      \url{https://www.boatloadpuzzles.com/playcrossword}} ou encore
    \url{http://games.usatoday.com/category/puzzles/}.
  \item Faites attention aux erreurs communes des francophones comme
      exipliqué    \href{https://pronunciationstudio.com/french-speakers-english-pronunciation-errors/}{ici}\footnote{\url{https://pronunciationstudio.com/french-speakers-english-pronunciation-errors/}}.
    \item Tenez compte des différences malgré les points communs
        comme expliqué dans cet        \href{http://esl.fis.edu/grammar/langdiff/french.htm}{article}\footnote{\url{http://esl.fis.edu/grammar/langdiff/french.htm}}.
      \item Prenez garde aux nombreuses nuances qui existent en
        anglais mais pas en français comme expliqué \href{https://www.eupedia.com/europe/missing_words_french.shtml}{ici}\footnote{\url{https://www.eupedia.com/europe/missing_words_french.shtml}}.
\end{enumerate}

\newpage

