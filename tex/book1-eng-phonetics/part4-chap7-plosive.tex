\chapter{Plosive (un bloc d'air complet)}\label{chap:plosive}

\speech{7}{consonnes \exFR{occlusives\CW{https://fr.wikipedia.org/wiki/Consonne_occlusive}} (\exEN{plosives\CW{https://en.wikipedia.org/wiki/Stop_consonant}})}

\newpage
\minitoc
\newpage

\section{Le \son~\phon{p} }\label{sec:p}

Ce \son a pour nom
technique\dyse{voiceless-bilabial-stop-p} :

\begin{itemize}
\item \exEN{Voiceless Bilabial Stop\CW{https://en.wikipedia.org/wiki/Voiceless_bilabial_stop}.}
\item \exFR{Consonne occlusive bilabiale sourde\CW{https://fr.wikipedia.org/wiki/Consonne_occlusive_bilabiale_sourde}.}
\end{itemize}

\begin{center}
  \begin{figure}[h]
    \centering
    \includegraphics[scale=.5]{../img/cpp/p-b-cpp}
    \caption{Consonnes p et b, source : \cpp}
    \label{fig:p-b}
  \end{figure}
\end{center}

\indicsound

\properukus{https://youtu.be/W3bI_PE_kNc}{https://youtu.be/V_n_rUKQSew}


\begin{enumerate}
\item \exEN{\href{http://www.wordreference.com/enfr/pause}{pause}} qui
  s'écrit phonétiquement
  \href{https://en.oxforddictionaries.com/definition/pause}{\phonm{pɔːz}}
  
  \begin{itemize}
  \item\exEN{\href{https://youtu.be/LNHBMFCzznE}{After} a brief \href{https://youtu.be/v\_UlZ0Y9Vho}{pause}, I continued.}
  \item\exFR{Après une courte pause, j'ai recommencé.}
  \end{itemize}

  \youglish{pause}
  
\item \exEN{\href{http://www.wordreference.com/enfr/pin}{pin}} qui s'écrit phonétiquement \href{https://en.oxforddictionaries.com/definition/pin}{\phonm{pɪn}}

  \begin{itemize}
  \item\exEN{She wore a diamond \href{https://youtu.be/DMoeYWQmRuQ}{pin} on her evening \href{https://youtu.be/qKNj5zG3NjA}{dress}.}
  \item\exFR{Elle portait une broche en diamants sur sa robe du soir.}
  \end{itemize}

  \youglish{pin}
  
\item \exEN{\href{http://www.wordreference.com/enfr/purpose}{purpose}} qui s'écrit phonétiquement \href{https://en.oxforddictionaries.com/definition/purpose}{\phonm{ˈpəːpəs}}

  \begin{itemize}
  \item\exEN{The \href{https://youtu.be/J8yhsbMULsQ}{purpose} of the game is to score \href{https://youtu.be/_vroLokIrCo}{points}.}
  \item\exFR{Le but du jeu consiste à marquer des points.}
  \end{itemize}

  \youglish{purpose}

\item \exEN{\href{http://www.wordreference.com/enfr/cap}{cap}} qui s'écrit
  phonétiquement
  \href{https://en.oxforddictionaries.com/definition/cap}{\phonm{kap}}.
  
  \begin{itemize}
  \item\exEN{The \href{https://youtu.be/ROXqddQYBH4}{baseball} player is wearing a \href{https://youtu.be/Dkzh8b5Mj3s}{cap} on his head.}
  \item\exFR{Le joueur de base-ball porte une casquette sur la tête.}
  \end{itemize}

  \youglish{cap}

\end{enumerate}

\newpage

\section{Le \son~\phon{b} }\label{sec:b}

Ce \son a pour nom
technique\dyse{voiced-bilabial-stop-b} :

\begin{itemize}
\item \exEN{Voiced Bilabial Stop\CW{https://en.wikipedia.org/wiki/Voiced_bilabial_stop}.}
\item \exFR{Consonne occlusive bilabiale voisée\CW{https://fr.wikipedia.org/wiki/Consonne_occlusive_bilabiale_vois\%C3\%A9e}.}
\end{itemize}

\begin{center}
  \begin{figure}[h]
    \centering
    \includegraphics[scale=.5]{../img/cpp/p-b-cpp}
    \caption{Consonnes p et b, source : \cpp}
    \label{fig:p-b}
  \end{figure}
\end{center}

\indicsound

\properukus{https://youtu.be/MPdBni6svgQ}{https://youtu.be/LbCOXRz7Uf8}


\begin{enumerate}
\item \exEN{\href{http://www.wordreference.com/enfr/bag}{bag}} qui s'écrit
  phonétiquement
  \href{https://en.oxforddictionaries.com/definition/bag}{\phonm{baɡ}}
  
  \begin{itemize}
  \item\exEN{I \href{https://youtu.be/TY4uxdAt4-M}{put} the fruit in a \href{https://youtu.be/DLQhGy7BIjQ}{bag}.}
  \item\exFR{J'ai mis les fruits dans un sac.}
  \end{itemize}

  \youglish{bag}
  
\item \exEN{\href{http://www.wordreference.com/enfr/bubble}{bubble}}
  qui s'écrit phonétiquement
  \href{https://en.oxforddictionaries.com/definition/bubble}{\phonm{ˈbʌb(ə)l}}
  
  \begin{itemize}
  \item\exEN{The \href{https://youtu.be/FE14vq6CDJk}{hot} soup was \href{https://youtu.be/h6OhzwkBAEc}{bubbling} in the saucepan.}
  \item\exFR{La soupe chaude bouillonnait dans la casserole.}
  \end{itemize}

  \youglish{bubble}
  
\item \exEN{\href{http://www.wordreference.com/enfr/build}{build}} qui
  s'écrit phonétiquement
  \href{https://en.oxforddictionaries.com/definition/build}{\phonm{bɪld}}
  
  \begin{itemize}
  \item\exEN{I \href{https://youtu.be/OVcbTvoh0L4}{bezeled} some wooden rods to \href{https://youtu.be/UC4eCvuxjIU}{build} a picture frame.}
  \item\exFR{J'ai taillé en biseau des baguettes de bois pour fabriquer un cadre.}
  \end{itemize}

  \youglish{build}
  
\item \exEN{\href{http://www.wordreference.com/enfr/robe}{robe}} qui s'écrit
  phonétiquement
  \href{https://en.oxforddictionaries.com/definition/robe}{\phonm{rəʊb}}
  
  \begin{itemize}
  \item\exEN{The \href{https://youtu.be/fFaWJLc18xU}{judge} entered the court room wearing her \href{https://youtu.be/eiObDiqVcPk}{robe}.}
  \item\exFR{La juge a fait \son entrée dans le tribunal portant sa robe.}
  \end{itemize}

  \youglish{robe}
  
\end{enumerate}

\newpage

\section{Le \son~\phon{t} }\label{sec:t}

  Ce \son a pour nom technique\dyse{voiceless-alveolar-stop-t} :% #1: lien
                                % vers le blog
  %
  \begin{itemize}%
  \item \exEN{Voicless Alveolar Stop\CW{https://en.wikipedia.org/wiki/Voiceless_dental_and_alveolar_stops}.}% #2: sound name, #3: wiki EN
  \item \exFR{Consonne alvéolaire sourde\CW{voiceless-alveolar-stop-t}.}% #4: nom du son, #5: wiki FR sinon blog voir
                         % package ifthen pour gérer ça
  \end{itemize}%

\begin{center}
  \begin{figure}[h]
    \centering
    \includegraphics[scale=.5]{../img/cpp/t-d-cpp}
    \caption{Consonnes t et d, source : \cpp}
    \label{fig:t-d}
  \end{figure}
\end{center}

  \indicsound%
  %
  \properukus{https://youtu.be/8gP9ygo2988}{https://youtu.be/mLlotV_0dRI}% #6: UK YT, #7: US YT


\begin{enumerate}
\item \exEN{\href{http://www.wordreference.com/enfr/time}{time}} qui s'écrit phonétiquement \href{https://en.oxforddictionaries.com/definition/time}{tʌɪm/}

  \begin{itemize}
  \item\exEN{It takes \href{https://youtu.be/5TbUxGZtwGI}{time} to get a good level in \href{https://youtu.be/H3r9bOkYW9s}{English}.}
  \item\exFR{Il faut du temps pour obtenir un bon niveau en anglais.}
  \end{itemize}

  \youglish{time}

\item \exEN{\href{http://www.wordreference.com/enfr/tow}{tow}} qui s'écrit
  phonétiquement
  \href{https://en.oxforddictionaries.com/definition/tow}{\phonm{təʊ}}
  
  \begin{itemize}
  \item\exEN{\href{https://youtu.be/h66Lvzlj1Uk}{Pleasure} craft are not permitted to \href{https://youtu.be/tGIx9uoJh9M}{tow} small personal
      boats or dinghies while transiting Canadian locks.}
  \item\exFR{Les embarcations de plaisance ne sont pas autorisées
      à remorquer de petits bateaux ou dériveurs personnels pendant
      le transit des écluses canadiennes.}
  \end{itemize}

  \youglish{tow}
    
\item \exEN{\href{http://www.wordreference.com/enfr/train}{train}} qui
  s'écrit phonétiquement
  \href{https://en.oxforddictionaries.com/definition/train}{\phonm{treɪn}}
  
  \begin{itemize}
  \item\exEN{I'm sorry, I \href{https://youtu.be/5EXC_rjs7tg}{missed} my \href{https://youtu.be/jgxKrH-O2Kk}{train} this morning.}
  \item\exFR{Je suis désolé, j'ai loupé mon train ce matin.}
  \end{itemize}

  \youglish{train}

\item \exEN{\href{http://www.wordreference.com/enfr/late}{late}} qui s'écrit
  phonétiquement
  \href{https://en.oxforddictionaries.com/definition/late}{\phonm{leɪt}}
  
  \begin{itemize}
  \item\exEN{It is very \href{https://youtu.be/YCGty3CBexc}{likely} that I will be \href{https://youtu.be/v\_HNcDj7-Kw}{late}.}
  \item\exFR{Il est très probable que j'arrive en retard.}
  \end{itemize}

  \youglish{late}
    
  \end{enumerate}

  \newpage
  
\section{Le \son~\phon{d}}\label{sec:d}

  Ce \son a pour nom technique\dyse{voiced-alveolar-stop-d} :% #1: lien
                                % vers le blog
  %
  \begin{itemize}%
  \item \exEN{Voiced Alveolar Stop\CW{https://en.wikipedia.org/wiki/Voiced_dental_and_alveolar_stops}.}% #2: sound name, #3: wiki EN
  \item \exFR{Consonne alvéolaire voisée\CW{voiced-alveolar-stop-d}.}% #4: nom du son, #5: wiki FR sinon blog voir
                         % package ifthen pour gérer ça
  \end{itemize}%

\begin{center}
  \begin{figure}[h]
    \centering
    \includegraphics[scale=.5]{../img/cpp/t-d-cpp}
    \caption{Consonnes t et d, source : \cpp}
    \label{fig:t-d}
  \end{figure}
\end{center}

  \indicsound%
  %
  \properukus{https://youtu.be/6R6hh1aKDiE}{https://youtu.be/N73xPe0x79g}% #6: UK YT, #7: US YT

\begin{enumerate}
\item \exEN{\href{http://www.wordreference.com/enfr/day}{day}} qui s'écrit
  phonétiquement
  \href{https://en.oxforddictionaries.com/definition/day}{\phonm{deɪ}}
  
  \begin{itemize}
  \item\exEN{One \href{https://youtu.be/sl9voSKJmEU}{day} you'll understand that \href{https://youtu.be/S2DnMGnAGNs}{practice makes perfect}.}
  \item\exFR{Un jour tu comprendras que la perfection n'est approchable que par la répétition.}
  \end{itemize}

  \youglish{day}
  
\item \exEN{\href{http://www.wordreference.com/enfr/door}{door}} qui s'écrit
  phonétiquement
  \href{https://en.oxforddictionaries.com/definition/door}{\phonm{dɔː}}
  
  \begin{itemize}
  \item\exEN{The \href{https://youtu.be/eDW\_yAwaHnc}{doors} are opened so you can \href{https://youtu.be/KDXOzr0GoA4}{come} in.}
  \item\exFR{Les portes sont ouvertes donc tu peux entrer.}
  \end{itemize}

  \youglish{door}
  
\item \exEN{\href{http://www.wordreference.com/enfr/down}{down}} qui s'écrit
  phonétiquement
  \href{https://en.oxforddictionaries.com/definition/down}{\phonm{daʊn}}
  
  \begin{itemize}
  \item\exEN{Following the \href{https://youtu.be/K--kIdOpbJM}{storm}, many trees are \href{https://youtu.be/pn4oaQNiNQc}{down}.}
  \item\exFR{Suite à la tempête, de nombreux arbres sont à terre.}
  \end{itemize}

  \youglish{down}
  
\item \exEN{\href{http://www.wordreference.com/enfr/drive}{drive}} qui s'écrit phonétiquement \href{https://en.oxforddictionaries.com/definition/drive}{\phonm{drʌɪv}}

  \begin{itemize}
  \item\exEN{The \href{https://youtu.be/mPBCO17bFms}{drive} to work is \href{https://youtu.be/vDjcWlCT8rg}{short}.}
  \item\exFR{Le trajet jusqu'au travail est court.}
  \end{itemize}

  \youglish{drive}

\end{enumerate}

\newpage

\section{Le \son~\phon{k}}\label{sec:k}

  Ce \son a pour nom technique\dyse{voiceless-velar-stop-k} :% #1: lien
                                % vers le blog
  %
  \begin{itemize}%
  \item \exEN{Voiceless Velar Stop\CW{https://en.wikipedia.org/wiki/Voiceless_velar_stop}.}% #2: sound name, #3: wiki EN
  \item \exFR{Consonne occlusive vélaire sourde\CW{https://fr.wikipedia.org/wiki/Consonne_occlusive_v\%C3\%A9laire_sourde}.}% #4: nom du son, #5: wiki FR sinon blog voir
                         % package ifthen pour gérer ça
  \end{itemize}

\begin{center}
  \begin{figure}[h]
    \centering
    \includegraphics[scale=.5]{../img/cpp/k-g-cpp}
    \caption{Consonnes k et g, source : \cpp}
    \label{fig:k-g}
  \end{figure}
\end{center}

  \indicsound%
  %
  \properukus{https://youtu.be/6YJdNE8na9M}{https://youtu.be/zxrveu6yu6E}% #6: UK YT, #7: US YT

\begin{enumerate}
\item \exEN{\href{http://www.wordreference.com/enfr/cash}{cash}} qui s'écrit
  phonétiquement
  \href{https://en.oxforddictionaries.com/definition/cash}{\phonm{kaʃ}}
  
  \begin{itemize}
  \item\exEN{The \href{https://youtu.be/VJ1OnnkwjrM}{supermarket} only accepts \href{https://youtu.be/ALGi0tcFCcw}{cash}.}
  \item\exFR{Le supermarché n'accepte que les espèces.}
  \end{itemize}

  \youglish{cash}
  
\item \exEN{\href{http://www.wordreference.com/enfr/cricket}{cricket}} qui
  s'écrit phonétiquement
  \href{https://en.oxforddictionaries.com/definition/cricket}{\phonm{ˈkrɪkɪt}}
  
  \begin{itemize}
  \item\exEN{In March, India's \href{https://youtu.be/c5oZPB-grGI}{cricket} team will be visiting
      Pakistan for the first time in a \href{https://youtu.be/tBkxOQodLnE}{decade}.}
  \item\exFR{Au mois de mars, l'équipe de cricket indienne se rendra au Pakistan pour la première fois depuis dix ans.}
  \end{itemize}

  \youglish{cricket}
  
\item \exEN{\href{http://www.wordreference.com/enfr/quick}{quick}} qui
  s'écrit phonétiquement
  \href{https://en.oxforddictionaries.com/definition/quick}{\phonm{kwɪk}}
  
  \begin{itemize}
  \item\exEN{We would \href{https://youtu.be/4bGTjagyJkQ}{appreciate} a \href{https://youtu.be/OB-YD47ddWI}{quick} reply.}
  \item\exFR{Nous apprécierions une réponse rapide.}
  \end{itemize}

  \youglish{quick}
  
\item \exEN{\href{http://www.wordreference.com/enfr/sock}{sock}} qui s'écrit
  phonétiquement
  \href{https://en.oxforddictionaries.com/definition/sock}{\phonm{sɒk}}
  
  \begin{itemize}
  \item\exEN{I put on \href{https://youtu.be/Eu1fW2BafnM}{socks} before putting on my \href{https://youtu.be/4srOE3pCCo8}{shoes}.}
  \item\exFR{J'ai enfilé des chaussettes avant de mettre mes chaussures.}
  \end{itemize}

  \youglish{sock}
  
\end{enumerate}

\newpage

\section{Le \son~\phon{g}}\label{sec:g}

  Ce \son a pour nom technique\dyse{voiced-velar-stop-g} :% #1: lien
                                % vers le blog
  %
  \begin{itemize}%
  \item \exEN{Voiced Velar Stop\CW{https://en.wikipedia.org/wiki/Voiced_velar_stop}.}% #2: sound name, #3: wiki EN
  \item \exFR{Consonne occlusive vélaire voisée\CW{https://fr.wikipedia.org/wiki/Consonne_occlusive_v\%C3\%A9laire_vois\%C3\%A9e}.}% #4: nom du son, #5: wiki FR sinon blog voir
                         % package ifthen pour gérer ça
  \end{itemize}

\begin{center}
  \begin{figure}[h]
    \centering
    \includegraphics[scale=.5]{../img/cpp/k-g-cpp}
    \caption{Consonnes k et g, source : \cpp}
    \label{fig:k-g}
  \end{figure}
\end{center}
  
  \indicsound
  %
  \properukus{https://youtu.be/8H_Xis2wigA}{https://youtu.be/vP5XKYvxe0Q}% #6: UK YT, #7: US YT
  
\begin{enumerate}
\item \exEN{\href{http://www.wordreference.com/enfr/girl}{girl}} qui s'écrit phonétiquement \href{https://en.oxforddictionaries.com/definition/girl}{\phonm{ɡəːl}}

  \begin{itemize}
  \item\exEN{\href{https://en.wikipedia.org/wiki/In\_the\_Pines}{My}
      \href{https://youtu.be/bpFuH8vcXbw}{girl},
      \href{https://genius.com/Nirvana-where-did-you-sleep-last-night-lyrics}{my}
      \href{https://youtu.be/PsfcUZBMSSg}{girl},
      \href{https://fr.wikipedia.org/wiki/Where\_Did\_You\_Sleep\_Last\_Night}{don't
        lie} to me.}
  \end{itemize}

  \youglish{girl}

\item \exEN{\href{http://www.wordreference.com/enfr/green}{green}} qui s'écrit phonétiquement \href{https://en.oxforddictionaries.com/definition/green}{ɡriːn/}. Exemple d'utilisation du mot :

  \begin{itemize}
  \item\exEN{The \href{https://youtu.be/s40b1e_cOeA}{mayor} launched a \href{https://youtu.be/a1BS7XnEZqc}{green} initiative to plant more
      trees.}
  \item\exFR{Le maire a lancé une initiative écologique pour planter davantage d'arbres.}
  \end{itemize}

  \youglish{green}

\item \exEN{\href{http://www.wordreference.com/enfr/grass}{grass}} qui s'écrit phonétiquement \href{https://en.oxforddictionaries.com/definition/grass}{\phonm{ɡrɑːs}}

  \begin{itemize}
  \item\exEN{\href{https://youtu.be/6NzSxKpyS2I}{Cows} feed on fresh \href{https://youtu.be/QsfJscoMx5M}{grass}.}
  \item\exFR{Les vaches se nourrissent d'herbe fraîche.}
  \end{itemize}

  \youglish{grass}
  
\item \exEN{\href{http://www.wordreference.com/enfr/flag}{flag}} qui s'écrit
  phonétiquement
  \href{https://en.oxforddictionaries.com/definition/flag}{\phonm{flaɡ}}
  
  \begin{itemize}
  \item\exEN{The \href{https://youtu.be/fiyYoe678yI}{vessel} flies the British \href{https://youtu.be/EBl2PVjVNqA}{flag}.}
  \item\exFR{Le navire bat pavillon britannique.}
  \end{itemize}

  \youglish{flag}
  
\end{enumerate}

\newpage

\section{Le \son~\phon{?}  \href{https://en.wikipedia.org/wiki/Glottal\_stop}{glottal}
  }\label{sec:glottal}

\begin{center}
  \begin{figure}[h]
    \centering
    \includegraphics[scale=.5]{../img/lodge/closed-glottis-lodge}
    \caption{\exEN{Closed Glottis}, source : \lodge}
    \label{fig:clos-glot}
  \end{figure}
\end{center}

  Ce \son a pour nom technique\dyse{glottal-stop} :% #1: lien
                                % vers le blog
  %
  \begin{itemize}%
  \item \exEN{Glottal Stop\CW{https://en.wikipedia.org/wiki/Glottal_stop}.}% #2: sound name, #3: wiki EN
  \item \exFR{Coup de glotte \CW{https://fr.wikipedia.org/wiki/Coup_de_glotte}.}% #4: nom du son, #5: wiki FR sinon blog voir
                         % package ifthen pour gérer ça
  \end{itemize}

\begin{center}
  \begin{figure}[h]
    \centering
    \includegraphics[scale=.5]{../img/lodge/open-glottis-lodge}
    \caption{\exEN{Open Glottis}, source : \lodge}
    \label{fig:open-glot}
  \end{figure}
\end{center}

  \indicsound
  %
  \properukus{https://youtu.be/A11_co6jJsI}{https://youtu.be/Vabg-EUHOQk}% #6: UK YT, #7: US YT
  
\begin{enumerate}
\item \exEN{\href{http://www.wordreference.com/enfr/football}{football}} qui
  s'écrit phonétiquement
  \href{https://www.phon.ucl.ac.uk/home/wells/phoneticsymbolsforenglish.htm}{\phonm{ˈfʊ?bɔːl}}
  
  \begin{itemize}
  \item\exEN{This summer the \href{https://youtu.be/6v5Ao0tYhBw}{football} \href{https://youtu.be/fTYgpGdvFa4}{world cup} will be in Russia
      and twenty four years ago it was in \href{https://youtu.be/mAYvjOzh1ag}{America}.}
  \item\exFR{Cet été la coupe du monde de football sera en Russie
      et il y a vingt-quatre ans c'était en Amérique.}
  \end{itemize}

  \youglish{football}
  
\item \href{http://www.wordreference.com/enfr/department}{department}
  qui s'écrit phonétiquement
  \href{https://en.oxforddictionaries.com/definition/department}{\phonm{dɪˈpɑː?m(ə)nt}}
  
  \begin{itemize}
  \item\exEN{\href{https://youtu.be/dDxOyf-LLhw}{Guadeloupe} is an overseas \href{https://youtu.be/0CUWPGLVRoU}{department} of France.}
  \item\exFR{La Guadeloupe est un département d'outre-mer de la France.}
  \end{itemize}

  \youglish{department}

\item \exEN{\href{http://www.wordreference.com/enfr/apartment}{apartment}}
  qui s'écrit phonétiquement
  \href{https://tophonetics.com/}{\phonm{əˈpɑː?mənt}}
  
  \begin{itemize}
  \item\exEN{My \href{https://youtu.be/H0HjU9956Z8}{apartment} is not in your \href{https://youtu.be/XQ4h1sDpKno}{department}.}
  \item\exFR{Mon appartement n'est pas dans votre département.}
  \end{itemize}

  \youglish{apartment}

\item \exEN{\href{http://www.wordreference.com/enfr/button}{button}} qui
  s'écrit phonétiquement
  \href{https://en.wikipedia.org/wiki/Glottal\_stop}{\phonm{ˈbɐʔn̩n}}
  
  \begin{itemize}
  \item\exEN{Click the \href{https://youtu.be/IJcwc5Gz8K0}{button} to subscribe.}
  \item\exFR{Cliquez sur le bouton pour vous abonner.}
  \end{itemize}

  \youglish{button}
  
\end{enumerate}

\begin{center}
  \begin{figure}[h]
    \centering
    \includegraphics[scale=.5]{../img/cpp/glot-stop-cpp}
    \caption{\exEN{Glottal Stop}, source : \cpp}
    \label{fig:f-v}
  \end{figure}
\end{center}


\newpage
\minitoc
\newpage

