\chapter{Quelques mots sur les annexes}\label{chap:mots-annex}

Vous trouverez en annexes différentes classifications des sons. En
effet, il existe de nombreuses façons de regrouper les sons et les
linguistes ne sont pas toujours d'accord entre eux pour savoir
laquelle serait <<~la meilleure~>>. Pour ma part, n'étant pas
linguiste professionnel et ayant une approche pragmatique je vous
recommanderais de ne pas accorder trop d'importance à ces
considérations. Comme je l'ai dit et répété ce qui compte c'est votre
capacité d'écoute et votre production sonore. C'est pour ça que le
document est particulièrement centré sur la connexion avec de
nombreuses sources variées\footnote{Cette richesse de documentation
  vidéo sera beaucoup moins abondante concernant les autres
  langues malheureusement.}.
Si vous souhaitez aller plus loin en anglais vous pouvez tout à fait
vous inscrire dans l'une de mes formations en passant par le
formulaire de contact sur mon blog
\url{http://doyouspeakenglish.fr/contact/}.

Vous trouverez également d'autres conseils et astuces pour améliorer
votre apprentissage et/ou pratique de l'anglais. J'ai aussi ajouté
quelques poèmes, dont un élaboré à des fins pédagogiques par le
British Council qui propose des applications de qualités, notamment
une sur la phonétique, et des
\href{https://learnenglish.britishcouncil.org/en/english-grammar}{outils
  en
  ligne}\footnote{\url{https://learnenglish.britishcouncil.org/en/english-grammar}}.

J'espère que ce livre vous aura aporté de la valeur et tous vos
commentaires seront les bienvenus que ça soit sur le blog en dessous
des articles ou sous les vidéos de la chaîne YouTube ou via le
formulaire de contact.

Merci pour votre attention et bon voyage au pays des langues.
