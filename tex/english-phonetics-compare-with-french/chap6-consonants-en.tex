\chapter{\exEN{Consonant sounds}}
\label{sec:org7af88d6}
Un son de consonne est fait en bloquant l'air quand il quitte la bouche. Nous utilisons la langue, les lèvres, les dents, la bouche et la gorge pour bloquer l'air. Certains sons utilisent la voix (exprimé), d'autres seulement l'air (sans voix).
\section{Plosive (un bloc d'air complet)}
\label{sec:orga048006}
Il y a 7 consonnes occlusives en anglais britannique.  
\subsection{Le son [p] comme dans les mots anglais}
\label{sec:orge071836}
\begin{enumerate}
\item \href{http://www.wordreference.com/enfr/pause}{pause} qui s'écrit phonétiquement \href{https://en.oxforddictionaries.com/definition/pause}{\emph{pɔːz}}. Exemple d'utilisation du mot :
\begin{description}
\item[{english}] \textenglish{After a brief \href{https://youtu.be/v\_UlZ0Y9Vho}{pause}, I continued.}
\item[{français}] Après une courte pause, j'ai recommencé.
\end{description}
\item \href{http://www.wordreference.com/enfr/pin}{pin} qui s'écrit phonétiquement \href{https://en.oxforddictionaries.com/definition/pin}{\emph{pɪn}}. Exemple d'utilisation du mot :
\begin{description}
\item[{english}] \textenglish{She wore a diamond \href{https://youtu.be/DMoeYWQmRuQ}{pin} on her evening dress.}
\item[{français}] Elle portait une broche en diamants sur sa robe du
soir.
\end{description}
\item \href{http://www.wordreference.com/enfr/purpose}{purpose} qui s'écrit phonétiquement \href{https://en.oxforddictionaries.com/definition/purpose}{\emph{ˈpəːpəs}}. Exemple d'utilisation du mot :
\begin{description}
\item[{english}] \textenglish{The \href{https://youtu.be/J8yhsbMULsQ}{purpose} of the game is to score points.}
\item[{français}] Le but du jeu consiste à marquer des points.
\end{description}
\item \href{http://www.wordreference.com/enfr/cap}{cap} qui s'écrit phonétiquement \href{https://en.oxforddictionaries.com/definition/cap}{\emph{kap}}. Exemple d'utilisation du mot :
\begin{description}
\item[{english}] \textenglish{The baseball player is wearing a \href{https://youtu.be/Dkzh8b5Mj3s}{cap} on his head.}
\item[{français}] Le joueur de base-ball porte une casquette sur la
tête.
\end{description}
\end{enumerate}
\subsection{Le son [b] comme dans les mots anglais}
\label{sec:orgab621dc}
\begin{enumerate}
\item \href{http://www.wordreference.com/enfr/bag}{bag} qui s'écrit phonétiquement \href{https://en.oxforddictionaries.com/definition/bag}{\emph{baɡ}}. Exemple d'utilisation du mot :
\begin{description}
\item[{english}] \textenglish{I put the fruit in a \href{https://youtu.be/DLQhGy7BIjQ}{bag}.}
\item[{français}] J'ai mis les fruits dans un sac.
\end{description}
\item \href{http://www.wordreference.com/enfr/bubble}{bubble} qui s'écrit phonétiquement \href{https://en.oxforddictionaries.com/definition/bubble}{\emph{ˈbʌb(ə)l}}. Exemple d'utilisation du mot :
\begin{description}
\item[{english}] \textenglish{The hot soup was \href{https://youtu.be/h6OhzwkBAEc}{bubbling} in the saucepan.}
\item[{française}] La soupe chaude bouillonnait dans la casserole.
\end{description}
\item \href{http://www.wordreference.com/enfr/build}{build} qui s'écrit phonétiquement \href{https://en.oxforddictionaries.com/definition/build}{\emph{bɪld}}. Exemple d'utilisation du mot :
\begin{description}
\item[{english}] \textenglish{I bezeled some wooden rods to \href{https://youtu.be/UC4eCvuxjIU}{build} a picture frame.}
\item[{français}] J'ai taillé en biseau des baguettes de bois pour fabriquer un cadre.
\end{description}
\item \href{http://www.wordreference.com/enfr/robe}{robe} qui s'écrit phonétiquement \href{https://en.oxforddictionaries.com/definition/robe}{\emph{rəʊb}}. Exemple d'utilisation du mot :
\begin{description}
\item[{english}] \textenglish{The judge entered the court room wearing her \href{https://youtu.be/eiObDiqVcPk}{robe}.}
\item[{français}] La juge a fait son entrée dans le tribunal portant sa
robe.
\end{description}
\end{enumerate}
\subsection{Le son [t] comme dans les mots anglais}
\label{sec:org461eecd}
\begin{enumerate}
\item \href{http://www.wordreference.com/enfr/time}{time} qui s'écrit phonétiquement \href{https://en.oxforddictionaries.com/definition/time}{tʌɪm/}. Exemple d'utilisation du mot :
\begin{description}
\item[{english}] \textenglish{It takes \href{https://youtu.be/A7pI96Osp9c}{time} to get a good level in English.}
\item[{français}] Il faut du temps pour obtenir un bon niveau en anglais.
\end{description}
\item \href{http://www.wordreference.com/enfr/tow}{tow} qui s'écrit phonétiquement \href{https://en.oxforddictionaries.com/definition/tow}{\emph{təʊ}}. Exemple d'utilisation du mot :
\begin{description}
\item[{english}] \textenglish{Pleasure craft are not permitted to \href{https://youtu.be/tGIx9uoJh9M}{tow} small personal
boats or dinghies while transiting Canadian locks.}
\item[{français}] Les embarcations de plaisance ne sont pas autorisées
à remorquer de petits bateaux ou dériveurs personnels pendant
le transit des écluses canadiennes.
\end{description}
\item \href{http://www.wordreference.com/enfr/train}{train} qui s'écrit phonétiquement \href{https://en.oxforddictionaries.com/definition/train}{\emph{treɪn}}. Exemple d'utilisation du mot :
\begin{description}
\item[{english}] \textenglish{I'm sorry, I missed my \href{https://youtu.be/jgxKrH-O2Kk}{train} this morning.}
\item[{français}] Je suis désolé, j'ai loupé mon train ce matin.
\end{description}
\item \href{http://www.wordreference.com/enfr/late}{late} qui s'écrit phonétiquement \href{https://en.oxforddictionaries.com/definition/late}{\emph{leɪt}}. Exemple d'utilisation du mot : 
\begin{description}
\item[{english}] \textenglish{It is very likely that I will be \href{https://youtu.be/v\_HNcDj7-Kw}{late}.}
\item[{français}] Il est très probable que j'arrive en retard.
\end{description}
\end{enumerate}
\subsection{Le son [d] comme dans les mots anglais}
\label{sec:org6c913ef}
\begin{enumerate}
\item \href{http://www.wordreference.com/enfr/day}{day} qui s'écrit phonétiquement \href{https://en.oxforddictionaries.com/definition/day}{\emph{deɪ}}. Exemple d'utilisation du mot :
\begin{description}
\item[{english}] \textenglish{One \href{https://youtu.be/sl9voSKJmEU}{day} you'll understand that practice makes perfect.}
\item[{français}] Un jour tu comprendras que la perfection n'est
approchable que par la répétition.
\end{description}
\item \href{http://www.wordreference.com/enfr/door}{door} qui s'écrit phonétiquement \href{https://en.oxforddictionaries.com/definition/door}{\emph{dɔː}}. Exemple d'utilisation du mot :
\begin{description}
\item[{english}] \textenglish{The \href{https://youtu.be/eDW\_yAwaHnc}{doors} are opened so you can come in.}
\item[{français}] Les portes sont ouvertes donc tu peux entrer.
\end{description}
\item \href{http://www.wordreference.com/enfr/down}{down} qui s'écrit phonétiquement \href{https://en.oxforddictionaries.com/definition/down}{\emph{daʊn}}. Exemple d'utilisation du mot :
\begin{description}
\item[{english}] \textenglish{Following the storm, many trees are \href{https://youtu.be/pn4oaQNiNQc}{down}.}
\item[{français}] Suite à la tempête, de nombreux arbres sont à terre.
\end{description}
\item \href{http://www.wordreference.com/enfr/drive}{drive} qui s'écrit phonétiquement \href{https://en.oxforddictionaries.com/definition/drive}{\emph{drʌɪv}}. Exemple d'utilisation du mot :
\begin{description}
\item[{english}] \textenglish{The \href{https://youtu.be/mPBCO17bFms}{drive} to work is short.}
\item[{français}] Le trajet jusqu'au travail est court.
\end{description}
\end{enumerate}
\subsection{Le son [k] comme dans les mots anglais}
\label{sec:org5ed979e}
\begin{enumerate}
\item \href{http://www.wordreference.com/enfr/cash}{cash} qui s'écrit phonétiquement \href{https://en.oxforddictionaries.com/definition/cash}{\emph{kaʃ}}. Exemple d'utilisation du mot :
\begin{description}
\item[{english}] \textenglish{The supermarket only accepts \href{https://youtu.be/ALGi0tcFCcw}{cash}.}
\item[{français}] Le supermarché n'accepte que les espèces.
\end{description}
\item \href{http://www.wordreference.com/enfr/cricket}{cricket} qui s'écrit phonétiquement \href{https://en.oxforddictionaries.com/definition/cricket}{\emph{ˈkrɪkɪt}}. Exemple d'utilisation du mot :
\begin{description}
\item[{english}] \textenglish{In March, India's \href{https://youtu.be/c5oZPB-grGI}{cricket} team will be visiting
Pakistan for the first time in a decade.}
\item[{français}] Au mois de mars, l'équipe de cricket indienne se rendra au Pakistan pour la première fois depuis dix ans.
\end{description}
\item \href{http://www.wordreference.com/enfr/quick}{quick} qui s'écrit phonétiquement \href{https://en.oxforddictionaries.com/definition/quick}{\emph{kwɪk}}. Exemple d'utilisation du mot :
\begin{description}
\item[{english}] \textenglish{We would appreciate a \href{https://youtu.be/OB-YD47ddWI}{quick} reply.}
\item[{française}] Nous apprécierions une réponse rapide.
\end{description}
\item \href{http://www.wordreference.com/enfr/sock}{sock} qui s'écrit phonétiquement \href{https://en.oxforddictionaries.com/definition/sock}{\emph{sɒk}}. Exemple d'utilisation du mot :
\begin{description}
\item[{english}] \textenglish{I put on \href{https://youtu.be/Eu1fW2BafnM}{socks} before putting on my shoes.}
\item[{français}] J'ai enfilé des chaussettes avant de mettre mes
chaussures.
\end{description}
\end{enumerate}
\subsection{Le son [g] comme dans les mots anglais}
\label{sec:orgf88a6b3}
\begin{enumerate}
\item \href{http://www.wordreference.com/enfr/girl}{girl} qui s'écrit phonétiquement \href{https://en.oxforddictionaries.com/definition/girl}{\emph{ɡəːl}}. Exemple d'utilisation du mot :
\begin{description}
\item[{english}] \textenglish{\href{https://en.wikipedia.org/wiki/In\_the\_Pines}{My}
\href{https://youtu.be/bpFuH8vcXbw}{girl},
\href{https://genius.com/Nirvana-where-did-you-sleep-last-night-lyrics}{my}
\href{https://youtu.be/PsfcUZBMSSg}{girl},
\href{https://fr.wikipedia.org/wiki/Where\_Did\_You\_Sleep\_Last\_Night}{don't
lie} to me.}
\end{description}
\item \href{http://www.wordreference.com/enfr/green}{green} qui s'écrit phonétiquement \href{https://en.oxforddictionaries.com/definition/green}{ɡriːn/}. Exemple d'utilisation du mot :
\begin{description}
\item[{english}] \textenglish{The mayor launched a \href{https://youtu.be/a1BS7XnEZqc}{green} initiative to plant more
trees.}
\item[{français}] Le maire a lancé une initiative écologique pour
planter davantage d'arbres.
\end{description}
\item \href{http://www.wordreference.com/enfr/grass}{grass} qui s'écrit phonétiquement \href{https://en.oxforddictionaries.com/definition/grass}{\emph{ɡrɑːs}}. Exemple d'utilisation du mot :
\begin{description}
\item[{english}] \textenglish{Cows feed on fresh \href{https://youtu.be/QsfJscoMx5M}{grass}.}
\item[{français}] Les vaches se nourrissent d'herbe fraîche.
\end{description}
\item \href{http://www.wordreference.com/enfr/flag}{flag} qui s'écrit phonétiquement \href{https://en.oxforddictionaries.com/definition/flag}{\emph{flaɡ}}. Exemple d'utilisation du mot :
\begin{description}
\item[{english}] \textenglish{The vessel flies the British \href{https://youtu.be/EBl2PVjVNqA}{flag}.}
\item[{français}] Le navire bat pavillon britannique.
\end{description}
\end{enumerate}
\subsection{Le son \href{https://en.wikipedia.org/wiki/Glottal\_stop}{glottal} [?] comme dans les mots anglais}
\label{sec:org630b613}
\begin{enumerate}
\item \href{http://www.wordreference.com/enfr/football}{football} qui s'écrit phonétiquement \href{https://www.phon.ucl.ac.uk/home/wells/phoneticsymbolsforenglish.htm}{\emph{ˈfʊ?bɔːl}}. Exemple d'utilisation du mot :
\begin{description}
\item[{english}] \textenglish{This summer the \href{https://youtu.be/6v5Ao0tYhBw}{football} \href{https://youtu.be/zVr3dTMY9Ag}{world cup} will be in Russia
and twenty four years ago it was in \href{https://youtu.be/mAYvjOzh1ag}{America}.}
\item[{français}] Cet été la coupe du monde de football sera en Russie
et il y a vingt-quatre ans c'était en Amérique.
\end{description}
\item \href{http://www.wordreference.com/enfr/department}{department} qui s'écrit phonétiquement \href{https://en.oxforddictionaries.com/definition/department}{\emph{dɪˈpɑː?m(ə)nt}}. Exemple d'utilisation du mot :
\begin{description}
\item[{english}] \textenglish{Guadeloupe is an overseas \href{https://youtu.be/0CUWPGLVRoU}{department} of France.}
\item[{français}] La Guadeloupe est un département d'outre-mer de la
France.
\end{description}
\item \href{http://www.wordreference.com/enfr/apartment}{apartment} qui s'écrit phonétiquement \href{https://tophonetics.com/}{\emph{əˈpɑː?mənt}}. Exemple d'utilisation du mot :
\begin{description}
\item[{english}] \textenglish{My \href{https://youtu.be/H0HjU9956Z8}{apartment} is not in your department.}
\item[{français}] Mon appartement n'est pas dans votre département.
\end{description}
\item \href{http://www.wordreference.com/enfr/button}{button} qui s'écrit phonétiquement \href{https://en.wikipedia.org/wiki/Glottal\_stop}{\emph{ˈbɐʔn̩n}}. Exemple d'utilisation du mot :
\begin{description}
\item[{english}] \textenglish{Click the \href{https://youtu.be/IJcwc5Gz8K0}{button} to subscribe.}
\item[{français}] Cliquez sur le bouton pour vous abonner.
\end{description}
\end{enumerate}
\section{Fricative (une compression constante de l'air)}
\label{sec:org3b742bc}
\subsection{Le son [f] comme dans les mots anglais}
\label{sec:org2a4125a}
\begin{enumerate}
\item \href{http://www.wordreference.com/enfr/fish}{fish} qui s'écrit phonétiquement \href{https://en.oxforddictionaries.com/definition/fish}{\emph{fɪʃ}}. Exemple d'utilisation du mot : 
\begin{description}
\item[{english}] \textenglish{He prefers \href{https://youtu.be/rEm4ynLtGx4}{fish} to meat.}
\item[{français}] Il préfère le poisson à la viande.
\end{description}
\item \href{http://www.wordreference.com/enfr/friday}{friday} qui s'écrit phonétiquement \href{https://en.oxforddictionaries.com/definition/friday}{\emph{ˈfrʌɪdi}}. Exemple d'utilisation du mot :
\begin{description}
\item[{english}] \textenglish{The ship sailed from the port on \href{https://youtu.be/lQ\_pgrjjHLo}{Friday}.}
\item[{français}] Le bateau a quitté le port vendredi.
\end{description}
\item \href{http://www.wordreference.com/enfr/full}{full} qui s'écrit phonétiquement \href{https://en.oxforddictionaries.com/definition/full}{\emph{fʊl}}. Exemple d'utilisation du mot : 
\begin{description}
\item[{english}] \textenglish{The \href{https://youtu.be/LR73DrKX\_bs}{full} report is hundreds of pages long.}
\item[{français}] Le rapport complet fait des centaines de pages.
\end{description}
\item \href{http://www.wordreference.com/enfr/knife}{knife} qui s'écrit phonétiquement \href{https://en.oxforddictionaries.com/definition/knife}{\emph{nʌɪf}}. Exemple d'utilisation du mot : 
\begin{description}
\item[{english}] \textenglish{The blunt \href{https://youtu.be/JUyzH9HpkqE}{knife} could not cut the rope.}
\item[{français}] Le couteau émoussé ne pouvait pas couper la corde.
\end{description}
\end{enumerate}
\subsection{Le son [v] comme dans les mots anglais}
\label{sec:org1c51acc}
\begin{enumerate}
\item \href{http://www.wordreference.com/enfr/cave}{cave} qui s'écrit phonétiquement \href{https://en.oxforddictionaries.com/definition/cave}{\emph{keɪv}}. Exemple d'utilisation du mot :
\begin{description}
\item[{english}] \textenglish{Plato is famous for his myth of the \href{https://youtu.be/kZQbkzwwinI}{cave}.}
\item[{français}] Platon est célèbre pour son mythe de la caverne.
\end{description}
\item \href{http://www.wordreference.com/enfr/vest}{vest} qui s'écrit phonétiquement \href{https://en.oxforddictionaries.com/definition/vest}{\emph{vɛst}}. Exemple d'utilisation du mot :
\begin{description}
\item[{english}] \textenglish{The committee was \href{https://youtu.be/E4cjvxydHuU}{vested} with the government's full
authority.}
\item[{français}] Le comité était investi de toute l'autorité du
gouvernement.
\end{description}
\item \href{http://www.wordreference.com/enfr/view}{view} qui s'écrit phonétiquement \href{https://en.oxforddictionaries.com/definition/view}{\emph{vjuː}}. Exemple d'utilisation du mot :
\begin{description}
\item[{english}] \textenglish{There is a splendid \href{https://youtu.be/gODvA\_SdXCY}{view} from the balcony.}
\item[{français}] Il y a une vue splendide depuis le balcon.
\end{description}
\item \href{http://www.wordreference.com/enfr/village}{village} qui s'écrit phonétiquement \href{https://en.oxforddictionaries.com/definition/village}{\emph{ˈvɪlɪdʒ}}. Exemple d'utilisation du mot :
\begin{description}
\item[{english}] \textenglish{The \href{https://youtu.be/Xq8mt6WuD-E}{village} is peaceful at night.}
\item[{français}] Le village est tranquille la nuit.
\end{description}
\end{enumerate}
\subsection{Le son [θ] comme dans les mots anglais}
\label{sec:org4f7f160}
\begin{enumerate}
\item \href{http://www.wordreference.com/enfr/author}{author} qui s'écrit phonétiquement \href{https://en.oxforddictionaries.com/definition/author}{\emph{ˈɔːθə}}. Exemple d'utilisation du mot :
\begin{description}
\item[{english}] \textenglish{I am the \href{https://youtu.be/lyGivD8aJi4}{author} of this document.}
\item[{français}] Je suis l'auteur de ce document.
\end{description}
\item \href{http://www.wordreference.com/enfr/path}{path} qui s'écrit phonétiquement \href{https://en.oxforddictionaries.com/definition/path}{\emph{pɑːθ}}. Exemple d'utilisation du mot :
\begin{description}
\item[{english}] \textenglish{A fork in the road splits it into two \href{https://youtu.be/EZdFE-nnyyQ}{paths}.}
\item[{français}] Un embranchement sur la route la divise en deux
sentiers.
\end{description}
\item \href{http://www.wordreference.com/enfr/thing}{thing} qui s'écrit phonétiquement \href{https://en.oxforddictionaries.com/definition/thing}{\emph{θɪŋ}}. Exemple d'utilisation du mot :
\begin{description}
\item[{english}] \textenglish{Windsurfing is not really my \href{https://youtu.be/h-pmsrw8XNE}{thing}; I prefer surfing.}
\item[{français}] La planche à voile n'est pas vraiment mon truc ; je
préfère surfer.
\end{description}
\item \href{http://www.wordreference.com/enfr/think}{think} qui s'écrit phonétiquement \href{https://en.oxforddictionaries.com/definition/think}{\emph{θɪŋk}}. Exemple d'utilisation du mot :
\begin{description}
\item[{english}] \textenglish{I \href{https://youtu.be/PpD8OvMTRiE}{think} my solution is the best.}
\item[{français}] Je considère que ma solution est la meilleure.
\end{description}
\end{enumerate}
\subsection{Le son [ð] comme dans les mots anglais}
\label{sec:orge02dbdd}
\begin{enumerate}
\item \href{http://www.wordreference.com/enfr/this}{this} qui s'écrit phonétiquement \href{https://en.oxforddictionaries.com/definition/this}{\emph{ðɪs}}. Exemple d'utilisation du mot :
\begin{description}
\item[{english}] \textenglish{The implementation of \href{https://youtu.be/KqzlYTmFBGY}{this} principle will, as a
consequence, generate more data than currently available.}
\item[{français}] L'application de ce principe va donc générer plus de
données que ce qui est actuellement disponible.
\end{description}
\item \href{http://www.wordreference.com/enfr/other}{other} qui s'écrit phonétiquement \href{https://en.oxforddictionaries.com/definition/other}{\emph{ˈʌðə}}. Exemple d'utilisation du mot :
\begin{description}
\item[{english}] \textenglish{The woman was selling apples and \href{https://youtu.be/9gXP8wcICqQ}{other} fruits.}
\item[{français}] La femme vendait des pommes et d'autres fruits.
\end{description}
\item \href{http://www.wordreference.com/enfr/breathe}{breathe} qui s'écrit phonétiquement \href{https://en.oxforddictionaries.com/definition/breathe}{\emph{briːð}}. Exemple d'utilisation du mot :
\begin{description}
\item[{english}] \textenglish{The air we \href{https://youtu.be/V8rtJRlLdI8}{breathe} is invisible.}
\item[{français}] L'air que nous respirons est invisible.
\end{description}
\item \href{http://www.wordreference.com/enfr/bathe}{bathe} qui s'écrit phonétiquement \href{https://dictionary.cambridge.org/dictionary/english/bathe}{\emph{beɪð}}. Exemple d'utilisation du mot :
\begin{description}
\item[{english}] \textenglish{Can I \href{https://youtu.be/U9V8cx2buG0}{bathe} my baby from the first hours of their
life?}
\item[{français}] Puis-je baigner mon bébé dès ses premières heures de
vie ?
\end{description}
\end{enumerate}
\subsection{Le son [s] comme dans les mots anglais}
\label{sec:org08b2b88}
\begin{enumerate}
\item \href{http://www.wordreference.com/enfr/kiss}{kiss} qui s'écrit phonétiquement \href{https://en.oxforddictionaries.com/definition/kiss}{\emph{kɪs}}. Exemple d'utilisation du mot :
\begin{description}
\item[{english}] \textenglish{The princess \href{https://youtu.be/vMbVzr7WqIo}{kissed} the frog.}
\item[{français}] La princesse a embrassé la grenouille.
\end{description}
\item \href{http://www.wordreference.com/enfr/cease}{cease} qui s'écrit phonétiquement \href{https://en.oxforddictionaries.com/definition/cease}{\emph{siːs}}. Exemple d'utilisation du mot :
\begin{description}
\item[{english}] \textenglish{My wife never \href{https://youtu.be/6m9bEMejTKI}{ceases} to amaze me.}
\item[{français}] Ma femme ne cesse de m'étonner.
\end{description}
\item \href{http://www.wordreference.com/enfr/sister}{sister} qui s'écrit phonétiquement \href{https://en.oxforddictionaries.com/definition/sister}{\emph{ˈsɪstə}}. Exemple d'utilisation du mot :
\begin{description}
\item[{english}] \textenglish{Many \href{https://youtu.be/SBNB13EeRx4}{sisters} live in the convent.}
\item[{français}] De nombreuses religieuses vivent dans le couvent.
\end{description}
\item \href{http://www.wordreference.com/enfr/sight}{sight} qui s'écrit phonétiquement \href{https://en.oxforddictionaries.com/definition/sight}{\emph{sʌɪt}}. Exemple d'utilisation du mot : 
\begin{description}
\item[{english}] \textenglish{I witnessed a strange \href{https://youtu.be/JeiVf30VDDU}{sight} in the street.}
\item[{français}] J'ai été témoin d'une scène étrange dans la rue.
\end{description}
\end{enumerate}
\subsection{Le son [z] comme dans les mots anglais}
\label{sec:orgeeffa49}
\begin{enumerate}
\item \href{http://www.wordreference.com/enfr/buzz}{buzz} qui s'écrit phonétiquement \href{https://en.oxforddictionaries.com/definition/buzz}{\emph{bʌz}}. Exemple d'utilisation du mot :
\begin{description}
\item[{english}] \textenglish{The news caused a \href{https://youtu.be/OoQJUNv-Jlg}{buzz} in the audience.}
\item[{français}] La nouvelle a provoqué l'effervescence du public.
\end{description}
\item \href{http://www.wordreference.com/enfr/crazy}{crazy} qui s'écrit phonétiquement \href{https://en.oxforddictionaries.com/definition/crazy}{\emph{ˈkreɪzi}}. Exemple d'utilisation du mot :
\begin{description}
\item[{english}] \textenglish{My aunt is \href{https://youtu.be/U0EW0s1fN-8}{crazy} about her cats.}
\item[{français}] Ma tante est dingue de ses chats.
\end{description}
\item \href{http://www.wordreference.com/enfr/lazy}{lazy} qui s'écrit phonétiquement \href{https://en.oxforddictionaries.com/definition/lazy}{\emph{ˈleɪzi}}. Exemple d'utilisation du mot :
\begin{description}
\item[{english}] \textenglish{My son is smart but incredibly \href{https://youtu.be/3ev7GXzFTPg}{lazy}.}
\item[{français}] Mon fils est intelligent mais extrêmement paresseux.
\end{description}
\item \href{http://www.wordreference.com/enfr/nose}{nose} qui s'écrit phonétiquement \href{https://en.oxforddictionaries.com/definition/nose}{\emph{nəʊz}}. Exemple d'utilisation du mot :
\begin{description}
\item[{english}] \textenglish{The tip of my \href{https://youtu.be/1G-nn-b4TJA}{nose} is cold.}
\item[{français}] Le bout de mon nez est froid.
\end{description}
\end{enumerate}
\subsection{Le son [ʃ] comme dans les mots anglais}
\label{sec:org8d21030}
\begin{enumerate}
\item \href{http://www.wordreference.com/enfr/cash}{cash} qui s'écrit phonétiquement \href{https://en.oxforddictionaries.com/definition/cash}{\emph{kaʃ}}. Exemple d'utilisation du mot :
\begin{description}
\item[{english}] \textenglish{The \href{https://youtu.be/4ahHWROn8M0}{cash} he received for his invention is a windfall.}
\item[{français}] L'argent qu'il a reçu pour son invention est une aubaine.
\end{description}
\item \href{http://www.wordreference.com/enfr/national}{national} qui s'écrit phonétiquement \href{https://en.oxforddictionaries.com/definition/national}{\emph{ˈnaʃ(ə)n(ə)l}}. Exemple d'utilisation du mot : 
\begin{description}
\item[{english}] \textenglish{The country's beautiful landscapes are a subject of
\href{https://youtu.be/xZvzCOQ-TPA}{national} pride.}
\item[{français}] Les beaux paysages du pays sont un objet de fierté
nationale.
\end{description}
\item \href{http://www.wordreference.com/enfr/crash}{crash} qui s'écrit phonétiquement \href{https://en.oxforddictionaries.com/definition/crash}{\emph{kraʃ}}. Exemple d'utilisation du mot :
\begin{description}
\item[{english}] \textenglish{All passengers on the plane survived the \href{https://youtu.be/Jw81bRYUzVM}{crash}.}
\item[{français}] Tous les passagers de l'avion ont survécu à
l'accident.
\end{description}
\item \href{http://www.wordreference.com/enfr/ship}{ship} qui s'écrit phonétiquement \href{https://en.oxforddictionaries.com/definition/ship}{\emph{ʃɪp}}. Exemple d'utilisation du mot :
\begin{description}
\item[{english}] \textenglish{The company mainly \href{https://youtu.be/LLkGsfOfgUw}{ships} parcels to Europe.}
\item[{français}] L'entreprise expédie principalement des colis vers
l'Europe.
\end{description}
\end{enumerate}
\subsection{Le son [ʒ] comme dans les mots anglais}
\label{sec:orga03cd2d}
\begin{enumerate}
\item \href{http://www.wordreference.com/enfr/leisure}{leisure} qui s'écrit phonétiquement \href{https://en.oxforddictionaries.com/definition/leisure}{\emph{ˈlɛʒə}}. Exemple d'utilisation du mot :
\begin{description}
\item[{english}] \textenglish{Everyone needs moments of \href{https://youtu.be/VSRFE7E4qWI}{leisure} to relax.}
\item[{français}] Tout le monde a besoin de moments de loisir pour se
détendre.
\end{description}
\item \href{http://www.wordreference.com/enfr/measure}{measure} qui s'écrit phonétiquement \href{https://en.oxforddictionaries.com/definition/measure}{\emph{ˈmɛʒə}}. Exemple d'utilisation du mot :
\begin{description}
\item[{english}] \textenglish{This application \href{https://youtu.be/bN60fb9fzKg}{measures} the speed of the Internet
connection.}
\item[{français}] Cette application calcule la vitesse de la connexion
Internet.
\end{description}
\item \href{http://www.wordreference.com/enfr/pleasure}{pleasure} qui s'écrit phonétiquement \href{https://en.oxforddictionaries.com/definition/pleasure}{\emph{ˈplɛʒə}}. Exemple d'utilisation du mot :
\begin{description}
\item[{english}] \textenglish{I read your book with great \href{https://youtu.be/Q4-VK5uqY34}{pleasure}.}
\item[{français}] J'ai lu votre livre avec grand plaisir.
\end{description}
\item \href{http://www.wordreference.com/enfr/vision}{vision} qui s'écrit phonétiquement \href{https://en.oxforddictionaries.com/definition/vision}{\emph{ˈvɪʒ(ə)n}}. Exemple d'utilisation du mot :
\begin{description}
\item[{english}] \textenglish{The teacher's \href{https://youtu.be/lk7lIhAmwHI}{vision} was getting fuzzy so he put his
glasses on.}
\item[{français}] Comme sa vision devenait floue, le professeur a mis
ses lunettes
\end{description}
\end{enumerate}
\subsection{Le son [h] comme dans les mots anglais}
\label{sec:orga471592}
\begin{enumerate}
\item \href{http://www.wordreference.com/enfr/ahead}{ahead} qui s'écrit phonétiquement \href{https://en.oxforddictionaries.com/definition/ahead}{\emph{əˈhɛd}}. Exemple d'utilisation du mot : 
\begin{description}
\item[{english}] \textenglish{It has been major, important and time-consuming work,
because we in actual fact have demanding and important tasks
\href{https://youtu.be/1rLpIOzKaBA}{ahead} of us.}
\item[{français}] C'est un travail énorme, important et très long, dans
la mesure où les missions qui nous attendent sont importantes
et exigeantes.
\end{description}
\item \href{http://www.wordreference.com/enfr/hello}{hello} qui s'écrit phonétiquement \href{https://en.oxforddictionaries.com/definition/hello}{\emph{hɛˈləʊ}}. Exemple d'utilisation du mot :
\begin{description}
\item[{english}] \textenglish{\href{https://youtu.be/62XB9IbMnxQ}{Hello} \href{https://en.wikipedia.org/wiki/\%2522Hello,\_World!\%2522\_program}{world}!}
\item[{français}] Bonjour le monde !
\end{description}
\item \href{http://www.wordreference.com/enfr/high}{high} qui s'écrit phonétiquement \href{https://en.oxforddictionaries.com/definition/high}{\emph{hʌɪ}}. Exemple d'utilisation du mot :
\begin{description}
\item[{english}] \textenglish{\href{https://youtu.be/F7lj4LknWO8}{High} walls surrounded the \href{https://youtu.be/hclQLklBHNs}{castle}.}
\item[{français}] De hauts murs entouraient le château.
\end{description}
\item \href{http://www.wordreference.com/enfr/whole}{whole} qui s'écrit phonétiquement \href{https://en.oxforddictionaries.com/definition/whole}{\emph{həʊl}}. Exemple d'utilisation du mot :
\begin{description}
\item[{english}] \textenglish{The environment concerns society as a \href{https://youtu.be/bJnw1ma6Xks}{whole}.}
\item[{français}] L'environnement concerne l'ensemble de la société.
\end{description}
\end{enumerate}
\section{Africative (plosive + fricative)}
\label{sec:org54afee9}
\subsection{Le son [tʃ] comme dans les mots anglais}
\label{sec:orgd294ba3}
\begin{enumerate}
\item \href{http://www.wordreference.com/enfr/cheese}{cheese} qui s'écrit phonétiquement \href{https://en.oxforddictionaries.com/definition/cheese}{\emph{tʃiːz}}. Exemple d'utilisation de
ce mot :
\begin{description}
\item[{english}] \textenglish{The \href{https://youtu.be/xYyP9o8wXtc}{cheese} had an awful smell.}
\item[{français}] Le fromage dégageait une odeur horrible.
\end{description}
\item \href{http://www.wordreference.com/enfr/match}{match} qui s'écrit phonétiquement \href{https://en.oxforddictionaries.com/definition/match}{\emph{matʃ}}. Exemple d'utilisation de
ce mot :
\begin{description}
\item[{english}] \textenglish{The password \href{https://youtu.be/-o\_IoZdtbWs}{matches} the one in the database.}
\item[{français}] Le mot de passe correspond à celui de la base de
données.
\end{description}
\item \href{http://www.wordreference.com/enfr/nature}{nature} qui s'écrit phonétiquement \href{https://en.oxforddictionaries.com/definition/nature}{\emph{ˈneɪtʃə}}. Exemple d'utilisation
de ce mot :
\begin{description}
\item[{english}] \textenglish{Preserving \href{https://youtu.be/K\_jwPJM0QSc}{nature} is a matter of public concern.}
\item[{français}] Préserver la nature est une question de
responsabilité publique.
\end{description}
\item \href{http://www.wordreference.com/enfr/watch}{watch} qui s'écrit phonétiquement \href{https://en.oxforddictionaries.com/definition/watch}{\emph{wɒtʃ}}. Exemple d'utilisation de
ce mot :
\begin{description}
\item[{english}] \textenglish{When my parents go out I have to \href{https://youtu.be/Eya0daHX-Fw}{watch} my little
sister.}
\item[{français}] Quand mes parents sortent je dois surveiller ma
petite soeur.
\end{description}
\end{enumerate}
\subsection{Le son [dʒ] comme dans les mots anglais}
\label{sec:org83b259c}
\begin{enumerate}
\item \href{http://www.wordreference.com/enfr/age}{age} qui s'écrit phonétiquement \href{https://en.oxforddictionaries.com/definition/age}{\emph{eɪdʒ}}. Exemple d'utilisation de ce
mot :
\begin{description}
\item[{english}] \textenglish{\href{https://youtu.be/wKU5khnuY\_Y}{Age} and inactivity reduce joint mobility.}
\item[{français}] L'âge et l'inactivité réduisent la mobilité
articulaire.
\end{description}
\item \href{http://www.wordreference.com/enfr/joy}{joy} qui s'écrit phonétiquement \href{https://en.oxforddictionaries.com/definition/joy}{\emph{dʒɔɪ}}. Exemple d'utilisation de ce
mot :
\begin{description}
\item[{english}] \textenglish{The \href{https://youtu.be/TyYIxGL2p6c}{joy} of \href{https://www.amazon.fr/gp/product/B013RQ72R2/ref=as\_li\_tl?ie=UTF8\&camp=1642\&creative=6746\&creativeASIN=B013RQ72R2\&linkCode=as2\&tag=wwwbecomefree-21\&linkId=e8ebecacb076d66dd3e5a435789050d5}{phonetics}.}
\item[{français}] La joie de la phonétique.
\end{description}
\item \href{http://www.wordreference.com/enfr/juggle}{juggle} qui s'écrit phonétiquement \href{https://en.oxforddictionaries.com/definition/juggle}{\emph{ˈdʒʌɡ(ə)l}}. Exemple
d'utilisation de ce mot :
\begin{description}
\item[{english}] \textenglish{He can \href{https://youtu.be/kCt1bmSASCI}{juggle} with five balls.}
\item[{français}] Il peut jongler avec cinq balles.
\end{description}
\item \href{http://www.wordreference.com/enfr/soldier}{soldier} qui s'écrit phonétiquement \href{https://en.oxforddictionaries.com/definition/soldier}{\emph{ˈsəʊldʒə}}. Exemple
d'utilisation de ce mot : 
\begin{description}
\item[{english}] \textenglish{The \href{https://youtu.be/ucoSdNM2Atw}{soldier} defused the bomb.}
\item[{français}] Le soldat a désamorcé la bombe.
\end{description}
\end{enumerate}
\section{Nasal (air libéré par le nez)}
\label{sec:org63d4d3c}
\subsection{Le son [m] comme dans les mots anglais}
\label{sec:org6dbd0b1}
\begin{enumerate}
\item \href{http://www.wordreference.com/enfr/calm}{calm} qui s'écrit phonétiquement \href{https://en.oxforddictionaries.com/definition/calm}{\emph{kɑːm}}. Exemple d'utilisation de ce
mot :
\begin{description}
\item[{english}] \textenglish{He kept \href{https://youtu.be/1tXBl3Q5Ibc}{calm} in order not to start a scrap.}
\item[{français}] Il est resté calme afin de ne pas déclencher une
bagarre.
\end{description}
\item \href{http://www.wordreference.com/enfr/hammer}{hammer} qui s'écrit phonétiquement \href{https://en.oxforddictionaries.com/definition/hammer}{\emph{ˈhamə}}. Exemple d'utilisation de
ce mot :
\begin{description}
\item[{english}] \textenglish{I need a screwdriver and a \href{https://youtu.be/t5l2AUlD8Sk}{hammer} to fix the shelf.}
\item[{français}] J'ai besoin d'un tournevis et d'un marteau pour
réparer l'étagère.
\end{description}
\item \href{http://www.wordreference.com/enfr/mad}{mad} qui s'écrit phonétiquement \href{https://en.oxforddictionaries.com/definition/mad}{\emph{mad}}. Exemple d'utilisation de ce
mot :
\begin{description}
\item[{english}] \textenglish{The scientist must be \href{https://youtu.be/Oa-ae6\_okmg}{mad} to try such experiments.}
\item[{français}] Le scientifique doit être dingue pour tenter de telles expériences.
\end{description}
\item \href{http://www.wordreference.com/enfr/sum}{sum} qui s'écrit phonétiquement \href{https://en.oxforddictionaries.com/definition/sum}{\emph{sʌm}}. Exemple d'utilisation de ce
mot : 
\begin{description}
\item[{english}] \textenglish{The \href{https://youtu.be/ymUTWzsoiIg}{sum} is indicated on the invoice.}
\item[{français}] Le total est indiqué sur la facture.
\end{description}
\end{enumerate}
\subsection{Le son [n] comme dans les mots anglais}
\label{sec:org1fe12d0}
\begin{enumerate}
\item \href{http://www.wordreference.com/enfr/know}{know} qui s'écrit phonétiquement \href{https://en.oxforddictionaries.com/definition/know}{\emph{nəʊ}}. Exemple d'utilisation du
mot : 
\begin{description}
\item[{english}] \textenglish{I \href{https://youtu.be/j-CwwdwQV54}{know} a good restaurant nearby.}
\item[{français}] Je connais un bon restaurant à proximité.
\end{description}
\item \href{http://www.wordreference.com/enfr/nobody}{nobody} qui s'écrit phonétiquement \href{https://en.oxforddictionaries.com/definition/nobody}{\emph{ˈnəʊbədi}}. Exemple d'utilisation
de ce mot :
\begin{description}
\item[{english}] \textenglish{At the time \href{https://youtu.be/icE0AqVSnzo}{nobody} could have known that it would take
six months.}
\item[{français}] Personne ne pouvait à ce moment savoir que ceci prendrait six mois.
\end{description}
\item \href{http://www.wordreference.com/enfr/funny}{funny} qui s'écrit phonétiquement \href{https://en.oxforddictionaries.com/definition/funny}{\emph{ˈfʌni}}. Exemple d'utilisation du
mot : 
\begin{description}
\item[{english}] \textenglish{All visitors, especially the children, found the clown
\href{https://youtu.be/CNXOu7gPEXM}{funny}.}
\item[{français}] Tous les visiteurs, surtout les enfants, ont trouvé
le clown amusant.
\end{description}
\item \href{http://www.wordreference.com/enfr/turn}{turn} qui s'écrit phonétiquement \href{https://en.oxforddictionaries.com/definition/turn}{\emph{təːn}}. Exemple d'utilistion de ce
mot : 
\begin{description}
\item[{english}] \textenglish{The negotiations have taken a decisive \href{https://youtu.be/z4g45vTgczE}{turn} today.}
\item[{français}] Les négociations ont pris un tournant décisif
aujourd'hui.
\end{description}
\end{enumerate}
\subsection{Le son [ŋ] comme dans les mots anglais}
\label{sec:org4b41268}
\begin{enumerate}
\item \href{http://www.wordreference.com/enfr/anger}{anger} qui s'écrit phonétiquement \href{https://en.oxforddictionaries.com/definition/anger}{\emph{ˈaŋɡə}}. Exemple d'utilisation de
ce mot : 
\begin{description}
\item[{english}] \textenglish{He suddenly unleashed his \href{https://youtu.be/lw64e7JVRj0}{anger}.}
\item[{français}] Il a soudainement déchaîné sa colère.
\end{description}
\item \href{http://www.wordreference.com/enfr/bang}{bang} qui s'écrit phonétiquement \href{https://en.oxforddictionaries.com/definition/bang}{\emph{baŋ}}. Exemple d'utilisation de ce
mot : 
\begin{description}
\item[{english}] \textenglish{A loud \href{https://youtu.be/N-AgYXz2n9Y}{bang} woke me up in the middle of the night.}
\item[{français}] Un grand fracas m'a réveillé en pleine nuit.
\end{description}
\item \href{http://www.wordreference.com/enfr/king}{king} qui s'écrit phonétiquement \href{https://en.oxforddictionaries.com/definition/king}{\emph{kɪŋ}}. Exemple d'utilisation de ce
mot : 
\begin{description}
\item[{english}] \textenglish{The \href{https://youtu.be/MRgFeZa\_I48}{king} and queen live in a magnificent palace.}
\item[{français}] Le roi et la reine habitent dans un palais magnifique.
\end{description}
\item \href{http://www.wordreference.com/enfr/thanks}{thanks} qui s'écrit phonétiquement \href{https://en.oxforddictionaries.com/definition/thanks}{\emph{θaŋks}}. Exemple d'utilisation de
ce mot :
\begin{description}
\item[{english}] \textenglish{We solved the problem \href{https://youtu.be/hQiipuDbbxw}{thanks} to a concerted effort.}
\item[{français}] Nous avons résolu le problème grâce à un effort
concerté.
\end{description}
\end{enumerate}
\section{Approximant (bloc d'air partiel, semblable à une voyelle)}
\label{sec:org222f673}
\subsection{Le son [w] comme dans les mots anglais}
\label{sec:org6303a28}
\begin{enumerate}
\item \href{http://www.wordreference.com/enfr/one}{one} qui s'écrit phonétiquement \href{https://en.oxforddictionaries.com/definition/one}{\emph{wʌn}}. Exemple d'utilisation du
mot : 
\begin{description}
\item[{english}] \textenglish{\href{https://youtu.be/aSNJ00iAZ7I}{One} never knows where to begin, so let's start with
the number \href{https://youtu.be/jHRXlK2SnQ8}{one}.}
\item[{français}] On ne sait jamais par où commencer, alors commençons
par le numéro un..
\end{description}
\item \href{http://www.wordreference.com/enfr/queen}{queen} qui s'écrit phonétiquement \href{https://en.oxforddictionaries.com/definition/queen}{\emph{kwiːn}}. Exemple d'utilisation de
ce mot :
\begin{description}
\item[{english}] \textenglish{The \href{https://youtu.be/Jmd4OLzhQw0}{queen} chooses her entourage very carefully.}
\item[{français}] La reine choisit son entourage très soigneusement.
\end{description}
\item \href{http://www.wordreference.com/enfr/wall}{wall} qui s'écrit phonétiquement \href{https://en.oxforddictionaries.com/definition/wall}{\emph{wɔːl}}. Exemple d'utilisation de ce
mot : 
\begin{description}
\item[{english}] \textenglish{The shelf is attached to the \href{https://youtu.be/BN5Z28Dfl7o}{wall}.}
\item[{français}] L'étagère est fixée au mur.
\end{description}
\item \href{http://www.wordreference.com/enfr/world}{world} qui s'écrit phonétiquement \href{https://en.oxforddictionaries.com/definition/world}{\emph{wəːld}}. Exemple d'utilisation de
ce mot :
\begin{description}
\item[{english}] \textenglish{The company entered the \href{https://youtu.be/fzDft0DZRUw}{world} market with great
success.}
\item[{français}] L'entreprise est entrée sur le marché mondial avec
grand succès.
\end{description}
\end{enumerate}
\subsection{Le son [j] comme dans les mots anglais}
\label{sec:orgacfdead}
\begin{enumerate}
\item \href{http://www.wordreference.com/enfr/beauty}{beauty} qui s'écrit phonétiquement \href{https://en.oxforddictionaries.com/definition/beauty}{\emph{ˈbjuːti}}. Exemple d'utilisation
de ce mot :
\begin{description}
\item[{english}] \textenglish{The actress is the embodiment of talent and \href{https://youtu.be/IzwWXNxFiyA}{beauty}.}
\item[{français}] L'actrice est l'incarnation du talent et de la
beauté.
\end{description}
\item \href{http://www.wordreference.com/enfr/few}{few} qui s'écrit phonétiquement \href{https://en.oxforddictionaries.com/definition/few}{\emph{fjuː}}. Exemple d'utilisation de ce
mot : 
\begin{description}
\item[{english}] \textenglish{I gave my friend a \href{https://youtu.be/6E2hYDIFDIU}{few} tips to save money.}
\item[{français}] J'ai donné quelques conseils à mon ami pour
économiser de l'argent.
\end{description}
\item \href{http://www.wordreference.com/enfr/usual}{usual} qui s'écrit phonétiquement \href{https://en.oxforddictionaries.com/definition/usual}{\emph{ˈjuːʒʊəl}}. Exemple d'utilisation
du mot :
\begin{description}
\item[{english}] \textenglish{My mother made her \href{https://youtu.be/ThLRPCs8uzc}{usual} cake for my birthday.}
\item[{français}] Ma mère a fait son gâteau traditionnel pour mon
anniversaire.
\end{description}
\item \href{http://www.wordreference.com/enfr/yellow}{yellow} qui s'écrit phonétiquement \href{https://en.oxforddictionaries.com/definition/yellow}{\emph{ˈjɛləʊ}}. Exemple d'utilisation
du mot :
\begin{description}
\item[{english}] \textenglish{We all live in a \href{https://youtu.be/m2uTFF\_3MaA}{yellow} \href{https://www.lacoccinelle.net/245633.html}{submarine}.}
\item[{français}] Nous vivons tous dans un sous-marin jaune.
\end{description}
\end{enumerate}
\subsection{Le son [r] comme dans les mots anglais}
\label{sec:org5353d95}
\begin{enumerate}
\item \href{http://www.wordreference.com/enfr/arrange}{arrange} qui s'écrit phonétiquement \href{https://en.oxforddictionaries.com/definition/arrange}{\emph{əˈreɪn(d)ʒ}}. Exemple
d'utilisation de ce mot :
\begin{description}
\item[{english}] \textenglish{We can \href{https://youtu.be/oD5RzpwbrIc}{arrange} another meeting if necessary.}
\item[{français}] Nous pouvons organiser une autre réunion si
nécessaire.
\end{description}
\item \href{http://www.wordreference.com/enfr/road}{road} qui s'écrit phonétiquement \href{https://en.oxforddictionaries.com/definition/road}{\emph{rəʊd}}. Exemple d'utilisation de ce
mot :
\begin{description}
\item[{english}] \textenglish{The \href{https://youtu.be/bO2xMNU9bTw}{road} passes through the forest.}
\item[{français}] La route passe par la forêt.
\end{description}
\item \href{http://www.wordreference.com/enfr/sorry}{sorry} qui s'écrit phonétiquement \href{https://en.oxforddictionaries.com/definition/sorry}{\emph{ˈsɒri}}. Exemple d'utilisation de
ce mot :
\begin{description}
\item[{english}] \textenglish{I am \href{https://youtu.be/ahCwKDyS5OE}{sorry} for any inconvenience I may have caused.}
\item[{français}] Je suis désolé pour tout inconvénient que j'ai pu causer.
\end{description}
\item \href{http://www.wordreference.com/enfr/wrong}{wrong} qui s'écrit phonétiquement \href{https://en.oxforddictionaries.com/definition/wrong}{\emph{rɒŋ}}. Exemple d'utilisation de ce
mot :
\begin{description}
\item[{english}] \textenglish{There are no \href{https://youtu.be/a5e0z1\_uwHY}{wrong} answers to this question.}
\item[{français}] Il n'y a pas de mauvaises réponses à cette question.
\end{description}
\end{enumerate}
\subsection{Le son [l] comme dans les mots anglais}
\label{sec:org273e167}
\begin{enumerate}
\item \href{http://www.wordreference.com/enfr/feel}{feel} qui s'écrit phonétiquement \href{https://en.oxforddictionaries.com/definition/feel}{\emph{fiːl}}. Exemple d'utilisation de ce
mot : 
\begin{description}
\item[{english}] \textenglish{I \href{https://youtu.be/4k4SP01l6rY}{feel} \href{https://youtu.be/DuDeBcpLITQ}{good} because I slept well.}
\item[{français}] Je me sens bien car j'ai bien dormi.
\end{description}
\item \href{http://www.wordreference.com/enfr/law}{law} qui s'écrit phonétiquement \href{https://en.oxforddictionaries.com/definition/law}{\emph{lɔː}}. Exemple d'utilisation de ce
mot :
\begin{description}
\item[{english}] \textenglish{I want to become a judge, so I have to study \href{https://youtu.be/GEy6ThJwE3s}{law}.}
\item[{français}] Je souhaite devenir juge, je dois donc étudier le droit.
\end{description}
\item \href{http://www.wordreference.com/enfr/light}{light} qui s'écrit phonétiquement \href{https://en.oxforddictionaries.com/definition/light}{\emph{lʌɪt}}. Exemple d'utilisation de ce
mot :
\begin{description}
\item[{english}] \textenglish{Light attracts \href{https://youtu.be/tSkCqj6T\_NQ}{moths}.}
\item[{français}] La lumière attire les papillons de nuit.
\end{description}
\item \href{http://www.wordreference.com/enfr/valley}{valley} qui s'écrit phonétiquement \href{https://en.oxforddictionaries.com/definition/valley}{\emph{ˈvali}}. Exemple d'utilisation de
ce mot :
\begin{description}
\item[{english}] \textenglish{This beautiful \href{https://youtu.be/wQFgG\_HsI0w}{valley} is covered with flowers.}
\item[{français}] Cette superbe vallée est recouverte de fleurs.
\end{description}
\end{enumerate}
\subsection{Le son [ɫ] comme dans les mots anglais}
\label{sec:org78aa354}
\begin{enumerate}
\item \href{http://www.wordreference.com/enfr/full}{full} qui s'écrit phonétiquement \href{https://home.cc.umanitoba.ca/\~krussll/phonetics/narrower/dark-l.html}{\emph{fʊɫ} or \emph{fɫ̩\}}}. Exemple
d'utilisation du mot :
\begin{description}
\item[{english}] \textenglish{His \href{https://youtu.be/TLmlgCteCMw}{full} name appears on his passport.}
\item[{français}] Son nom complet figure sur son passeport.
\end{description}
\item \href{http://www.wordreference.com/enfr/flail}{flail} qui s'écrit phonétiquement \href{https://home.cc.umanitoba.ca/\~krussll/phonetics/narrower/dark-l.html}{\emph{fleɫ}}. Exemple d'utilisation de
ce mot :
\begin{description}
\item[{english}] \textenglish{In Europe, a rope \href{https://youtu.be/AGf7n7iUF\_k}{flail} has been tried with some
success.}
\item[{français}] En Europe, un fléau de corde a fait l'objet d'essais
qui ont été raisonnablement satisfaisants.
\end{description}
\item \href{http://www.wordreference.com/enfr/little}{little} qui s'écrit phonétiquement \href{https://home.cc.umanitoba.ca/\~krussll/phonetics/narrower/dark-l.html}{\emph{ˈlɪɾɫ̩\}}}. Exemple d'utilisation
de ce mot :
\begin{description}
\item[{english}] \textenglish{The car salesman offers many options at \href{https://youtu.be/FokJGK639R4}{little} cost.}
\item[{français}] Le vendeur de voitures propose de nombreuses options à faible coût.
\end{description}
\item \href{http://www.wordreference.com/enfr/milk}{milk} qui s'écrit phonétiquement \href{https://home.cc.umanitoba.ca/\~krussll/phonetics/narrower/dark-l.html}{\emph{mɪɫk}}. Exemple d'utilisation de ce
mot :
\begin{description}
\item[{english}] \textenglish{The farmer \href{https://youtu.be/XBOE2CC0YYY}{milks} his cows every morning.}
\item[{français}] Le fermier trait ses vaches tous les matins.
\end{description}
\end{enumerate}