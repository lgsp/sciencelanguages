\chapter{Transcriptions phonétiques en 4 parties}

Voici les deux seules et uniques pages données dans la méthode nommée \ad. Elles
regroupent les sons en 4 catégories :
\begin{enumerate}
\item Voyelles brèves
\item Voyelles allongées
\item Voyelles doubles (diphtongues)
\item Consonnes  
\end{enumerate}

Tous les éléments présentés dans les tableaux suivants sont cliquables
et permettent ainsi de retrouver la section consacré au son pour plus
de détails.

\notation

Lorsque certains choix ont été fait, comme par exemple le
problématique \phon{e}, j'ai conservé celles qui sont dans le livre
pour montrer le problème concret évoqué plus haut dans mon livre.

\newpage

\section{Voyelles brèves}

\begin{center}
  \begin{table}[h]
    \centering
    \begin{tabular}[t]{ccc}
      API                         & Mot exemple  & Transcription \\ \\
      \hyperlink{soni}{\phon{ɪ}}  & \oxford{sit} & \wordref{sit}{sɪt}\\ \\
      \hyperlink{sonae}{\phon{æ}} & \oxford{cat} & \wordref{cat}{kæt}\\ \\
      \hyperlink{oa}{\phon{ɒ}}    & \oxford{shop}& \wordref{shop}{ʃɒp}\\ \\
      \hyperlink{omega}{\phon{ʊ}} & \oxford{put} & \wordref{put}{pʊt}\\ \\
      \hyperlink{sone}{\phon{e}}  & \oxford{ten} & \wordref{ten}{ten}\\ \\
      \hyperlink{sonup}{\phon{ʌ}} & \oxford{cup} & \wordref{cup}{kʌp}\\ \\
      \hyperlink{sonenv}{\phon{ə}}& \oxford{ago} & \wordref{ago}{əgəʊ}\\ \\
    \end{tabular}
    \caption{Voyelles brèves}
    \label{fig:voybrev}
  \end{table}
\end{center}

\newpage

\section{Voyelles allongées}

\begin{center}
  \begin{table}[h]
    \centering
    \begin{tabular}[t]{ccc}
      API                                & Mot exemple   & Transcription \\ \\
      \hyperlink{ilong}{\phon{iː}}       & \oxford{tea}  & \wordref{tea}{tiː}\\ \\
      \hyperlink{sonalong}{\phon{aː}}    & \oxford{car}  & \wordref{car}{kaː}\\ \\
      \hyperlink{oouvert}{\phon{ɔː}}     & \oxford{ball} & \wordref{ball}{bɔːl}\\ \\
      \hyperlink{ulong}{\phon{uː}}       & \oxford{boot} & \wordref{boot}{buːt}\\ \\
      \hyperlink{sonenvlong}{\phon{ɜː}}  & \oxford{bird} & \wordref{bird}{bɜːd}\\ \\
    \end{tabular}
    \caption{Voyelles allongées}
    \label{fig:voylong}
  \end{table}
\end{center}

\section{Voyelles doubles (diphtongues)}

\begin{center}
  \begin{table}[h]
    \centering
    \begin{tabular}[t]{ccc}
      API                       & Mot exemple    & Transcription \\\\
      \hyperlink{ai}{\phon{aɪ}} & \oxford{buy}   & \wordref{buy}{baɪ}\\\\
      \hyperlink{ei}{\phon{eɪ}} & \oxford{day}   & \wordref{day}{deɪ}\\\\
      \hyperlink{oi}{\phon{ɔɪ}} & \oxford{boy}   & \wordref{boy}{ɔɪ}\\\\
      \hyperlink{ao}{\phon{aʊ}} & \oxford{brown} & \wordref{brown}{braʊn}\\\\
      \hyperlink{eo}{\phon{əʊ}} & \oxford{no}    & \wordref{no}{nəʊ}\\\\
      \hyperlink{ie}{\phon{ɪə}} & \oxford{beer}  & \wordref{beer}{bɪə}\\\\
      \hyperlink{oe}{\phon{ʊə}} & \oxford{tour}  & \wordref{beer}{tʊə}\\\\
      \hyperlink{ee}{\phon{eə}} & \oxford{air}   & \wordref{air}{eə}\\\\
    \end{tabular}
    \caption{Diphtongues}
    \label{fig:diphtong}
  \end{table}
\end{center}

\newpage

\section{Consonnes}

\begin{center}
  \begin{table}[h]
    \centering
    \begin{tabular}[t]{ccc}
      API                       & Mot exemple     & Transcription \\\\
      \hyperlink{th}{\phon{ð}}  & \oxford{this}   & \wordref{this}{ðɪs}\\\\
      \hyperlink{ss}{\phon{θ}}  & \oxford{thin}   & \wordref{thin}{θin}\\\\
      \hyperlink{ing}{\phon{ŋ}} & \oxford{sing}   & \wordref{sing}{siŋ}\\\\
      \hyperlink{ez}{\phon{ʒ}}  &\oxford{pleasure}&\wordref{pleasure}{pleʒə}\\\\
      \hyperlink{dj}{\phon{dʒ}} & \oxford{jam}    & \wordref{jam}{dʒam}\\\\
      \hyperlink{ch}{\phon{ʃ}}  & \oxford{shoe}   & \wordref{shoe}{ʃuː}\\\\
      \hyperlink{tch}{\phon{tʃ}}& \oxford{chips}  & \wordref{chips}{tʃips}\\\\
      \hyperlink{h}{\phon{h}}   & \oxford{hat}    & \wordref{hat}{hat}\\\\
    \end{tabular}
    \caption{Consonnes}
    \label{fig:cons}
  \end{table}
\end{center}

\chapter{L'alphabet anglais avec la phonétique de chaque lettre}

Au lieu de vous proposer la classification phonétique du
\besch en 4 parties (identiques à la précédente classification), je
vous propose leur alphabet phonétique avec en bonus un lien vers un article
de mon
\href{http://doyouspeakenglish.fr/english-alphabet/}{blog}\footnote{\url{http://doyouspeakenglish.fr/english-alphabet/}}
avec des vidéos. 

\begin{center}
  \begin{table}[h]
    \centering
    \begin{tabular}[t]{cccccc}
      \letr{A}{ei}{eɪ}& \letr{B}{ilong}{biː}& \letr{C}{ilong}{siː}&%
                                                                    \letr{D}{ilong}{diː}& \letr{E}{ilong}{iː}& \letr{F}{sone}{ef}\\
      \\
      \letr{G}{dj}{dʒiː}& \letr{H}{tch}{eɪtʃ}& \letr{I}{ai}{aɪ}&%
                                                                 \letr{J}{dj}{dʒeɪ}& \letr{K}{ei}{keɪ}& \letr{L}{sone}{el}\\
      \\
      \letr{M}{sone}{em}& \letr{N}{sone}{en}& \letr{O}{eo}{əʊ}&%
                                                                \letr{P}{ilong}{piː}& \letr{Q}{ulong}{kjuː}& \letr{R}{sonalong}{aː}\\
      \\
      \letr{S}{sone}{es}& \letr{T}{ilong}{tiː}& \letr{U}{ulong}{juː}&%
                                                                      \letr{V}{ilong}{viː}& \letr{W}{sonup}{dʌbljuː}& \letr{X}{sone}{eks}\\
      \\
      \letr{Y}{ai}{waɪ}&%
                         \letr{Z}{sone}{zed} \uks{https://youtu.be/-TV4QpjgCE4}&%
                                                                                 \letr{Z}{ilong}{ziː} \uss{https://youtu.be/HysVxhemAe4}&&&\\
      \\
    \end{tabular}
    \caption{\exEN{English Alphabet}}
    \label{fig:alphabet}
  \end{table}
\end{center}

\chapter{Trucs et astuces}

Dans cette partie je vais vous livrer un ensemble de 10 trucs et astuces
pour améliorer votre niveau en anglais. C'est parti pour le top 10 !

\begin{enumerate}
  \item \'Ecoutez la bonne prononciation ! Normalement si vous avec
  cliquez sur les nombreux liens que je vous ai fourni vous avez dû au
  moins écouter les prononciations des dictionnaires Cambridge, Oxford
  et Wordreference. Ajouté à cela vous avez églament dû écouter des
  vidéos YouTube avec des américains et des britanniques. Et enfin des
  australiens grâce à YouGlish. Et bien je vous en donne encore deux autres qui
  s'appellent \href{http://www.howjsay.com/}{howjsay}\footnote{\url{http://www.howjsay.com/}}
  et \href{https://fr.forvo.com/}{forvo}\footnote{\url{https://fr.forvo.com/}}.
  \item Enregistrez-vous ! Que ça soit avec la webcam de votre
  ordinateur ou simplement le micro, avec votre smartphone, avec des
  apps comme parlez écoutez ou avec
  \href{https://vocaroo.com/}{Vocaroo}\footnote{\url{https://vocaroo.com/}}.
  Enregistrez votre voix et/ou idéalement votre visage également car
  les sons se construisent avec les muscles du visage.
  \item Parmi les vidéos que j'ai partagé avec vous, vous avez
    probablement dû remarquer que certaines revenez régulièrement
    comme par exemple Rachel. Il se trouve qu'elle a fait une
    excellente
    \href{https://www.youtube.com/playlist?list=PL27A5D7DE7D02373A}{playlist}
    que je vous recommande dans laquelle elle présente les exercices
    de Benjamin Franklin.
  \item Mettez en application les 30 astuces que je partage dans la
    playlist \href{https://www.youtube.com/playlist?list=PLfKvL-VUSKAnf4oZzkI3q24X4FJrGzcGr}{30 façons d'apprendre l'anglais}\footnote{\url{https://www.youtube.com/playlist?list=PLfKvL-VUSKAnf4oZzkI3q24X4FJrGzcGr}}.
  \item Testez vos connaissances de la culture anglo-saxonne à l'aide
    de ce petit \href{https://www.quizz.biz/quizz-436703.html}{quizz}\footnote{\url{https://www.quizz.biz/quizz-436703.html}}.
  \item Testez vos connaissances des expressions idiomatiques très
    présentes en anglais avec ce    \href{https://www.quizz.biz/quizz-918091.html}{quizz}\footnote{\url{https://www.quizz.biz/quizz-918091.html}}.
  \item Amusez-vous ! Par exemple avec ce site de jeux conçus
    spécialement pour apprendre l'anglais
    \url{https://www.gamestolearnenglish.com/} ou si vous voulez
    monter en puissance
    \href{https://www.boatloadpuzzles.com/playcrossword}{Boatload
      Puzzles}\footnote{Attention ils sont très durs à finir !
      \url{https://www.boatloadpuzzles.com/playcrossword}} ou encore
    \url{http://games.usatoday.com/category/puzzles/}.
  \item Faites attention aux erreurs communes des francophones comme
      exipliqué    \href{https://pronunciationstudio.com/french-speakers-english-pronunciation-errors/}{ici}\footnote{\url{https://pronunciationstudio.com/french-speakers-english-pronunciation-errors/}}.
    \item Tenez compte des différences malgré les points communs
        comme expliqué dans cet        \href{http://esl.fis.edu/grammar/langdiff/french.htm}{article}\footnote{\url{http://esl.fis.edu/grammar/langdiff/french.htm}}.
      \item Prenez garde aux nombreuses nuances qui existent en
        anglais mais pas en français comme expliqué \href{https://www.eupedia.com/europe/missing_words_french.shtml}{ici}\footnote{\url{https://www.eupedia.com/europe/missing_words_french.shtml}}.
\end{enumerate}

\newpage

\chapter{Les sons de l'anglais américain qui n'existent pas en
  français}\label{chap:sonricain}

\section{Consonnes}

\begin{center}
  \begin{table}[h]
    \centering
    \begin{tabular}[t]{ccc}
      API                       & Vidéo exemple   & Transcription\\\\
      \hyperlink{th}{\phon{ð}}  & \uss{https://youtu.be/UudCgpMubF0} & \wordref{father}{ˈfɑːðə}\\\\
      \hyperlink{ss}{\phon{θ}}  & \uss{https://youtu.be/76Nsqo0utJk}  & \wordref{think}{θɪŋk}\\\\
      \hyperlink{dl}{\phon{ɫ}}  & \uss{https://youtu.be/Q2yvSja9G98}   & \wordref{milk}{mɪɫk}\\\\
      \hyperlink{r}{\phon{ɹ}}   &\uss{https://youtu.be/Jq_yIbrD01c}   &\wordref{proud}{pɹaʊd}\\\\
      \hyperlink{h}{\phon{h}}   & \uss{https://youtu.be/uOG-4ZjR7ic}  & \wordref{honey}{hʌnɪ}\\\\
    \end{tabular}
    \caption[\exEN{American Consonants}]{Consonnes qui n'existent pas
      en français}
    \label{fig:uscons}
  \end{table}
\end{center}

\newpage
\section{Voyelles}

\begin{center}
  \begin{table}[h]
    \centering
    \begin{tabular}[t]{ccc}
      API                        & Vidéo exemple   & Transcription\\\\
      \hyperlink{sonae}{\phon{æ}}& \uss{https://youtu.be/mynucZiy-Ug} & \wordref{cat}{kæt}\\\\
      \hyperlink{soni}{\phon{ɪ}} & \uss{https://youtu.be/76Nsqo0utJk}  & \wordref{sit}{sɪt}\\\\
      \phon{ɝ}  & \uss{https://youtu.be/6ppOrwjvslc}   & \wordref{first}{fɝst}\\\\
      \phon{ɚ}  & \uss{https://youtu.be/AzNRoSGBh44}   & \wordref{enter}{ˈɛntɚ}\\\\
      \hyperlink{omega}{\phon{ʊ}} & \uss{https://youtu.be/kOOu8RpjlmU} & \wordref{put}{pʊt}\\ \\
    \end{tabular}
    \caption{Voyelles qui n'existent pas en français}
    \label{fig:usvow}
  \end{table}
\end{center}

Les sons \phon{ɝ} et \phon{ɚ} font partis de ce que les americains
appellent les \href{https://www.youtube.com/watch?v=3XRTN5gW4oU\&list=PLpEsvqK-KhCtodZfH9Gc6VfuNsnnKKKXU}{R-colored} vowels sounds\CW{https://en.wikipedia.org/wiki/General_American\#R-colored_vowels}.

\newpage
\section{Diphtongues}

\begin{center}
  \begin{table}[h]
    \centering
    \begin{tabular}[t]{ccc}
      API                       & Mot exemple    & Transcription \\\\
      \hyperlink{ai}{\phon{aɪ}} & \uss{https://youtu.be/8uD-GuuSgyk}   & \wordref{buy}{baɪ}\\\\
      \hyperlink{ei}{\phon{eɪ}} & \uss{https://youtu.be/XOuD6mFr6sQ}   & \wordref{day}{deɪ}\\\\
      \hyperlink{oi}{\phon{ɔɪ}} & \uss{https://youtu.be/ZfjPBN22mK8}   & \wordref{boy}{ɔɪ}\\\\
      \hyperlink{ao}{\phon{aʊ}} & \uss{https://youtu.be/i8KThVR713Q} &
                                                                       \wordref{brown}{braʊn}\\\\
      \hyperlink{oohm}{\phon{oʊ}} & \uss{https://youtu.be/BntgihRLSCA} & \wordref{go}{goʊ}\\\\
    \end{tabular}
    \caption{Diphtongues américaines}
    \label{fig:usdiphtong}
  \end{table}
\end{center}

\newpage

\chapter{Les outils concrets pour améliorer votre phonétique}

Dans cette partie je vais vous donner une liste (forcément non
exhaustive) d'outils concrets pour améliorer votre pratique. Alors
c'est parti !

\begin{enumerate}
\item \href{http://upodn.com/}{upodn} ce site permet de convertir du
  texte en transcription phonétique.
\item \href{https://tophonetics.com/}{tophonetics} il fait la même
  chose que le précédent sauf qu'il a (au moins) trois avantages. Tout
  d'abord il permet de changer la langue de l'interface. Ensuite il
  permet de traduire en anglais britannique ou en anglais américain ce
  qui est très pratique. Enfin il existe une application mobile
  disponible sur l'App Store et Google Play.
\item \href{https://easypronunciation.com/fr/}{easypronunciation} ce
  site présente encore plus d'avantages que le précédent parce qu'il
  permet aussi de transcrire dans d'autres langues que l'anglais.
\item \href{http://www.photransedit.com/}{photransedit} le site est
  totalement en anglais. Par contre il a (au moins) trois atouts
  considérables. Le premier est qu'il dispose d'une version de bureau
  que vous pouvez installer sur votre ordinateur. Le deuxième est que
  sa version \exEN{online} se déciline en trois parties : texte,
  clavier et bibliothèque de textes déjà transcrits\footnote{Et rien
    que ça, ça vaut de l'or !}. Enfin son troisième atout est qu'il
  dispose de nombreuses ressources vers des blogs et sites web
  spécialisés dans la phonétique.
\item \href{http://www.phonemicchart.com/}{phonemicchart} propose
  moins d'options. En revanche il propose un tableau bien pratique
  lorsque qu'on a besoin d'écrire les symboles phonétiques qui ne sont
  pas forcément aisés à taper au clavier. De plus il dispose d'un
  moteur de transcription (uniquement pour des mots) et de nombreuses
  ressources complémentaires (notamment des \exEN{flashcards}).
\end{enumerate}

\newpage

\chapter{Un peu de poésie}

Voici de la poésie visuelle\footnote{Pour plus de poésie
        américaine cliquer \href{https://amzn.to/2F96Hs0}{ici}.}.
\begin{center}
  \begin{figure}[h]
    \centering
    {\small
\begin{verbatim}
                          Dusk
                        Above the
                   water hang the
                             loud
                            flies
                            Here
                           O so
                          gray
                         then
                        What              A pale signal will appear
                       When          Soon before its shadow fades
                      Where        Here in this pool of opened eye
                       In us     No Upon us As at the very edges
                        of where we take shape in the dark air
                         this object bares its image awakening
                           ripples of recognition that will
                              brush darkness up into light
even after this bird this hour both drift by atop the perfect sad instant now
                              already passing out of sight
                           toward yet-untroubled reflection
                         this image bears its object darkening
                        into memorial shades Scattered bits of
                       light     No of water Or something accross
                       water       Breaking up No Being regathered
                        soon         Yet by then a swan will have 
                         gone             Yes out of mind into what
                          vast 
                           pale
                            hush
                             of a 
                             place 
                              past 
                    sudden dark as
                         if a swan
                            sang
\end{verbatim}
      }

    \caption[\exEN{Swan and
      Shadow}]{\href{https://emacs.stackexchange.com/questions/18011/poem-in-org-mode-verse-environment-mangled-by-latex-export}{\exEN{John
          Hollander's Swan and Shadow}}}
    \label{fig:swan}
  \end{figure}
\end{center}

\newpage

% \chapter{Un peu de poésie version phonétique}

% Voici de la poésie visuelle\footnote{Pour plus de poésie
%         américaine cliquer \href{https://amzn.to/2F96Hs0}{ici}.} en
%       version phonétique américaine puisque l'auteur \href{https://fr.wikipedia.org/wiki/John_Hollander}{John
%         Hollander} était américain.
% \begin{center}
%   \begin{figure}[h]
%     \centering
%     {\small
% \begin{verbatim}
%                           dʌsk 
%                         əˈbʌv ði 
%                    ˈwɔtər hæŋ ði 
%                              laʊd 
%                             flaɪz 
%                             hir 
%                            oʊ soʊ 
%                           greɪ 
%                          ðɛn 
%                         wʌt              ə peɪl ˈsɪgnəl wɪl əˈpɪr 
%                        wɛn          sun bɪˈfɔr ɪts ˈʃæˌdoʊ feɪdz 
%                       wɛr        hir ɪn ðɪs pul əv ˈoʊpənd aɪ 
%                        ɪn ʌs     noʊ əˈpɔn əs əz ət ðə ˈvɛri ˈɛʤəz 
%                         əv wɛr wi teɪk ʃeɪp ɪn ðə dɑrk ɛr 
%                          ðɪs ˈɑbʤɛkt bɛrz ɪts ˈɪməʤ əˈweɪkənɪŋ 
%                            ˈrɪpəlz əv ˌrɛkəgˈnɪʃən ðət wɪl 
%                               brʌʃ ˈdɑrknəs ʌp ˈɪntə laɪt 
% ˈivɪn ˈæftər ðɪs bɜrd ðɪs ˈaʊər boʊθ drɪft baɪ əˈtɑp ðə ˈpɜrˌfɪkt sæd ˈɪnstənt naʊ 
%                               ɔlˈrɛdi ˈpæsɪŋ aʊt əv saɪt 
%                            təˈwɔrd jɛt-ənˈtrʌbəld rəˈflɛkʃən 
%                          ðɪs ˈɪməʤ bɛrz ɪts ˈɑbʤɛkt ˈdɑrkənɪŋ 
%                         ˈɪntə məˈmɔriəl ʃeɪdz ˈskætərd bɪts ʌv  
%                        laɪt     noʊ əv ˈwɔtər ɔr ˈsʌmθɪŋ əˈkrɔs 
%                        ˈwɔtər       ˈbreɪkɪŋ ʌp noʊ ˈbiɪŋ ˌriˈgæðərd 
%                         sun         jɛt baɪ ðɛn ə swɑn wɪl hæv  
%                          gɔn             jɛs aʊt əv maɪnd ˈɪntə wʌt  
%                           væst  
%                            peɪl 
%                             hʌʃ 
%                              əv eɪ 
%                              pleɪs 
%                               pæst 
%                     ˈsʌdən dɑrk æz 
%                          ɪf ə swɑn 
%                             sæŋ 
% \end{verbatim}
%       }

%     \caption[\exEN{Swan and
%       Shadow}]{\href{https://www.poetryfoundation.org/poets/john-hollander}{\exEN{John
%           Hollander's Swan and Shadow}}}
%     \label{fig:swan}
%   \end{figure}
% \end{center}

% \newpage

\chapter{Lorsque la poésie justifie l'étude de la phonétique}

Pour écouter ce poème rendez-vous sur cet
\href{http://doyouspeakenglish.fr/un-poeme-qui-justifie-letude-de-la-phonetique/}{article}
de mon blog.
\begin{itemize}
\item \exEN{Ration never rhymes with nation,}
\item \textcolor{teal}{ˈræʃən ˈnɛvə raɪmz wɪð ˈneɪʃən,}
\item \exEN{Say prefer, but preferable,}
\item \textcolor{teal}{seɪ priˈfɜː, bʌt ˈprɛfərəbl,}
\item \exEN{Comfortable and vegetable.}
\item \textcolor{teal}{ˈkʌmfətəbl ænd ˈvɛʤɪtəbl.}
\item \exEN{B must not be heard in doubt,}
\item \textcolor{teal}{biː mʌst nɒt biː hɜːd ɪn daʊt,}
\item \exEN{Debt and dumb both leave it out, }
\item \textcolor{teal}{dɛt ænd dʌm bəʊθ liːv ɪt aʊt,}
\item \exEN{In the words psychology,}
\item \textcolor{teal}{ɪn ðə wɜːdz saɪˈkɒləʤi,}
\item \exEN{Psychic, and psychiatry,}
\item \textcolor{teal}{ˈsaɪkɪk, ænd saɪˈkaɪətri,}
\item \exEN{You must never sound the p.}
\item \textcolor{teal}{juː mʌst ˈnɛvə saʊnd ðə piː.}
\item \exEN{Psychiatrist you call the man}
\item \textcolor{teal}{saɪˈkaɪətrɪst juː kɔːl ðə mæn}
\item \exEN{Who cures the complex, if he can.}
\item \textcolor{teal}{huː kjʊəz ðə ˈkɒmplɛks, ɪf hiː kæn.}
\item \exEN{In architect, chi is k.}
\item \textcolor{teal}{ɪn ˈɑːkɪtɛkt, ʧiː ɪz keɪ.}
\item \exEN{In arch it is the other way.}
\item \textcolor{teal}{ɪn ɑːʧ ɪt ɪz ði ˈʌðə weɪ.}
\item \exEN{Please remember to say iron}
\item \textcolor{teal}{pliːz rɪˈmɛmbə tə seɪ ˈaɪən}
\item \exEN{So that it'll rhyme with lion.}
\item \textcolor{teal}{səʊ ðət ˈɪtl raɪm wɪð ˈlaɪən.}
\item \exEN{Advertisers advertse,}
\item \textcolor{teal}{ˈædvətaɪzəz advertse,}
\item \exEN{Advertisements will put you wise.}
\item \textcolor{teal}{ədˈvɜːtɪsmənts wɪl pʊt jʊ waɪz.}
\item \exEN{Time when work is done is leisure,}
\item \textcolor{teal}{taɪm wɛn wɜːk s dʌn z ˈlɛʒə,}
\item \exEN{Fill it up with useful pleasure.}
\item \textcolor{teal}{fɪl ɪt ʌp wɪð ˈjuːsfʊl ˈplɛʒə.}
\item \exEN{Accidental, accident,}  
\item \textcolor{teal}{ˌæksɪˈdɛntl, ˈæksɪdənt,}
\item \exEN{Sound the g in ignorant.}
\item \textcolor{teal}{saʊnd ðə ʤiː ɪn ˈɪgnərənt.}
\item \exEN{Relative, but relation,}
\item \textcolor{teal}{ˈrɛlətɪv, bət rɪˈleɪʃən,}
\item \exEN{Then say creature, creation.}
\item \textcolor{teal}{ðɛn seɪ ˈkriːʧə, kri(ː)ˈeɪʃən.}
\item \exEN{Say the a in gas quite short,}
\item \textcolor{teal}{seɪ ði ə ɪn gæs kwaɪt ʃɔːt,}
\item \exEN{Bought remember rhymes with thwart,}
\item \textcolor{teal}{bɔːt rɪˈmɛmbə raɪmz wɪð θwɔːt,}
\item \exEN{Drought must always rhyme with bout,}
\item \textcolor{teal}{draʊt məst ˈɔːlweɪz raɪm wɪð baʊt,}
\item \exEN{In daughter leave the "gh" out.}
\item \textcolor{teal}{ɪn ˈdɔːtə liːv ðiː "gh" aʊt.}
\item \exEN{Wear a boot upon your foot.}
\item \textcolor{teal}{weər ə buːt əpən jə fʊt.}
\item \exEN{Root can never rhyme with soot.}
\item \textcolor{teal}{ruːt kən ˈnɛvə raɪm wɪð sʊt.}
\item \exEN{In muscle, sc is s,}
\item \textcolor{teal}{ɪn ˈmʌsl, sc z ɛs,}
\item \exEN{In muscular, it's sk, yes!}
\item \textcolor{teal}{ɪn ˈmʌskjʊlə, ɪts sk, jɛs!}
\item \exEN{Choir must always rhyme with wire,}
\item \textcolor{teal}{ˈkwaɪə məst ˈɔːlweɪz raɪm wɪð ˈwaɪə,}
\item \exEN{That again will rhyme with liar.}
\item \textcolor{teal}{ðæt əˈgɛn wɪl raɪm wɪð ˈlaɪə.}
\item \exEN{Then remember it's address.}
\item \textcolor{teal}{ðɛn rɪˈmɛmbər ɪts əˈdrɛs.}
\item \exEN{With an accent like posses.}
\item \textcolor{teal}{wɪð ən ˈæksənt laɪk ˈpɒsiz.}
\item \exEN{G in sign must be silent be,}
\item \textcolor{teal}{ʤiː ɪn saɪn məst bi ˈsaɪlənt biː,}
\item \exEN{In signature, pronounce the g.}
\item \textcolor{teal}{ɪn ˈsɪgnɪʧə, prəˈnaʊns ðə ʤiː.}
\item \exEN{Please remember, say towards}
\item \textcolor{teal}{pliːz rɪˈmɛmbə, seɪ təˈwɔːdz}
\item \exEN{Just as if it ryhmed with boards.}
\item \textcolor{teal}{dʒəst əz ɪf ɪt raɪmd wɪð bɔːdz}
\item \exEN{Weight's like wait, but not like height.}
\item \textcolor{teal}{weɪts laɪk weɪt, bət nɒt laɪk haɪt.}
\item \exEN{Which should always rhyme with might.}
\item \textcolor{teal}{wɪʧ ʃəd ˈɔːlweɪz raɪm wɪð maɪt.}
\item \exEN{Sew is just the same as so,}
\item \textcolor{teal}{səʊ z ʤəst ðə seɪm əz səʊ,}
\item \exEN{Tie a ribbon in a bow.}
\item \textcolor{teal}{taɪ ə ˈrɪbən ɪn ə baʊ.}
\item \exEN{When You meet the queen you bow,}
\item \textcolor{teal}{wɛn jʊ miːt ðə kwiːn jʊ baʊ,}
\item \exEN{Which again must rhyme with how.}
\item \textcolor{teal}{wɪʧ əˈgɛn məst raɪm wɪð haʊ.}
\item \exEN{In perfect English make a start.}
\item \textcolor{teal}{ɪn ˈpɜːfɪkt ˈɪŋglɪʃ meɪk ə stɑːt.}
\item \exEN{Learn this little rhyme by heart.}
\item \textcolor{teal}{lɜːn ðɪs ˈlɪtl raɪm baɪ hɑːt.}                    
\end{itemize}
