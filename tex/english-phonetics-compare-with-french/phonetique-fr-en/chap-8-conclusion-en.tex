\part{Conclusion}\label{chap:conc}

Voilà, nous sommes arrivés à la fin de ce voyage. Si je prends la
métaphore du voyage c'est à dessein parce que de la même manière que
l'on découvre une nouvelle destination pour la première fois il en va
de même pour un nouveau domaine tel que la phonétique. Lors du premier
voyage on est souvent impressionné, surpris mais à moins d'y passer
beaucoup de temps en général on ne vit qu'une expérience
superficielle. Selon le degré d'implication on peut déjà pressentir le
désir d'approfondir. Et c'est précisément mon but avec ce premier
ouvrage sur le sujet.

Après vous avoir présenté très brièvement les outils qui m'ont permis
d'écrire cet ouvrage, et pour lesquels je propose d'ailleurs des
formations (avis aux amateurs\footnote{Si vous êtes intéressés vous
  pouvez me contacter via le formulaire de contact de mon blog sur
  l'anglais \url{http://doyouspeakenglish.fr/contact/} et je vous
  répondrais avec plaisir.}), j'ai essayé de partager avec vous mon
goût prononcé pour découvrir la linguistique. En effet, j'ai voulu
partager avec vous ma joie de découvrir ce merveilleux domaine qui
embrasse toutes les disciplines possibles puisqu'il s'agit de l'étude
de l'outil que nous utilisons chaque jour pour communiquer, et ce,
quelle que soit notre activité.

Les plus pragmatiques et/ou impatients d'entre vous ont certainement
sauté directement sur la partie pratique qui consiste à étudier
concrètement comment écouter et essayer de formuler les sons
correctement. Avec les nombreux exemples que je vous ai fourni ainsi
que les liens innombrables je pense que vous pourrez utiliser ce guide
aussi longtemps qu'il sera possible d'héberger des vidéos et audios en
ligne\dots{} autant dire pendant très longtemps ! Ne vous jetez pas à
corps perdu dans l'ouvrage mais essayez plutôt d'organiser vos
apprentissage. En effet, je pense qu'une étude quotidienne par session
de 10 à 20-25 minutes peut véritablement considérablement augmenter
votre niveau. Le mot \underline{quotidien} a toute son importance
ici. Il est vraiment \textbf{fondamental} d'être régulier, c'est la
clé de la réussite. Je vous rappelle également que vous pouvez vous
inspirez des 30 astuces que j'ai partagé dans cette
\href{https://www.youtube.com/playlist?list=PLfKvL-VUSKAnf4oZzkI3q24X4FJrGzcGr}{playlist}\footnote{\url{https://www.youtube.com/playlist?list=PLfKvL-VUSKAnf4oZzkI3q24X4FJrGzcGr}}.

Prenez également le temps de consulter les nombreuses annexes que j'ai
richement fourni. Je pense en particulier aux outils concrets pour
vous permettre de transcrire instantannément les mots, groupes de
mots, phrases et même petits paragraphes en symboles phonétiques. Mais
n'oubliez pas que le but du jeu n'est pas d'apprendre les symboles par
c{\oe}ur, le but est d'être capable de percevoir les différences à
l'oreille et de produire les sons les plus compréhensibles possibles
pour vos interlocuteurs. Savoir écrire tous les symboles sans savoir
entendre ou produire les sons est inutile. Bien sûr on peut prendre du
plaisir à jouer avec ce nouvel ensemble de symbole mais cela doit
venir après la maîtrise de l'écoute et de la parole. Pratiquez,
pratiquez, et encore pratiquez\footnote{Les anglo-saxons disent
  \exEN{Practice makes perfect} à juste titre !}. 

D'ailleurs je suis d'ores et déjà motivé pour poursuivre mon
exploration de l'univers merveilleux de la phonétique. Comme je l'ai
déjà dit plus haut dans l'ouvrage l'API ne concerne pas que l'anglais,
il concerne toutes les langues humaines. Bien qu'il me serait
difficile de traiter toutes les langues du monde, je suis déjà en
cours de rédaction d'une introduction à la phonétique française. Et
une bonne partie des sons et symboles utilisés pour l'anglais sont
valables également pour le français. Parmi les
\href{https://www.youtube.com/playlist?list=PLfKvL-VUSKAnkBk88BAb3oq1MlGVnhwcY}{autres
  langues} que je compte explorer comme je l'ai déjà fait à plusieurs
reprises il y a :
\begin{itemize}
  \item  l'\href{https://www.youtube.com/playlist?list=PLfKvL-VUSKAnM9MWJT9F1z1QZTdb73i7r}{allemand}
  \item
    l'\href{https://www.youtube.com/playlist?list=PLfKvL-VUSKAkXu2x3Fp74QxxYUVP43haA}{arabe}
  \item le
    \href{https://www.youtube.com/playlist?list=PLfKvL-VUSKAl4R0Mh7sKvQjqCsiEEa6D9}{chinois}
  \item
    l'\href{https://www.youtube.com/playlist?list=PLfKvL-VUSKAm_p6ikI_pTbxNuHco73REt}{espagnol}
  \item
    l'\href{https://www.youtube.com/playlist?list=PLfKvL-VUSKAkbDhpbtXc7RdroMBBeTJx0}{hébreu}
  \item le
    \href{https://www.youtube.com/playlist?list=PLfKvL-VUSKAn0zUUPYsMDd8_1J_UtfRxh}{portugais}
  \item le \href{https://www.youtube.com/playlist?list=PLfKvL-VUSKAk0YrJ3rV6cBj-w6rNCeOJB}{russe}
  \end{itemize}

Pour chaque langue je proposerai des connexions avec la phonétique
anglaise et la phonétique française. C'est pour cette raison que j'ai
commencé par l'anglais et que j'enchaîne avec le français.

Vous trouverez en annexes différentes classifications des sons. En
effet, il existe de nombreuses façons de regrouper les sons et les
linguistes ne sont pas toujours d'accord entre eux pour savoir
laquelle serait <<~la meilleure~>>. Pour ma part, n'étant pas
linguiste professionnel et ayant une approche pragmatique je vous
recommanderais de ne pas accorder trop d'importance à ces
considérations. Comme je l'ai dit et répété ce qui compte c'est votre
capacité d'écoute et votre production sonore. C'est pour ça que le
document est particulièrement centré sur la connexion avec de
nombreuses sources variées\footnote{Cette richesse de documentation
  vidéo sera beaucoup moins abondante concernant les autres
  langues malheureusement.}.
Si vous souhaitez aller plus loin en anglais vous pouvez tout à fait
vous inscrire dans l'une de mes formations en passant par le
formulaire de contact sur mon blog
\url{http://doyouspeakenglish.fr/contact/}.

J'espère que ce livre vous aura aporté de la valeur et tous vos
commentaires seront les bienvenus que ça soit sur le blog en dessous
des articles ou sous les vidéos de la chaîne YouTube ou via le
formulaire de contact.

Merci pour votre attention et bon voyage au pays des langues.
