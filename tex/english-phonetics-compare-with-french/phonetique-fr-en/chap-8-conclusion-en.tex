\chapter{Ce que vous avez réussi à parcourir}\label{chap:success}

Bravo à vous, vous avez réussi à parcourir mon livre de bout en
bout. Voilà, nous sommes arrivés à la fin de ce voyage. Si je prends
la métaphore du voyage c'est à dessein parce que de la même manière
que l'on découvre une nouvelle destination pour la première fois il en
va de même pour un nouveau domaine tel que la phonétique. Lors du
premier voyage on est souvent impressionné, surpris mais à moins d'y
passer beaucoup de temps en général on ne vit qu'une expérience
superficielle. Selon le degré d'implication on peut déjà pressentir le
désir d'approfondir. Et c'est précisément mon but avec ce premier
ouvrage sur le sujet.

J'espère que cette première lecture vous aura donné envie de le relire
encore et encore. En effet, mon but est de vous avoir fourni un guide
qui vous servira autant de fois que nécessaire pour parfaire votre
utilisation et votre compréhension des sons de la langue anglaise.

Après vous avoir présenté très brièvement les outils qui m'ont permis
d'écrire cet ouvrage, et pour lesquels je propose d'ailleurs des
formations (avis aux amateurs\footnote{Si vous êtes intéressés vous
  pouvez me contacter via le formulaire de contact de mon blog sur
  l'anglais \url{http://doyouspeakenglish.fr/contact/} et je vous
  répondrais avec plaisir.}), j'ai essayé de partager avec vous mon
goût prononcé pour découvrir la linguistique.

En effet, j'ai voulu partager avec vous ma joie de découvrir ce
merveilleux domaine qui embrasse toutes les disciplines possibles
puisqu'il s'agit de l'étude de l'outil que nous utilisons chaque jour
pour communiquer, et ce, quelle que soit notre activité.

Les plus pragmatiques et/ou impatients d'entre vous ont certainement
sauté directement sur la partie pratique qui consiste à étudier
concrètement comment écouter et essayer de formuler les sons
correctement. Avec les nombreux exemples que je vous ai fourni ainsi
que les liens innombrables je pense que vous pourrez utiliser ce guide
aussi longtemps qu'il sera possible d'héberger des vidéos et audios en
ligne\dots{} autant dire pendant très longtemps !

\chapter{Et pour quelques conseils de plus}\label{chap:conseils}

\section{En quoi ce livre est différent des autres ?}\label{sec:diff}

Ce livre n'est pas un livre traditionnel ! Loin s'en faut, ce livre
tire toute sa force de notre capacité à être hyperconnecté à
l'Internet ou devrais-je dire aux internets. En effet, la force
principale de ce livre est de rassembler en un seul endroit de
nombreux outils, de nombreux liens qui vous seront très utiles tout au
long de votre apprentissage de l'anglais.

\section{Mise en garde contre une lecture linéaire}\label{sec:lin}

Ne vous jetez pas à corps perdu dans l'ouvrage mais essayez plutôt
d'organiser vos apprentissage. En effet, je pense qu'une étude
quotidienne par session de 10 à 20-25 minutes peut véritablement,
considérablement, augmenter votre niveau. Le mot \underline{quotidien}
revêt toute son importance ici. Il est vraiment \textbf{fondamental}
d'être régulier, c'est la clé de la réussite. Je vous rappelle
également que vous pouvez vous inspirez des 30 astuces que j'ai
partagé dans cette \href{https://www.youtube.com/playlist?list=PLfKvL-VUSKAnf4oZzkI3q24X4FJrGzcGr}{playlist}\footnote{\url{https://www.youtube.com/playlist?list=PLfKvL-VUSKAnf4oZzkI3q24X4FJrGzcGr}}.

N'ouvrez pas trop d'onglets à la fois, concentrez-vous sur quelques
sons à la fois. Je pense qu'au delà de 2 ou 3 par jour ça commence à
faire beaucoup. Bien sûr, vous êtes libre d'adapter la dose à votre
ressenti. Mais sachant que chaque son est présenté via 4 exemples avec
en plus de nombreux liens externes je pense qu'une étude d'un son par
jour me paraît déjà beaucoup. À vrai dire je pense qu'un son par
semaine est le meilleur équilibre.

\section{Plan d'actions}\label{sec:plan}

Voici un plan d'actions que je proposerais pour celles et ceux qui
veulent un guide précis pour tirer le meilleur parti de cet ouvrage
dans votre apprentissage.

\begin{enumerate}
\item Les jours 1 à 4 vous allez analyser chaque mot exemple qui
  illustre le son. Pour chaque mot construisez vos propres
  phrases. \'Ecrivez-les d'abord, lisez-les puis essayez d'improviser
  en les utilisant. Si les liens fournis vous dirigent vers des
  chansons essayez de fredonner l'air. Et si les musiques ne vous
  plaisent pas cherchez-en avec les mots proposés. Pour chaque jour
  étudier tous les liens qui se trouve dans la phrase exemple en
  question. Appropriez-vous chaque mot en essayant de construire vos
  propres exemples.
\item  Ensuite les jours 5 à 7 vous allez vous concentrer sur les variations
 possibles entre l'accent britannique, américain et australien grâce
 aux liens fournis par YouGlish. Pour les accents britanniques et
 américains notez et utilisez les mots donnés en exemples par les
 professeurs dans les vidéos. Ils enrichiront votre vocabulaire et
 votre base de groupes de mots associés au son en question.
\end{enumerate}

\section{Conseils généraux}\label{sec:gen}

Il est important de ne pas aller plus vite que la musique mais il est
aussi important de ne pas se focaliser sur un seul accent. Le monde
se rétrécit chaque jour et à moins que vous soyiez amené à vous fixer
dans une région en particulier, il vaut mieux prendre l'habitude
d'éduquer son oreille aux variations. En revanche pour votre
prononciation vous pouvez effectivement vous concentrer sur un accent
en particulier. A priori je vous conseillerais celui qui vous semble
le plus simple.  

Prenez également le temps de consulter les nombreuses annexes que j'ai
richement fourni. Je pense en particulier aux outils concrets pour
vous permettre de transcrire instantannément les mots, groupes de
mots, phrases et même petits paragraphes en symboles phonétiques. Mais
n'oubliez pas que le but du jeu n'est pas d'apprendre les symboles par
c{\oe}ur, le but est d'être capable de percevoir les différences à
l'oreille et de produire les sons les plus compréhensibles possibles
pour vos interlocuteurs. Savoir écrire tous les symboles sans savoir
entendre ou produire les sons est inutile. Bien sûr on peut prendre du
plaisir à jouer avec ce nouvel ensemble de symbole mais cela doit
venir après la maîtrise de l'écoute et de la parole. Pratiquez,
pratiquez, et encore pratiquez\footnote{Les anglo-saxons disent
  \exEN{Practice makes perfect} à juste titre !}. 

\chapter{Mes autres projets}\label{chap:projets}

\section{Pourquoi \TeX{} et d'où ça me vient ?}\label{sec:tex}

L'écriture de cet ouvrage m'a demandé un nombre incalculable d'heures
de travail. Et cela en un temps relativement réduit puisque j'ai dû y
passer environ 3 mois. C'est une véritable aventure qui m'a replongé
dans l'univers passionnant de la programmation \TeX{} que j'avais
découvert lorsque j'étais étudiant en mathématiques. À l'époque je
devais rédiger mon mémoire de travail d'étude et de recherche sur les
structures non euclidiennes des fonctions holomorphes à plusieurs
variables complexes\footnote{Quelque chose de ce genre là
  \url{https://www.math.u-psud.fr/~merker/Enseignement/Master-2-2017-2018/hartogs-separe.pdf}
pour ceux que ça intéressent ou encore \url{https://www.math.u-psud.fr/~perrin/Conferences/Romilly.pdf}.}\dots{} oui il s'agit bien de mots qui font tous partis de la
langue française mais qui sont inintelligibles pour le commun des
mortels. Preuve que l'on peut parler une autre langue tout en
utilisant officiellement la même langue. Mais je m'égare, ce n'est pas
l'objet de ce livre que de traiter de mathématiques qui sont selon moi
un langage tout aussi passionnant que le langage humain\footnote{Elles
feront l'objet de prochains livres à venir une fois que j'en aurais
fini avec ce premier cycle sur la phonétique articulatoire.}. \TeX{}
est un langage qui a été inventé par un enseignant-chercheur en
mathématiques et informatique et il est désormais utilisé par de
nombreux individus qui n'ont pas spécialement d'affinité avec les
mathématiques comme par exemple des chercheurs en sciences
humaines. En particulier, la communauté des linguistes s'est plutôt
bien investie\footnote{Comme on peut le voir ici
  \url{http://cl.indiana.edu/~md7/08/latex/slides.pdf} par exemple ou
  encore là \url{https://www.tug.org/TUGboat/tb25-1/peter.pdf} et
  enfin de façon plus succinte sur ce \href{https://allthingslinguistic.com/post/50042310246/what-is-latex-and-why-do-linguists-love-it}{blog}.} dans l'appropriation de cet \href{http://stefanocoretta.altervista.org/xelatex-linguistics/}{outil}.

\section{Mon projet Babel}\label{sec:babel}

D'ailleurs je suis d'ores et déjà motivé pour poursuivre mon
exploration de l'univers merveilleux de la phonétique\footnote{Et
  aussi de la linguistique en général parce que je trouve ça très
  intéressant et séduisant pour l'esprit et la pratique qui est
  beaucoup plus accessible au commun des mortels que les
  mathématiques, ce qui en fait un outil plus agréable pour échanger
  avec ses semblables bipèdes qu'on appelle encore Homo Sapiens
  Sapiens.}. Comme je l'ai déjà dit plus haut dans l'ouvrage l'API ne
concerne pas que l'anglais, il concerne toutes les langues
humaines. Bien qu'il me serait difficile de traiter toutes les langues
du monde, je suis déjà en cours de rédaction d'une introduction à la
phonétique française. Et une bonne partie des sons et symboles
utilisés pour l'anglais sont valables également pour le
français (donc si ça vous intéresse vous ne serez pas dépaysé). Parmi les \href{https://www.youtube.com/playlist?list=PLfKvL-VUSKAnkBk88BAb3oq1MlGVnhwcY}{autres
  langues} que je compte explorer comme je l'ai déjà fait à plusieurs
reprises il y a :
\begin{itemize}
  \item  l'\href{https://www.youtube.com/playlist?list=PLfKvL-VUSKAnM9MWJT9F1z1QZTdb73i7r}{allemand}
  \item
    l'\href{https://www.youtube.com/playlist?list=PLfKvL-VUSKAkXu2x3Fp74QxxYUVP43haA}{arabe}
  \item le
    \href{https://www.youtube.com/playlist?list=PLfKvL-VUSKAl4R0Mh7sKvQjqCsiEEa6D9}{chinois}
  \item
    l'\href{https://www.youtube.com/playlist?list=PLfKvL-VUSKAm_p6ikI_pTbxNuHco73REt}{espagnol}
  \item
    l'\href{https://www.youtube.com/playlist?list=PLfKvL-VUSKAkbDhpbtXc7RdroMBBeTJx0}{hébreu}
  \item le
    \href{https://www.youtube.com/playlist?list=PLfKvL-VUSKAn0zUUPYsMDd8_1J_UtfRxh}{portugais}
  \item le \href{https://www.youtube.com/playlist?list=PLfKvL-VUSKAk0YrJ3rV6cBj-w6rNCeOJB}{russe}
  \end{itemize}

Pour chaque langue je proposerai des connexions avec la phonétique
anglaise et la phonétique française. C'est pour cette raison que j'ai
commencé par l'anglais et que j'enchaîne avec le français.

\chapter{Quelques mots sur les annexes}\label{chap:mots-annex}

Vous trouverez en annexes différentes classifications des sons. En
effet, il existe de nombreuses façons de regrouper les sons et les
linguistes ne sont pas toujours d'accord entre eux pour savoir
laquelle serait <<~la meilleure~>>. Pour ma part, n'étant pas
linguiste professionnel et ayant une approche pragmatique je vous
recommanderais de ne pas accorder trop d'importance à ces
considérations. Comme je l'ai dit et répété ce qui compte c'est votre
capacité d'écoute et votre production sonore. C'est pour ça que le
document est particulièrement centré sur la connexion avec de
nombreuses sources variées\footnote{Cette richesse de documentation
  vidéo sera beaucoup moins abondante concernant les autres
  langues malheureusement.}.
Si vous souhaitez aller plus loin en anglais vous pouvez tout à fait
vous inscrire dans l'une de mes formations en passant par le
formulaire de contact sur mon blog
\url{http://doyouspeakenglish.fr/contact/}.

Vous trouverez également d'autres conseils et astuces pour améliorer
votre apprentissage et/ou pratique de l'anglais. J'ai aussi ajouté
quelques poèmes, dont un élaboré à des fins pédagogiques par le
British Council qui propose des applications de qualités, notamment
une sur la phonétique, et des
\href{https://learnenglish.britishcouncil.org/en/english-grammar}{outils
  en
  ligne}\footnote{\url{https://learnenglish.britishcouncil.org/en/english-grammar}}.

J'espère que ce livre vous aura aporté de la valeur et tous vos
commentaires seront les bienvenus que ça soit sur le blog en dessous
des articles ou sous les vidéos de la chaîne YouTube ou via le
formulaire de contact.

Merci pour votre attention et bon voyage au pays des langues.
