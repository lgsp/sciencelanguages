\chapter{Comment lire se livre}\label{chap:howto}
\newpage
\minitoc
\newpage

\section{Au commencement était le\dots lien}\label{sec:link}

Ceci est un \href{https://youtu.be/K88qlGcd7Ek}{lien}
URL\footnote{Voir Wikipédia :
  \url{https://fr.wikipedia.org/wiki/Uniform_Resource_Locator} si vous
  souhaitez en savoir plus tout de suite, sinon voir
  page~\pageref{sec:side} (et ça c'est un lien interne).} qui va vous diriger vers une vidéo qui
explique les bonnes raisons qui justifient l'acquisition de ce
livre. Vous pouvez également la retrouver sur la \href{http://doyouspeakenglish.fr/contact/}{page de contact} de mon
blog\footnote{La vidéo se trouve en bas de la page de contact.}. Normalement vous
devriez avoir compris comment les
\href{https://fr.wikipedia.org/wiki/Uniform_Resource_Locator}{liens
  URL} sont représentés dans ce document. Les \hyperlink{linkin}{liens internes} \hypertarget{retour}{sont}\label{retour}
représentés de la même manière que ça\footnote{Ceci est une note de
  bas de page.}. Oui les notes de bas de page\footnote{Très fréquente
  et très utile dans ce livre.} sont des liens internes au
document. Il y aura également d'autres liens cliquables comme par
exemple les tables des matières\footnote{Oui il y en a plusieurs.}.

En effet, afin de favoriser la navigation dans le document j'ai mis à la
fois une table des matières générale en début d'ouvrage puis au début
et à la fin de chaque chapitre vous trouverez un sommaire de chapitre
qui vous permettra de sauter ou de revenir sur une section de
chapitre que vous souhaitez cibler en particulier. Ce document est
très similaire à un site web de par sa structure de réseau (de
nombreux éléments divers sont liés les uns aux autres). Il faut savoir
que le langage \TeX{}\CW{https://fr.wikipedia.org/wiki/TeX} est un
langage de marquage de texte ou \underline{langage balisé} au même
titre que le langage HTML\CW{https://fr.wikipedia.org/wiki/Langage_de_balisage\#Langage_HTML} qui sert à écrire toutes les pages Web
que vous consultez sur Internet. Bien que la connexion ne soit pas
directe, il faut savoir que le langage \TeX{} est antérieur au langage
HTML et qu'il était initialement utilisé majoritairement par les
milieux scientifiques universitaires\dots les mêmes milieux qui ont
fait naître Internet et le langage HTML. En ce sens Donald
Knuth\footnote{L'\href{https://fr.wikipedia.org/wiki/Donald_Knuth}{inventeur} du langage \TeX{}.} était un précurseur de
Tim Berners Lee\footnote{L'\href{https://fr.wikipedia.org/wiki/Tim_Berners-Lee}{inventeur} du Web.}.

Ne serait-il pas plus simple de décrire ces liens en parlant de leurs
couleurs et de leur apparence générale ? Probablement, mais il se
trouve que la magie de
\href{https://en.wikipedia.org/wiki/XeTeX}{\XeLaTeX} ou de \LaTeX{} ou
des descendants de \TeX{} en général consiste à permettre au concepteur
de séparer sens et forme.

Cela signifie que j'ai précisé à l'ordinateur une commande spéciale
pour lui indiquer que je veux que tel ou tel mot ou groupe de mots
soit un lien URL par exemple. Ce qui me permet de cibler des
catégories de mots récurrents (au sens logique) comme par exemple tous
les liens internes, tous les liens externes\dots 

Ensuite au début du document, dans la rubrique des paramétrages, je
peux choisir à loisir les différentes formes que prendront ces
commandes. Par conséquent, toute l'apparence du document peut être
modifiée en << raffale >> si j'ose dire, en une seule fois. Et au
moment où j'écris ces lignes je n'ai pas encore arrêté mon choix quant
à la forme que je souhaite qu'elle prenne\footnote{Je parle de la
  forme de l'apparence du document donc c'est bien un féminin
  singulier.}. Il vient que la preuve par l'exemple m'a semblé être la
plus façon la plus pertinente pour faire ma démonstration et vous
expliquer le fonctionnement des liens dans ce livre.

\newpage

\section{De quel côté de la Manche ? de l'Atlantique ?}\label{sec:side}

Et j'aurais pu poursuivre avec l'océan Indien pour rester sur les
frontières maritimes. Oui dans cette section nous allons parler du
codage pour faire la distinction entre les mots en \exEN{English} et
ceux en \exFR{Français}. En fait, cette distinction de couleur sera
faite lorsqu'il s'agira de traduire\footnote{Néanmoins, l'auteur de ce
  document s'est parfois laissé aller à abuser un petit peu de cette
  récréation artistique.}.

Par exemple :
\begin{itemize}
\item \exEN{English: This is an English sentence\footnote{\exEN{So British!}}.}
\item \exFR{Français : Ceci est une phrase française\footnote{\exFR{Oui elle
      a bien ses papiers, par contre j'ai pas vérifié pour ses grands-parents.}}}.  
\end{itemize}

Je pense que vous avez saisi le truc. En guise d'illustration
complémentaire voici une petite définition utile\footnote{Dont j'avais
annoncé la définition prochaine ici page~\pageref{sec:link} (encore un
lien interne, j'espère que vous suivez).}, URL : \exEN{Uniform
  Resource Locator} qui se traduit littéralement par
\exFR{Localisateur Uniforme de Ressource} tout de suite ça a plus
d'allure\footnote{Ben oui, ça fait LUR\dots toute ressemblance avec de
  l'humour serait fortuite.}.

La couleur du cadre qui entoure les éléments de la table des matières
ou encore celle des numéros de notes de bas de page est celle réservée
aux \hypertarget{linkin}{liens internes}\footnote{D'ailleurs si vous
  avez atteri sur cette page après avoir cliqué sur le
  \hyperlink{retour}{lien interne} page~\pageref{retour}, vous devriez
savoir comment faire pour y revenir.}. 

\newpage

\section{Et les sons dans tout ça ?}\label{sec:phonetics}

Le but de ce livre étant de traiter de la phonétique anglaise, les
sons seront omniprésent. D'ailleurs même si je vous ai expliqué page~\pageref{sec:side}

\newpage
\minitoc

