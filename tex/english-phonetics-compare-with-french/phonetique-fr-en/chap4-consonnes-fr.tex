\chapter{Consonnes}\label{chap:cons}

Les consnones sont des bruits de frottement ou d'explosion produits
par le souffle qui, portant ou non les vibrations des cordes vocales,
rencontre dans la bouche divers obstacles résultant de la fermeture ou
du resserrement des organes de la parole.
\newpage
\minitoc
\newpage

\begin{itemize}
\item D'après le degré d'ouverture ou de fermeture des organes, on
  distingue :
  \begin{itemize}
  \item les consonnes occlusives (ou explosives)
  \item les consonnes constrictives (ou fricatives)
  \item la consonne latérale
  \item la consonne vibrante
  \end{itemize}
\item D'après l'endroit où les organes buccaux se touchent, on
  distingue :
  \begin{itemize}
  \item les consonnes bilabiales (lèvres) et labio-dentales (lèvres et
    dents)
  \item les consonnes dentales (langue et incisives)
  \item les consonnes alvéolaires (langue et alvéoles)
  \item les consonnes palatales (langue et partie dure du palais)
  \item les consonnes vélaires (langue et voile du palais)
  \end{itemize}
\item Les consonnes diffèrent par la présence ou l'absence de
  vibrations des cordes vocales :
  \begin{itemize}
  \item elles sont sonores quand le souffle qui les produit provoque
    des vibrations des cordes vocales
  \item elles sont sourdes quand le souffle qui les produit ne
    provoque pas de vibrations des cordes vocales 
  \end{itemize}
\item D'après la voie d'échappement du souffle par la bouche ou par le
  nez, on distingue :
  \begin{itemize}
  \item les consonnes orales
  \item les consonnes nasales
  \end{itemize}
\item Il y a trois semi-consonnes  
\end{itemize}

\section{ Par degré d'ouverture ou de fermeture des
  organes}\label{sec:ouvr}
\subsection{Les occlusives du français}\label{subsec:occ}

Le français contient les occlusives suivantes\footnote{Voir
  \href{https://fr.wikipedia.org/wiki/Consonne_occlusive}{Wikipédia} pour plus de détails sur cette classification.} :
\begin{itemize}
\item orales sourdes (non voisées) :
  \begin{itemize}
  \item \son{p} comme dans les mots
    \exFR{papa} (\href{http://www.wordreference.com/fren/papa}{\exPH{papa}}),
  \item \son{t} comme dans les mots
    \exFR{tonton}
    (\href{http://www.wordreference.com/fren/tonton}{\exPH{tɔ̃ tɔ̃ }}),
  \item \son{k} comme dans les mots
    \exFR{kermesse} (\href{http://www.wordreference.com/fren/kermesse}{\exPH{kɛʀmɛs}}), 
  \end{itemize}
\item orales sonores (voisées) :
  \begin{itemize}
  \item \son{b} comme dans les mots
    \exFR{ballon}
    (\href{http://www.wordreference.com/fren/ballon}{\exPH{balɔ̃ }}),
  \item \son{d} comme dans les mots 
    \exFR{décrire} (\href{http://www.wordreference.com/fren/d\%C3\%A9crire}{\exPH{dekʀiʀ}}), 
  \end{itemize}
\item nasales sonores (voiées) :
  \begin{itemize}
  \item \son{m} comme dans les mots :
    \exFR{madame}
    (\href{http://www.wordreference.com/fren/madame}{\exPH{medam}}),
  \item \son{n} comme dans les mots : 
    \exFR{nana}
    (\href{http://www.wordreference.com/fren/nana}{\exPH{nana}}), 
  \item \son{ɲ} comme dans les mots :
    \exFR{régner}
    (\href{http://www.wordreference.com/fren/r\%C3\%A9gner}{\exPH{ʀeɲe}})
  \end{itemize}
\end{itemize}

\newpage
\minitoc
