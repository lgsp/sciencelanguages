\part{Vowel sounds}\label{chap:vow}


J'attire votre attention sur 2 points importants afin d'éviter toute
confusion :
\begin{enumerate}
\item D'une part, anglais il n'y a que 5 voyelles écrites : \exEN{a}, \exEN{e},
  \exEN{i}, \exEN{o} et \exEN{u}, et non pas 6 comme en \exFR{français},
  parce que le \exEN{y} n'est pas  considéré comme une voyelle écrite
  par les anglophones.
\item D'autre part, en phonétique on s'intéresse aux sons et non pas aux
  lettres, par conséquent lorsqu'on parlera de voyelle cela
  sous-entendra toujours\footnote{Sauf mention explicite du
    contraire.} que l'on parle de \underline{voyelle sonore}. En
  effet, la phonétique étant la science de la production sonore du
  langage cela ne ferait pas sens de s'intéresser aux lettres qui sont
  indépendantes de la physique du son et de la biologie de notre anatomie.
\end{enumerate}
Selon \href{https://fr.wikipedia.org/wiki/Phon\%C3\%A9tique}{Wikipédia} :
\begin{quote}
  La phonétique (du grec « phônê » qui signifie la « voix », le « son ») est une branche de la linguistique qui étudie les sons utilisés dans la communication parlée. À la différence de la phonologie, qui étudie comment sont agencés les phonèmes d'une langue pour former des mots, la phonétique concerne les sons eux-mêmes (les unités phonétiques, les « phones »), leur production, leur variation plutôt que leur contexte. La sémantique ne fait donc pas partie de ce niveau d'analyse linguistique.
\end{quote}

Un \textcolor{teal}{son} de \textcolor{teal}{voyelle}\CW{https://en.wikipedia.org/wiki/English_phonology\#Vowels}
est fait en façonnant l'air comme il quitte la
bouche. Nous utilisons les articulateurs pour façonner l'air --- les
lèvres, la mâchoire, la langue. L'\exEN{anglais} britannique utilise
12 positions de la bouche\footnote{Vous pouvez les voir et les
  entendre en consultant ce \href{https://pronunciationstudio.com/vowel02/}{site}.}.  


\chapter{Front Vowels (langue vers l'avant)}\label{chap:frontvow}

\speech{4}{voyelles antérieures\CW{https://fr.wikipedia.org/wiki/Voyelle_ant\%C3\%A9rieure}}

\newpage
\minitoc
\newpage


% Front Vowel chart
% By Kwamikagami - modified my file IPA vowel chart 2005.png, CC BY-SA 3.0, https://en.wikipedia.org/w/index.php?curid=46937434

\section{\son{iː}}\label{sec:ilong}

Ce \textcolor{teal}{son} a pour nom technique\dyse{clear-front-unrounded-vowel} :

\begin{itemize}
\item \exEN{Close Front Unrounded Vowel\CW{https://en.wikipedia.org/wiki/Close_front_unrounded_vowel}.}
\item \exFR{Voyelle fermée antérieure non arrondie\CW{https://fr.wikipedia.org/wiki/Voyelle_ferm\%C3\%A9e_ant\%C3\%A9rieure_non_arrondie}.}
\end{itemize}

\indicsound

\properukus{https://youtu.be/EuZa9-QbhG8}{https://youtu.be/PIu5WDIco0I}

\begin{enumerate}
\item \exEN{\href{http://www.wordreference.com/enfr/need}{need}} qui s'écrit en phonétique \href{https://en.oxforddictionaries.com/definition/need}{\exPH{niːd}}

  \begin{itemize}
  \item\exEN{I \href{https://youtu.be/p0quLJutRC8}{need} to work every day if I want to \href{https://www.youtube.com/watch?v=xqMozc4K4pg&list=PLE_vQWWxgaiHUFB8zsbcEYqJL0GHGdMLi}{improve} my level.}
  \item\exFR{Je dois travailler tous les jours si je veux
      améliorer mon niveau.}
  \end{itemize}

  \youglish{need}


\item \exEN{\href{http://www.wordreference.com/enfr/tea}{tea}} qui s'écrit en phonétique \href{https://en.oxforddictionaries.com/definition/tea}{\exPH{tiː}}

  \begin{itemize}
  \item\exEN{\href{https://youtu.be/S8SdWEQg6cE}{Every morning} we are
      used to drinking \href{https://youtu.be/Euh8dY4EU9o}{tea}.}
  \item\exFR{Tous les matins on a l'habitude de boire du thé.}
  \end{itemize}

  \youglish{tea}

\item \exEN{\href{http://www.wordreference.com/enfr/believe}{believe}}
  qui s'écrit phonétiquement
  \href{https://en.oxforddictionaries.com/definition/believe}{\exPH{bɪˈliːv}}
  
  \begin{itemize}
  \item\exEN{\href{https://youtu.be/GIQn8pab8Vc}{I believe I can fly.}}
  \item\exFR{Je crois que je peux voler.}
  \end{itemize}

  \youglish{believe}
  
\item \exEN{\href{http://www.wordreference.com/enfr/see}{see}} qui s'écrit
  phonétiquement
  \href{https://en.oxforddictionaries.com/definition/see}{\exPH{siː}}
  
  \begin{itemize}
  \item\exEN{What You \href{https://youtu.be/Dpf2yHjBVYM}{See} Is What You Get (\href{https://fr.wikipedia.org/wiki/What\_you\_see\_is\_what\_you\_get}{WYSIWYG})}
  \item\exFR{Ce que vous voyez est ce que vous obtenez.}
  \end{itemize}

  \youglish{see}
  
\end{enumerate}
\newpage

\section{\son{ɪ}}\label{sec:soni}

Ce \textcolor{teal}{son} a pour nom technique\dyse{near-close-near-front-unrounded-vowel} :

\begin{itemize}
\item \exEN{Near-Close Near-Front Unrounded Vowel\CW{https://en.wikipedia.org/wiki/Near-close_near-front_unrounded_vowel}.}
\item \exFR{Voyelle pré-fermée antérieure non arrondie\CW{https://fr.wikipedia.org/wiki/Voyelle_pr\%C3\%A9-ferm\%C3\%A9e_ant\%C3\%A9rieure_non_arrondie}.}
\end{itemize}

\indicsound

\properukus{https://youtu.be/7PpuPMrISVc}{https://youtu.be/Ok_HG-0lNCA}

\begin{enumerate}
\item \exEN{\href{http://www.wordreference.com/enfr/england}{England}} qui s'écrit en phonétique \href{https://en.oxforddictionaries.com/definition/england}{\exPH{ˈɪŋɡlənd}}

  \begin{itemize}
  \item\exEN{Last \href{https://youtu.be/j9xSxJRsmx0}{summer} I went to \href{https://youtu.be/QUPBesOdax8}{England}.}
  \item\exFR{L'été dernier je suis allé en Angleterre.}
  \end{itemize}

  \youglish{England}
  
\item \exEN{\href{http://www.wordreference.com/enfr/thin}{thin}} qui s'écrit en phonétique \href{https://en.oxforddictionaries.com/definition/thin}{\exPH{θɪn}}

  \begin{itemize}
  \item\exEN{Usually \href{https://youtu.be/IpRSyVcHu-k}{top models} are \href{https://youtu.be/LekA62H17bo}{thin}.}
  \item\exFR{Habituellement les mannequins sont minces.}
  \end{itemize}

  \youglish{thin}
  
\item \exEN{\href{http://www.wordreference.com/enfr/big}{big}} qui s'écrit phonétiquement \href{https://en.oxforddictionaries.com/definition/big}{\exPH{bɪɡ}}

  \begin{itemize}
  \item\exEN{\href{https://youtu.be/do7w0j4ybeY}{New York} has a nickname: the \href{https://youtu.be/Jha4OkG-ixw}{Big} Apple.}
  \item\exFR{New York a un surnom : la grosse pomme.}
  \end{itemize}

  \youglish{big}
  
\item \exEN{\href{http://www.wordreference.com/enfr/which}{which}} qui s'écrit phonétiquement \href{https://en.oxforddictionaries.com/definition/which}{\exPH{wɪtʃ}}

  \begin{itemize}
  \item\exEN{\href{https://youtu.be/DN74ZuSrtxY}{Pick up} a glass from the table. \href{https://youtu.be/5fR\_\_LXDkRg}{Which} one?}
  \item\exFR{Choisis un verre sur la table. Lequel ?}
  \end{itemize}

  \youglish{which}
  
\end{enumerate}
\newpage

\section{\son{ɛ}  parfois noté  \son[~]{\href{https://dictionary.cambridge.org/dictionary/english/bed}{e}}
  }\label{sec:sone}

Il est important de préciser que le \son[~]{e} n'existe pas en anglais
pour la bonne et simple raison que ce symbole phonétique représente le
<<~\exFR{é}~>> comme dans le mot français \exFR{beauté}. Je n'ai pas réussi à
trouver d'explication suffisamment détaillée si ce n'est une <<~raison
historique~>> évoquée sur une page de
\href{http://teflpedia.com/IPA_phonetic_symbol_\%E3\%80\%9A\%C9\%9B\%E3\%80\%9B}{teflpedia}\footnote{Teflpedia
est un site inspiré de Wikipédia consacré au TEFL: Teaching English as
a Foreign Language (Enseignement de l'Anglais comme Langue \'Etrangère
équivalent du FLE : Français Langue \'Etrangère).}.

\begin{center}
  \begin{figure}[h]
    \centering
    \includegraphics[scale=.75]{../img/teflopedia-tab-1}
    \caption[Quel symbole phonétique pour le son "è"]{Démocratiquement
      le \son[~]{ɛ} est préférable au \son[~]{e}}
    \label{fig:teflpedia-1}
  \end{figure}
\end{center}

Ce \textcolor{teal}{son} a pour nom technique\dyse{open-mid-front-unrounded-vowel} :

\begin{itemize}
\item \exEN{Open-Mid Front Unrounded Vowel\CW{https://en.wikipedia.org/wiki/Open-mid_front_unrounded_vowel}.}
\item \exFR{Voyelle mi-ouverte antérieure non arrondie\CW{https://fr.wikipedia.org/wiki/Voyelle_mi-ouverte_ant\%C3\%A9rieure_non_arrondie}.}
\end{itemize}

\indicsound

\properukus{https://youtu.be/JhBH_rtOXGA}{https://youtu.be/xKxV8XfigaE}

\begin{enumerate}
\item \exEN{\href{http://www.wordreference.com/enfr/bed}{bed}} qui s'écrit
  phonétiquement
  \href{https://en.oxforddictionaries.com/definition/bed}{\exPH{bɛd}}

  \begin{itemize}
  \item\exEN{\href{https://youtu.be/LnshfLOhr2Q}{It's time to go} to \href{https://youtu.be/urARKkLo6MY}{bed}.}
  \item\exFR{C'est l'heure d'aller se coucher.}
  \end{itemize}

  \youglish{bed}
  
\item \exEN{\href{http://www.wordreference.com/enfr/bread}{bread}} qui s'écrit phonétiquement \href{https://en.oxforddictionaries.com/definition/bread}{\exPH{brɛd}}
  
  \begin{itemize}
  \item\exEN{\href{https://youtu.be/yxgE3ifjZ94}{French people} are famous for their \href{https://youtu.be/Ynm9Wrznz4I}{bread}.}
  \item\exFR{Les Français sont célèbres pour leur pain.}
  \end{itemize}

  \youglish{bread}
  
\item \exEN{\href{http://www.wordreference.com/enfr/said}{said}} qui s'écrit
  phonétiquement
  \href{https://en.oxforddictionaries.com/definition/said}{\exPH{sɛd}}

  \begin{itemize}
  \item\exEN{\href{https://www.azlyrics.com/lyrics/beatles/yesterday.html}{Yesterday} you said that \href{https://youtu.be/9IDogHTQgM4}{same thing}.}
  \item\exFR{Hier tu as dit cette même chose.}
  \end{itemize}

  \youglish{said}
  
\item \exEN{\href{http://www.wordreference.com/enfr/friend}{friend}} qui
  s'écrit phonétiquement
  \href{https://en.oxforddictionaries.com/definition/friend}{\exPH{frɛnd}}

  \begin{itemize}
  \item\exEN{\href{https://youtu.be/q-9kPks0IfE}{I'll be there for you} my \href{https://youtu.be/CY8E6N5Nzec}{friend}.}
  \item\exFR{Je serais là pour toi mon ami(e).}
  \end{itemize}

  \youglish{friend}

% % \item Mot français qui utilise le même son
% % \label{chap:motfr}
% % \exFR{\href{http://www.wordreference.com/fren/m\%25C3\%25A8re}{mère}} qui s'écrit phonétiquement \href{http://www.larousse.fr/dictionnaires/francais-\exEN{anglais}/m\%25c3\%25a8re/50499}{\exPH{mεr}}
  
\end{enumerate}
\newpage

\section{\son{æ} noté aussi parfois
  \son[~]{\href{https://en.oxforddictionaries.com/definition/cat}{a}}
  }\label{sec:sonae}

Ce \textcolor{teal}{son} a pour nom technique\dyse{near-open-front-unrounded-vowel} :

\begin{itemize}
\item \exEN{Near-Open Front Unrounded Vowel\CW{https://en.wikipedia.org/wiki/Near-open_front_unrounded_vowel}.}
\item \exFR{Voyelle ouverte antérieure non arrondie\CW{https://fr.wikipedia.org/wiki/Voyelle_pr\%C3\%A9-ouverte_ant\%C3\%A9rieure_non_arrondie}.}
\end{itemize}

\indicsound

\properukus{https://youtu.be/NavmTDkd8Z8}{https://youtu.be/mynucZiy-Ug}

\begin{enumerate}
\item \exEN{\href{http://www.wordreference.com/enfr/bat}{bat}} qui s'écrit
  phonétiquement
  \href{https://dictionary.cambridge.org/dictionary/english/bat}{\exPH{bæt}}
  
  \begin{itemize}
  \item\exEN{Have you ever noticed that \href{https://www.youtube.com/watch?v=O24Ui015YXM}{Batman} means the \href{https://youtu.be/24howVwYgHY}{man} who
      is a \href{https://youtu.be/eozL5n2Plmc}{bat}?}
  \item\exFR{As-tu déjà remarqué que Batman signifie l'homme qui
      est une chauve-souris ?}
  \end{itemize}

  \youglish{bat}
  
\item \exEN{\href{http://www.wordreference.com/enfr/cat}{cat}} qui s'écrit
  phonétiquement
  \href{https://dictionary.cambridge.org/dictionary/english/cat}{\exPH{kæt}}

  \begin{itemize}
  \item\exEN{What \href{https://youtu.be/7FjChUY0zgQ}{about} Catwoman? Is she a \href{https://youtu.be/eNQazP-wdj4}{cat}?}
  \item\exFR{Qu'en est-il de Catwoman ? Est-elle une chatte ?}
  \end{itemize}

  \youglish{cat}
  
\item \exEN{\href{http://www.wordreference.com/enfr/that}{that}} qui s'écrit phonétiquement \href{https://dictionary.cambridge.org/dictionary/english/that}{\exPH{ðæt}} 
\begin{itemize}
\item\exEN{\href{https://youtu.be/HAlz5TiKOCM}{That} house is \href{https://youtu.be/qYvXk_bqlBk}{really} big.}
\item\exFR{Cette maison est vraiment grande.}
\end{itemize}

\youglish{that}

\item \exEN{\href{http://www.wordreference.com/enfr/hand}{hand}} qui s'écrit
  phonétiquement
  \href{https://dictionary.cambridge.org/dictionary/english/hand}{\exPH{hænd}}

  \begin{itemize}
  \item\exEN{\href{https://youtu.be/-ccNkksrfls}{Maradona} scored with his \href{https://youtu.be/KDKBY9FqwQg}{hand} during a famous
      match between Argentina and England.}
  \item\exFR{Maradona avait marqué avec sa main durant un célèbre
      match entre l'Argentine et l'Angleterre.}
  \end{itemize}

  \youglish{hand}
  
\end{enumerate}
\newpage
\minitoc
\newpage

\chapter{Centre Vowels (langue relativement plate)}\label{chap:centvow}

\speech{4}{voyelles centrales\CW{https://fr.wikipedia.org/wiki/Voyelle_centrale}}

\newpage
\minitoc
\newpage

\section{\son{ə} qui se note aussi parfois \son[~]{ɜ}  }\label{sec:sonenv}

Ce \textcolor{teal}{son} a pour nom technique\dyse{mid-central-vowel} :

\begin{itemize}
\item \exEN{Mid-Central Vowel\CW{https://en.wikipedia.org/wiki/Mid_central_vowel}.}
\item \exFR{Voyelle moyenne centrale\CW{https://fr.wikipedia.org/wiki/Voyelle_moyenne_centrale}.}
\end{itemize}

\indicsound

\properukus{https://youtu.be/RVvn6204I_Y}{https://youtu.be/m1mDSUSwNls}

\begin{enumerate}
\item \exEN{\href{http://www.wordreference.com/enfr/ago}{ago}} qui s'écrit
  phonétiquement
  \href{https://en.oxforddictionaries.com/definition/ago}{\exPH{əˈɡəʊ}}

  \begin{itemize}
  \item\exEN{I started to \href{https://youtu.be/G5dViczwTXo}{learn English} when I was in Middle School
      twenty-five years \href{https://youtu.be/RO4fWbM3WA8}{ago}!}
  \item\exFR{J'ai commencé à apprendre l'\exEN{anglais} quand j'étais au
      Collège il y a vingt-cinq ans !}
  \end{itemize}
  
\item \exEN{\href{http://www.wordreference.com/enfr/today}{today}} qui
  s'écrit phonétiquement
  \href{https://en.oxforddictionaries.com/definition/today}{\exPH{təˈdeɪ}}

  \begin{itemize}
  \item\exEN{\href{https://youtu.be/yCSLK0WCUd8}{Today} is \href{https://youtu.be/Sox7KmmAEZI}{Wednesday}.}
  \item\exFR{Aujourd'hui c'est mercredi.}
  \end{itemize}
  
\item \exEN{\href{http://www.wordreference.com/enfr/rhythm}{rhythm}}
  (attention il y a bien deux fois la lettre 'h') qui s'écrit phonétiquement
\href{https://en.oxforddictionaries.com/definition/rhythm}{\exPH{ˈrɪð(ə)m}}

\begin{itemize}
\item\exEN{Did you know that all \href{https://www.youtube.com/watch?v=W8B2E11TFe0&list=PLwwOk5fvpuuLgntaf5Z9QKh5uV2Vl2Xme}{languages} have their own \href{https://youtu.be/XQJVoS3SlX0}{rhythm}?}
\item\exFR{Saviez-vous que chaque langue a \textcolor{teal}{son} propre rythme ?}
\end{itemize}

\item \exEN{\href{http://www.wordreference.com/enfr/supply}{supply}} qui
  s'écrit phonétiquement
  \href{https://en.oxforddictionaries.com/definition/supply}{\exPH{səˈplʌɪ}}

  \begin{itemize}
  \item\exEN{Do not \href{https://youtu.be/Xh4ugYiXF-Q}{worry} I will always \href{https://youtu.be/qEd6QUbK2Mw}{supply} you with multimedia
      documents, audio links, videos, texts, and so on.}
  \item\exFR{Ne vous inquiétez pas, je vous fournirai toujours des
      documents multimédias, des liens audios, des vidéos, des textes\dots}
  \end{itemize}
\end{enumerate}
\newpage

\section{\son{ɜː} qui se  note aussi parfois \son[~]{əː} }\label{sec:sonenvlong}

Ce \textcolor{teal}{son} a pour nom technique\dyse{open-mid-central-unrounded-vowel} :

\begin{itemize}
\item \exEN{Open-Mid Central Unrounded Vowel\CW{https://en.wikipedia.org/wiki/Open-mid_central_unrounded_vowel}.}
\item \exFR{Voyelle mi-ouverte centrale non arrondie\CW{https://fr.wikipedia.org/wiki/Voyelle_mi-ouverte_centrale_non_arrondie}.}
\end{itemize}

\indicsound

\properukus{https://youtu.be/dweBtpz3gco}{https://youtu.be/Ehn6XixUBKs}

\begin{enumerate}
\item \exEN{\href{http://www.wordreference.com/enfr/bird}{bird}} qui s'écrit
  phonétiquement
  \href{https://dictionary.cambridge.org/fr/dictionnaire/anglais/bird}{\exPH{bɜːd}}
  
  \begin{itemize}
  \item\exEN{\href{https://genius.com/The-beatles-free-as-a-bird-lyrics}{Free as a bird.}}
  \item\exFR{Libre comme l'air (littéralement : libre tel un
      oiseau)}
  \end{itemize}
  
\item \exEN{\href{http://www.wordreference.com/enfr/turn}{turn}} qui s'écrit
  phonétiquement
  \href{https://dictionary.cambridge.org/fr/dictionnaire/anglais/turn}{\exPH{tɜːn}}
  
  \begin{itemize}
  \item\exEN{\href{https://youtu.be/WLTI2rWAlV4}{Turn} off your TV; in
      fact, you should \href{https://youtu.be/pixcJiRq6Tk}{sell it}.}
  \item\exFR{Éteins ta télé, en fait, tu devrais la vendre.}
  \end{itemize}
  
\item \exEN{\href{http://www.wordreference.com/enfr/worse}{worse}} qui
  s'écrit phonétiquement
  \href{https://dictionary.cambridge.org/fr/dictionnaire/anglais/worse}{\exPH{wɜːs}}

  \begin{itemize}
  \item\exEN{I don't know if watching silly cat videos on YouTube
      is \href{https://youtu.be/JHWhzS0zdOc}{worse} than watching TV, but you won't \href{https://youtu.be/-wcn2EbOIbQ}{improve} your
      intellectual level by doing so.}
  \item\exFR{Je ne sais pas si regarder des vidéos débiles de chat
      sur YouTube est pire que de regarder la télé, mais tu
      n'augmenteras pas ton niveau intellectuel en le faisant.}
  \end{itemize}
  
\item \exEN{\href{http://www.wordreference.com/enfr/learn}{learn}} qui
  s'écrit phonétiquement
  \href{https://dictionary.cambridge.org/fr/dictionnaire/anglais/learn}{\exPH{lɜːn}}
  
  \begin{itemize}
  \item\exEN{If you want to \href{https://youtu.be/1xXs7MAsB0w}{learn} \href{https://youtu.be/YEaSxhcns7Y}{English}, you need to \href{https://youtu.be/wmCAKUFKZ7Y}{practice}
      the sounds.}
  \item\exFR{Si tu veux apprendre l'\exEN{anglais}, il faut que tu
      pratiques les \textcolor{teal}{sons}.}
  \end{itemize}
  
\end{enumerate}
\newpage

\section{\son{ʌ}}\label{sec:sonup}

Ce \textcolor{teal}{son} a pour nom technique\dyse{open-mid-back-unrounded-vowel} :

\begin{itemize}
\item \exEN{Open-mid Back Unrounded Vowel\CW{https://en.wikipedia.org/wiki/Open-mid_back_unrounded_vowel}.}
\item \exFR{Voyelle mi-ouverte postérieure non arrondie\CW{https://fr.wikipedia.org/wiki/Voyelle_mi-ouverte_post\%C3\%A9rieure_non_arrondie}.}
\end{itemize}

\indicsound

\properukus{https://youtu.be/zUpF0pYoTZ8}{https://youtu.be/_63fTgbG-yQ}


\begin{enumerate}
\item \exEN{\href{http://www.wordreference.com/enfr/cup}{cup}} qui s'écrit
  phonétiquement
  \href{https://en.oxforddictionaries.com/definition/cup}{\exPH{kʌp}}

  \begin{itemize}
  \item\exEN{Do \href{https://youtu.be/R7iN71uJcG0}{you want} a \href{https://youtu.be/pjcOzqxu4JQ}{cup} of tea?}
  \item\exFR{Voulez-vous une tasse de thé ?}
  \end{itemize}
  
\item \exEN{\href{http://www.wordreference.com/enfr/something}{something}}
  qui s'écrit phonétiquement
  \href{https://en.oxforddictionaries.com/definition/something}{\exPH{ˈsʌmθɪŋ}}
  
  \begin{itemize}
  \item\exEN{She does \href{https://youtu.be/UelDrZ1aFeY}{something} special with \href{https://youtu.be/b4xcpMCPhfE}{her voice} that I can't
      \href{https://genius.com/The-beatles-something-lyrics}{describe}, but I like it.}
  \item\exFR{Elle fait quelque chose de spécial avec sa voix que
      je ne peux pas décrire, mais j'aime ça.}
  \end{itemize}
  
\item \exEN{\href{http://www.wordreference.com/enfr/fun}{fun}} qui s'écrit
  phonétiquement
  \href{https://en.oxforddictionaries.com/definition/fun}{\exPH{fʌn}}

  \begin{itemize}
  \item\exEN{Some studies have shown that having \href{https://youtu.be/KXJNoC6CuYE}{fun} is the \href{https://youtu.be/_f-qkGJBPts}{best}
      way to \href{https://youtu.be/p60rN9JEapg}{learn}.}
  \item\exFR{Des études ont montré que s'amuser est le meilleur
      moyen pour apprendre.}
  \end{itemize}
  
\item \exEN{\href{http://www.wordreference.com/enfr/luck}{luck}} qui s'écrit
  phonétiquement
  \href{https://en.oxforddictionaries.com/definition/luck}{\exPH{lʌk}}
  
  \begin{itemize}
  \item\exEN{They wish you good \href{https://youtu.be/LQCY2zL0Jr8}{luck} in your \href{https://youtu.be/o61dD6hwrdM}{studies}.}
  \item\exFR{Ils vous souhaietent bonne chance pour votre
      apprentissage.}
  \end{itemize}
  
\end{enumerate}
\newpage

\section{\son{ɑː} qui devrait plutôt être noté\son[~]{aː}}\label{sec:sonalong}

Ce \textcolor{teal}{son} a pour nom technique\dyse{open-back-unrounded-vowel} :

\begin{itemize}
\item \exEN{Open Back Unrounded Vowel\CW{https://en.wikipedia.org/wiki/Open_back_unrounded_vowel}.}
\item \exFR{Voyelle arrière non arrondie\CW{https://fr.wikipedia.org/wiki/Voyelle_ouverte_post\%C3\%A9rieure_non_arrondie}.}
\end{itemize}

\indicsound

\properukus{https://youtu.be/1F47WdIjn5U}{https://youtu.be/R5CY1UniS68}


\begin{enumerate}
\item \exEN{\href{http://www.wordreference.com/enfr/father}{father}} qui
  s'écrit phonétiquement
  \href{https://en.oxforddictionaries.com/definition/father}{\exPH{ˈfɑːðə}}
  
  \begin{itemize}
  \item\exEN{My \href{https://youtu.be/MZDAUbeSwNY}{father} used to
      tell me that you never \href{https://youtu.be/gYSZjGeK5VE}{waste
        your time} when you are thinking.}
  \item\exFR{Mon père avait l'habitude de me dire qu'on ne perd
      jamais son temps à réfléchir.}
  \end{itemize}
  
\item \exEN{\href{http://www.wordreference.com/enfr/arm}{arm}} qui s'écrit
  phonétiquement
  \href{https://en.oxforddictionaries.com/definition/arm}{\exPH{ɑːm}}

  \begin{itemize}
  \item\exEN{We are \href{https://youtu.be/H75AFiLGd8Y}{lucky} because we have two \href{https://youtu.be/tlhQghmuMf8}{arms} and two legs;
      less lucky if one of them is harmed.}
  \item\exFR{Nous avons la chance d'avoir deux bras et deux
      jambes; désolé si l'un d'eux est blessé.}
  \end{itemize}
  
\item \exEN{\href{http://www.wordreference.com/enfr/dance}{dance}} qui
  s'écrit phonétiquement
  \href{https://en.oxforddictionaries.com/definition/dance}{\exPH{dɑːns}}
  
  \begin{itemize}
  \item\exEN{Would you like to \href{https://youtu.be/aagbeWUDe7w}{dance} with me \href{https://youtu.be/0zQHNygI_ko}{pretty lady}?}
  \item\exFR{Veux-tu danser avec moi jolie demoiselle ?}
  \end{itemize}
  
\item \exEN{\href{http://www.wordreference.com/enfr/half}{half}} qui s'écrit
  phonétiquement
  \href{https://en.oxforddictionaries.com/definition/half}{\exPH{hɑːf}}
  
  \begin{itemize}
  \item\exEN{\href{https://youtu.be/XWamnSNgiCM}{Half} time! This is the \href{https://youtu.be/2I2kKXWjwhM}{right time} to get some drinks!}
  \item\exFR{Mi-temps ! C'est le bon moment pour prendre à boire !}
  \end{itemize}
  
\end{enumerate}

\newpage
\minitoc
\newpage

\chapter{Back Vowels (langue vers l'arrière)}\label{chap:backvow}

\speech{4}{voyelles postérieures\CW{https://fr.wikipedia.org/wiki/Voyelle_post\%C3\%A9rieure}}

\newpage
\minitoc
\newpage

\section{ \son{uː} }\label{sec:ulong}

Ce \textcolor{teal}{son} a pour nom technique\dyse{close-back-rounded-vowel} :

\begin{itemize}
\item \exEN{Close Back Rounded Vowel\CW{https://en.wikipedia.org/wiki/Close_back_rounded_vowel}.}
\item \exFR{Voyelle fermée postérieure arrondie\CW{https://fr.wikipedia.org/wiki/Voyelle_ferm\%C3\%A9e_post\%C3\%A9rieure_arrondie}.}
\end{itemize}

\indicsound

\properukus{https://youtu.be/qPB0Ajjs7nE}{https://youtu.be/lkM6CKBM2ns}

\begin{enumerate}
\item \exEN{\href{http://www.wordreference.com/enfr/too}{too}} qui s'écrit
  phonétiquement
  \href{https://en.oxforddictionaries.com/definition/too}{\exPH{tuː}}
  
  \begin{itemize}
  \item\exEN{I like to speak \href{https://youtu.be/H3r9bOkYW9s}{English}, and you? Me \href{https://youtu.be/RaveinO4\_vs}{too}.}
  \item\exFR{J'aime parler \exEN{Anglais}, et toi ? Moi aussi.}
  \end{itemize}
  
\item \exEN{\href{http://www.wordreference.com/enfr/few}{few}} qui s'écrit
  phonétiquement
  \href{https://en.oxforddictionaries.com/definition/few}{\exPH{fjuː}}
  
  \begin{itemize}
  \item\exEN{\href{https://youtu.be/r3TaGhdqEiA}{Few} people understand the key role of \href{https://www.youtube.com/watch?v=dtf8zGQj9GY&list=PLfLdA1jGDSu6exdSf9yQJWKgNqPviO4b4}{phonetics}.}
  \item\exFR{Peu de gens comprennent le rôle clé de la phonétique.}
  \end{itemize}
  
\item \exEN{\href{http://www.wordreference.com/enfr/rule}{rule}} qui s'écrit
  phonétiquement
  \href{https://en.oxforddictionaries.com/definition/rule}{\exPH{ruːl}}
  
  \begin{itemize}
  \item\exEN{\href{https://youtu.be/rStL7niR7gs}{Do you want} \href{https://amzn.to/2GneMiu}{to rule?}}
  \item\exFR{Voulez-vous diriger ?}
  \end{itemize}
  
\item \exEN{\href{http://www.wordreference.com/enfr/lose}{lose}} qui s'écrit
  phonétiquement
  \href{https://en.oxforddictionaries.com/definition/lose}{\exPH{luːz}}
  
  \begin{itemize}
  \item\exEN{You \href{https://youtu.be/UNcCTgA5lzo}{lose} \href{https://amzn.to/2GnOFbd}{the game} this time, do you want to try again?}
  \item\exFR{Vous avez perdu la partie cette fois, voulez-vous
      essayer à nouveau ?}
  \end{itemize}
  
\end{enumerate}

\newpage

\section{\son{ʊ} }\label{sec:omega}

Ce \textcolor{teal}{son} a pour nom technique\dyse{near-close-near-back-rounded-vowel} :

\begin{itemize}
\item \exEN{Near-Close Near-Back Rounded Vowel\CW{https://en.wikipedia.org/wiki/Near-close_near-back_rounded_vowel}.}
\item \exFR{Voyelle pré-fermée postérieure arrondie\CW{https://fr.wikipedia.org/wiki/Voyelle_pr\%C3\%A9-ferm\%C3\%A9e_post\%C3\%A9rieure_arrondie}.}
\end{itemize}

\indicsound

\properukus{https://youtu.be/5lOF-zRg8x0}{https://youtu.be/moLTR-dLQQY}

\begin{enumerate}
\item \exEN{\href{http://www.wordreference.com/enfr/good}{good}} qui
  s'écrit phonétiquement
  \href{https://en.oxforddictionaries.com/definition/good}{\exPH{ɡʊd}}
  
  \begin{itemize}
  \item\exEN{Your \href{https://youtu.be/GihybX7JyG4}{book} is \href{https://youtu.be/o3TQSaqHBtM}{good}.}
  \item\exFR{Votre le livre est bon.}
  \end{itemize}
  
\item \exEN{\href{http://www.wordreference.com/enfr/put}{put}} qui s'écrit
  phonétiquement
  \href{https://en.oxforddictionaries.com/definition/put}{\exPH{pʊt}}
  
  \begin{itemize}
  \item\exEN{\href{https://youtu.be/BSpoa7TsiD0}{Put} your energy in \href{https://youtu.be/bLMwe6kFFg0}{something you like}.}
  \item\exFR{Mettez votre énergie dans quelque chose que vous
      aimez.}
  \end{itemize}
  
\item \exEN{\href{http://www.wordreference.com/enfr/would}{would}} qui
  s'écrit phonétiquement
  \href{https://en.oxforddictionaries.com/definition/would}{\exPH{wʊd}}
  
  \begin{itemize}
  \item\exEN{\href{https://youtu.be/wRSNm3pr100}{Would} you like to \href{https://youtu.be/tiDvgH8yNhg}{drink} something?}
  \item\exFR{Voulez-vous boire quelque chose ?}
  \end{itemize}
  
\item \exEN{\href{http://www.wordreference.com/enfr/look}{look}} qui s'écrit
  phonétiquement
  \href{https://en.oxforddictionaries.com/definition/look}{\exPH{lʊk}}

  \begin{itemize}
  \item\exEN{\href{https://youtu.be/b4xcpMCPhfE}{Look} at \href{https://youtu.be/_SXm5nnzZJk}{this}!}
  \item\exFR{Regarde ça !}
  \end{itemize}
  
\end{enumerate}

\newpage

\section{ \son{ɔː} }\label{sec:oouvert}

Ce \textcolor{teal}{son} a pour nom technique\dyse{open-mid-back-rounded-vowel} :

\begin{itemize}
\item \exEN{Open-Mid Back Rounded Vowel\CW{https://en.wikipedia.org/wiki/Open-mid_back_rounded_vowel}.}
\item \exFR{Voyelle mi-ouverte postérieure arrondie\CW{https://fr.wikipedia.org/wiki/Voyelle_mi-ouverte_post\%C3\%A9rieure_arrondie}.}
\end{itemize}

\indicsound

\properukus{https://youtu.be/Bc1tCtP2ZSg}{https://youtu.be/pr_KAu-_Hmo}

\begin{enumerate}
\item \exEN{\href{http://www.wordreference.com/enfr/pork}{pork}} qui s'écrit
  phonétiquement
  \href{https://en.oxforddictionaries.com/definition/pork}{\exPH{pɔːk}}
  
  \begin{itemize}
  \item\exEN{Do you \href{https://youtu.be/ZJeI2VIEDY8}{eat} \href{https://youtu.be/WqTJbyfewzw}{pork}?}
  \item\exFR{Mangez-vous du porc ?}
  \end{itemize}

  \youglish{pork}
  
\item \exEN{\href{http://www.wordreference.com/enfr/law}{law}} qui s'écrit
  phonétiquement
  \href{https://en.oxforddictionaries.com/definition/law}{\exPH{lɔː}}

  \begin{itemize}
  \item\exEN{\href{https://youtu.be/us5CUAsH0U0}{Hackers like to say: code is law.}}
  \item\exFR{Les hackers aiment dire que le code est la loi.}
  \end{itemize}

  \youglish{law}
  
\item \exEN{\href{http://www.wordreference.com/enfr/taught}{taught}} qui
  s'écrit phonétiquement
  \href{https://en.oxforddictionaries.com/definition/taught}{\exPH{tɔːt}}
  
  \begin{itemize}
  \item\exEN{I \href{https://youtu.be/U2BG2\_K2fGk}{taught} you how to write \href{https://youtu.be/o8KppNXfx2k}{English phonetics} yesterday.}
  \item\exFR{Hier je t'ai enseigné comment écrire la phonétique
      \exEN{Anglaise}.}
  \end{itemize}

  \youglish{taught}
  
\item \exEN{\href{http://www.wordreference.com/enfr/thought}{thought}} qui
  s'écrit phonétiquement
  \href{https://en.oxforddictionaries.com/definition/thought}{\exPH{θɔːt}}
  
  \begin{itemize}
  \item\exEN{\href{https://youtu.be/XO273RIGifY}{Tell me} your \href{https://youtu.be/8kR-GDbYHhc}{thoughts}.}
  \item\exFR{Raconte-moi tes pensées.}
  \end{itemize}

  \youglish{thought}
  
\end{enumerate}

\newpage

\section{ \son{ɒ}}\label{sec:oa}

Ce \textcolor{teal}{son} a pour nom technique\dyse{open-back-rounded-vowel} :

\begin{itemize}
\item \exEN{Open Back Rounded Vowel\CW{https://en.wikipedia.org/wiki/Open_back_rounded_vowel}.}
\item \exFR{Voyelle ouverte postérieure arrondie\CW{https://fr.wikipedia.org/wiki/Voyelle_ouverte_post\%C3\%A9rieure_arrondie}.}
\end{itemize}

\indicsound

\begin{center}
  \uks{https://youtu.be/A3l-yWQfIW4}
\end{center}


\begin{enumerate}
\item \exEN{\href{http://www.wordreference.com/enfr/got}{got}} qui s'écrit
  phonétiquement
  \href{https://en.oxforddictionaries.com/definition/got}{\exPH{ɡɒt}}

  \begin{itemize}
  \item\exEN{I \href{https://youtu.be/Bo09BiPb24Y}{got} you. (slang: \href{https://youtu.be/EWRaAbVUkjA}{Gotcha})}
  \item\exFR{Je t'ai eu.} (argot : Gotcha)
  \end{itemize}

  \youglish{got}
  
\item \exEN{\href{http://www.wordreference.com/enfr/watch}{watch}} qui
  s'écrit phonétiquement
  \href{https://en.oxforddictionaries.com/definition/watch}{\exPH{wɒtʃ}}
  
  \begin{itemize}
  \item\exEN{\href{https://youtu.be/qOs8MagOfwg}{Watch} this video \href{https://youtu.be/cnIanivwpSU}{carefully}.}
  \item\exFR{Regardez attentivement cette vidéo.}
  \end{itemize}

  \youglish{watch}
  
\item \exEN{\href{http://www.wordreference.com/enfr/rob}{rob}} qui s'écrit
  phonétiquement
  \href{https://en.oxforddictionaries.com/definition/rob}{\exPH{rɒb}}

  \begin{itemize}
  \item\exEN{Are you planning to \href{https://youtu.be/X3uZ0Gf104A}{rob} a bank? I \href{https://youtu.be/CILQJnsD128}{discourage} you to do
      that.}
  \item\exFR{Êtes-vous en train d'envisager de cambrioler une
      banque~? Je vous déconseille de faire ça.}
  \end{itemize}

  \youglish{rob}
  
\item \exEN{\href{http://www.wordreference.com/enfr/top}{top}} qui s'écrit
  phonétiquement
  \href{https://en.oxforddictionaries.com/definition/top}{\exPH{tɒp}}

  \begin{itemize}
  \item\exEN{\href{https://youtu.be/gPaD513xWOY}{Top} videos are sometime very \href{https://youtu.be/M9i2HAE-ZSw}{boring}.}
  \item\exFR{Les vidéos de top sont parfois très ennuyeuses.}
  \end{itemize}

  \youglish{top}
  
\end{enumerate}

\newpage
\minitoc
\newpage

\chapter{Diphthong Vowels}\label{chap:diphtong}

\speech{8}{diphtongues\CW{https://fr.wikipedia.org/wiki/Diphtongue}}

\newpage
\minitoc
\newpage

\section{\son{eɪ} }\label{sec:ei}

\diph{e}{ɪ}{eɪ}{diphthong-1-7}{\son{e} n'existe pas à l'état isolé en
  anglais et pour le \son[~]{ɪ} voir page~\pageref{chap:soni}.}

\properukus{https://youtu.be/oTAzk9xm5i8}{https://youtu.be/0RXzfRcjk-s}

\begin{enumerate}
\item \exEN{\href{http://www.wordreference.com/enfr/snake}{snake}} qui
  s'écrit phonétiquement
  \href{https://en.oxforddictionaries.com/definition/snake}{\exPH{sneɪk}}

  \begin{itemize}
  \item\exEN{\href{https://youtu.be/MOltIVdyAHQ}{Snakes} regularly shed their \href{https://youtu.be/6kjaagQcYkc}{skin}.}
  \item\exFR{Les serpents perdent régulièrement leur peau.}
  \end{itemize}

  \youglish{snake}
  
\item \exEN{\href{http://www.wordreference.com/enfr/pay}{pay}} qui s'écrit
  phonétiquement
  \href{https://en.oxforddictionaries.com/definition/pay}{\exPH{peɪ}}

  \begin{itemize}
  \item\exEN{How much \href{https://youtu.be/wRSNm3pr100}{would} you be able to \href{https://youtu.be/mBuLm5XeF44}{pay} for additional
      content?}
  \item\exFR{Combien seriez-vous capable de payer pour du contenu
      supplémentaire ?}
  \end{itemize}

  \youglish{pay}
  
\item \exEN{\href{http://www.wordreference.com/enfr/mail}{mail}} qui s'écrit
  phonétiquement
  \href{https://en.oxforddictionaries.com/definition/mail}{\exPH{meɪl}}
  
  \begin{itemize}
  \item\exEN{The \href{https://youtu.be/CFNaCU3sXh8}{post office} redirected the \href{https://youtu.be/KX1CSSZa1v0}{mail} to my new address.}
  \item\exFR{Le bureau de poste a fait suivre le courrier à ma
      nouvelle adresse.}
  \end{itemize}

  \youglish{mail}
  
\item \exEN{\href{http://www.wordreference.com/enfr/great}{great}} qui
  s'écrit phonétiquement
  \href{https://en.oxforddictionaries.com/definition/great}{\exPH{ɡreɪt}}
  
  \begin{itemize}
  \item\exEN{Your \href{https://youtu.be/dBnpr3pkFlk}{content} is \href{https://youtu.be/e0qM84DWXzA}{great}!}
  \item\exFR{Ton contenu est génial !}
  \end{itemize}

  \youglish{great}
  
\end{enumerate}

\newpage

\section{ \son{ɔɪ} }\label{sec:oouverti}

\diph{ɔː}{ɪ}{ɔɪ}{diphthong-2-7}{\son{ɔː} a été étudié page~\pageref{chap:oouvert} et pour le \son[~]{ɪ} voir page~\pageref{chap:soni}.}

\properukus{https://youtu.be/M-8ZqxVJMf8}{https://youtu.be/ZfjPBN22mK8}

\begin{enumerate}
\item \exEN{\href{http://www.wordreference.com/enfr/toy}{toy}} qui s'écrit
  phonétiquement
  \href{https://en.oxforddictionaries.com/definition/toy}{\exPH{tɔɪ}}

  \begin{itemize}
  \item\exEN{The \href{https://youtu.be/UyzQMSlhQik}{little} boy was delighted with all his \href{https://youtu.be/1qbuZhVUj\_g}{toys}.}
  \item\exFR{Le petit garçon était enchanté par tous ses jouets.}
  \end{itemize}

  \youglish{toy}
  
\item \exEN{\href{http://www.wordreference.com/enfr/choice}{choice}} qui
  s'écrit phonétiquement
  \href{https://en.oxforddictionaries.com/definition/choice}{\exPH{tʃɔɪs}}

  \begin{itemize}
  \item\exEN{\href{https://youtu.be/2yKfkr2lqQM}{Looking} at my additional content is your \href{https://youtu.be/qBfeK\_IIHag}{choice}.}
  \item\exFR{Regarder mon contenu supplémentaire est votre choix.}
  \end{itemize}

  \youglish{choice}
  
\item \exEN{\href{http://www.wordreference.com/enfr/joy}{joy}} qui s'écrit
  phonétiquement
  \href{https://en.oxforddictionaries.com/definition/joy}{\exPH{dʒɔɪ}}

  \begin{itemize}
  \item\exEN{The music \href{https://youtu.be/jzUp8mdionM}{creates} a sensation of \href{https://youtu.be/-GjW1pSYgUk}{joy} and playfulness.}
  \item\exFR{La musique crée une sensation de joie et de gaieté.}
  \end{itemize}

  \youglish{joy}
  
\item \exEN{\href{http://www.wordreference.com/enfr/oyster}{oyster}} qui
  s'écrit phonétiquement
  \href{https://en.oxforddictionaries.com/definition/oyster}{\exPH{ˈɔɪstə}}

  \begin{itemize}
  \item\exEN{\href{https://youtu.be/arj7oStGLkU}{Inside} the \href{https://youtu.be/PVn6b9QQZeM}{oyster}, I found a pearl.}
  \item\exFR{À l'intérieur de l'huître, j'ai trouvé une perle.}
  \end{itemize}

  \youglish{oyster}
  
\end{enumerate}

\newpage

\section{ \son{aɪ} }\label{sec:ai}

\diph{aː}{ɪ}{aɪ}{diphthong-3-7}{\son{aː} a été étudié
  page~\pageref{chap:sonalong} et pour le \son[~]{ɪ} voir
  page~\pageref{chap:soni}.}

\properukus{https://youtu.be/ub9ONgsThKc}{https://youtu.be/8uD-GuuSgyk}

\begin{enumerate}
\item \exEN{\href{http://www.wordreference.com/enfr/my}{my}} qui s'écrit
  phonétiquement
  \href{https://dictionary.cambridge.org/dictionary/english/my}{\exPH{maɪ}}

  \begin{itemize}
  \item\exEN{\href{https://youtu.be/SMwEkjcEACM}{My} content is made to help you \href{https://www.youtube.com/watch?v=m\_uWS6K-VF8\&list=PL0J5xb8JH3VukoRHgk86Yr9BSVeBewCuZ}{progress in English}.}
  \item\exFR{Mon contenu est fait pour vous aider à progresser en
      \exEN{anglais}.}
  \end{itemize}

  \youglish{my}
  
\item \exEN{\href{http://www.wordreference.com/enfr/while}{while}} qui
  s'écrit phonétiquement
  \href{https://dictionary.cambridge.org/dictionary/english/while}{\exPH{waɪl}}

  \begin{itemize}
  \item\exEN{\href{https://youtu.be/O040xuq2FR0}{She} partied \href{https://youtu.be/8q182kWAhiM}{while} I worked.}
  \item\exFR{Elle faisait la fête alors que je travaillais.}
  \end{itemize}

  \youglish{while}
  
\item \exEN{\href{http://www.wordreference.com/enfr/might}{might}} qui
  s'écrit phonétiquement
  \href{https://dictionary.cambridge.org/dictionary/english/might}{\exPH{maɪt}}

  \begin{itemize}
  \item\exEN{\href{https://youtu.be/gGMSfiH850o}{Hurricanes} show us the \href{https://youtu.be/Nqlr35WnqTk}{might} of nature.}
  \item\exFR{Les ouragans nous démontrent la puissance de la
      nature.}
  \end{itemize}

  \youglish{might}
  
\item \exEN{\href{http://www.wordreference.com/enfr/life}{life}} qui s'écrit
  phonétiquement
  \href{https://dictionary.cambridge.org/dictionary/english/life}{\exPH{laɪf}}. Exemple
  d'utilisation du mot :

  \begin{itemize}
  \item\exEN{The author \href{https://youtu.be/5x0ARyfNyGc}{withdrew} from public \href{https://youtu.be/zyKGKoGACVk}{life}.}
  \item\exFR{L'auteur s'est retiré de la vie publique.}
  \end{itemize}

  \youglish{life}
  
\end{enumerate}

\newpage

\section{ \son{əʊ} pour les anglais et\son[~]{oʊ} pour les américains}\label{sec:enenvomegaenv}

\begin{itemize}
\item Pour l'anglais britannique on a :
  \diph{ə}{ʊ}{əʊ}{diphthong-4-7}{\son{ə} a été étudié
  page~\pageref{sec:sonenv} et pour le \son[~]{ʊ} voir
  page~\pageref{sec:omega}.}
  \uks{https://youtu.be/Z-pZswbP0-g}
\item Pour l'anglais américain on a :
  \diph{o}{ʊ}{oʊ}{diphthong-4-7}{\son{o} n'a pas véritablement été étudié
  parce que ce n'est pas un symbole phonétique dans la langue anglaise
  et pour le \son[~]{ʊ} voir page~\pageref{sec:omega}.}
  \uss{https://youtu.be/Civ7UBZP99}
\end{itemize}

\begin{enumerate}
\item \exEN{\href{http://www.wordreference.com/enfr/alone}{alone}} qui
  s'écrit phonétiquement
  \href{https://en.oxforddictionaries.com/definition/alone}{\exPH{əˈləʊn}}

  \begin{itemize}
  \item\exEN{I experience real \href{https://youtu.be/cnsk7iXFCtY}{joy} when I am alone in \href{https://youtu.be/9wAft7t8k2c}{nature}.}
  \item\exFR{Je ressens une joie réelle quand je suis seul dans la
      nature.}
  \end{itemize}

  \youglish{alone}
  
\item \exEN{\href{http://www.wordreference.com/enfr/goat}{goat}} qui s'écrit
  phonétiquement
  \href{https://en.oxforddictionaries.com/definition/goat}{\exPH{ɡəʊt}}

  \begin{itemize}
  \item\exEN{Behind a door there is a \href{https://youtu.be/VXw0XGOVQvw}{sports} car and behind each of
      the other two there is a \href{https://youtu.be/4Lb-6rxZxx0}{goat}.}
  \item\exFR{Derrière une porte il y a une voiture de sport et
      derrière chacune des deux autres il y a une chèvre.}
  \end{itemize}

  \youglish{goat}
  
\item \exEN{\href{http://www.wordreference.com/enfr/hope}{hope}} qui s'écrit
  phonétiquement
  \href{https://en.oxforddictionaries.com/definition/hope}{\exPH{həʊp}}

  \begin{itemize}
  \item\exEN{I \href{https://youtu.be/\_pKcv0Fml-A}{hope} you will \href{https://youtu.be/aGSKrC7dGcY}{enjoy} your stay.}
    \item\exFR{J'espère que vous apprécierez votre séjour.}
    \end{itemize}

    \youglish{hope}
    
\item \exEN{\href{http://www.wordreference.com/enfr/road}{road}} qui s'écrit
  phonétiquement
  \href{https://en.oxforddictionaries.com/definition/road}{\exPH{rəʊd}}

  \begin{itemize}
  \item\exEN{\href{https://youtu.be/jzmy6iUGDo8}{Body like a back road.}}
  \item\exFR{Un corps comme une route de retour.}
  \end{itemize}

  \youglish{road}
  
\end{enumerate}

\newpage

\section{ \son{ʊə} pour les anglais et juste le\son[~]{ʊ} pour les
  américains}\label{sec:omegaenvenenv}

\begin{itemize}
\item Pour l'anglais britannique on a :
  \diph{ʊ}{ə}{ʊə}{diphthong-8}{\son{ʊ} a été étudié
  page~\pageref{chap:omega} et pour le \son[~]{ə} voir
  page~\pageref{chap:sonenv}.}
  \uks{https://youtu.be/feseqejnkL0}
\item Pour l'anglais américain voir le \son[~]{ʊ} page~\pageref{chap:omega}.
  \uss{https://youtu.be/moLTR-dLQQY}
\end{itemize}

\begin{enumerate}
\item \exEN{\href{http://www.wordreference.com/enfr/tourist}{tourist}} qui
  s'écrit phonétiquement
  \href{https://en.oxforddictionaries.com/definition/tourist}{\exPH{/ˈtʊərɪst/}}

  \begin{itemize}
  \item\exEN{As a \href{https://youtu.be/98H5AN_vfOY}{tourist} I like to go where \href{https://youtu.be/PCco_YBsU3w}{others} don't.}
  \item\exFR{En tant que touriste j'aime aller là où les autres ne
      vont pas.}
  \end{itemize}

  \youglish{tourist}
  
\item \exEN{\href{http://www.wordreference.com/enfr/boor}{boor}} qui s'écrit
  phonétiquement
  \href{https://en.oxforddictionaries.com/definition/boor}{\exPH{bʊə}}

  \begin{itemize}
  \item\exEN{Nowadays it is difficult to know how to \href{https://youtu.be/Gp2GMujyC58}{behave} with women
      in order to avoid to be considered as a \href{https://youtu.be/nHe5NgSRw3A}{boor} or a boring man.}
  \item\exFR{De nos jours il est difficile de savoir comment se
      comporter avec les femmes afin d'éviter d'être considéré comme
      un rustre ou un homme ennuyeux.}
  \end{itemize}

  \youglish{boor}
  
\item \exEN{\href{http://www.wordreference.com/enfr/pure}{pure}} qui s'écrit
  phonétiquement
  \href{https://en.oxforddictionaries.com/definition/pure}{\exPH{pjʊə}}

  \begin{itemize}
  \item\exEN{It is \href{https://youtu.be/kw7tpfeRHFI}{pure} chance if you
      are an English \href{https://youtu.be/ChZJ1Q3GSuI}{native} speaker or not because nobody choose his parents.}
    \item\exFR{C'est du pur hasard si vous êtes un anglophone natif ou
        pas parce que personne ne choisit ses parents.}
    \end{itemize}

    \youglish{pure}
    
\item \exEN{\href{http://www.wordreference.com/enfr/sure}{sure}} qui s'écrit
  phonétiquement
  \href{https://en.oxforddictionaries.com/definition/sure}{\exPH{ʃʊə}}

  \begin{itemize}
  \item\exEN{I'm \href{https://youtu.be/4HPR8WDwWb0}{sure} you will \href{https://youtu.be/QG2p53z67vk}{succeed}.}
  \item\exFR{Je suis sûr que tu vas réussir.}
  \end{itemize}

  \youglish{sure}
  
\end{enumerate}

\newpage

\section{ \son{aʊ} }\label{sec:aomega}

\diph{aː}{ʊ}{aʊ}{diphthong-5-7}{\son{aː} a été étudié
  page~\pageref{chap:sonalong} et pour le \son[~]{ʊ} voir
  page~\pageref{chap:omega}.}

\properukus{https://youtu.be/JrWuLH_AYM4}{https://youtu.be/-V690OA75bA}

\begin{enumerate}
\item \exEN{\href{http://www.wordreference.com/enfr/now}{now}} qui s'écrit
  phonétiquement
  \href{https://en.oxforddictionaries.com/definition/now}{\exPH{naʊ}}

  \begin{itemize}
  \item\exEN{I am \href{https://youtu.be/xcpxjx2fy\_E}{now} completely free and \href{https://youtu.be/q86Z6Xw_C0w}{unencumbered}.}
  \item\exFR{Je suis désormais complètement libre et sans contrainte.}
  \end{itemize}

  \youglish{now}
  
\item \exEN{\href{http://www.wordreference.com/enfr/round}{round}} qui
  s'écrit phonétiquement
  \href{https://en.oxforddictionaries.com/definition/round}{\exPH{raʊnd}}

  \begin{itemize}
  \item\exEN{The \href{https://youtu.be/LV7JviaH-HU}{boxer} won the fight in the second \href{https://youtu.be/oGTBax-Cu4Q}{round}.}
  \item\exFR{Le boxeur a gagné le combat au deuxième round.}
  \end{itemize}

  \youglish{round}
  
\item \exEN{\href{http://www.wordreference.com/enfr/mouth}{mouth}} qui
  s'écrit phonétiquement
  \href{https://en.oxforddictionaries.com/definition/mouth}{\exPH{maʊθ}}

  \begin{itemize}
  \item\exEN{In \href{https://youtu.be/FYH8DsU2WCk}{order} to produce a vowel you need to open your
      \href{https://youtu.be/kkDHKSNrJ5g}{mouth}.}
  \item\exFR{Afin de produire une voyelle vous devez ouvrir votre
      bouche.}
  \end{itemize}

  \youglish{mouth}
  
\item \exEN{\href{http://www.wordreference.com/enfr/brown}{brown}} qui
  s'écrit phonétiquement
  \href{https://en.oxforddictionaries.com/definition/brown}{\exPH{braʊn}}

  \begin{itemize}
  \item\exEN{\href{https://youtu.be/OwTXBBU0JLo}{Brown} is just a \href{https://youtu.be/ufWtK3qizPA}{colour}.}
  \item\exFR{Le marron est juste une couleur.}
  \end{itemize}

  \youglish{brown}
  
\end{enumerate}

\newpage

\section{ \son{ɪə} }\label{sec:ieenv}

\begin{itemize}
\item Pour l'anglais britannique on a :
  \diph{ɪ}{ə}{ɪə}{diphthong-6-7}{\son{ɪ} a été étudié
  page~\pageref{chap:soni} et pour le \son[~]{ə} voir
  page~\pageref{chap:sonenv}.}
  \uks{https://www.youtube.com/watch?v=AnjNcqUKSsE}
\item Pour l'anglais américain ce n'est pas une diphtongue c'est tout
  simplement le\son[~]{ɪ} qui a été étudié page~\pageref{chap:soni}.
  \uss{https://youtu.be/-km81q6DIlM}
\end{itemize}

\begin{enumerate}
\item \exEN{\href{http://www.wordreference.com/enfr/weird}{weird}} qui s'écrit phonétiquement \href{https://en.oxforddictionaries.com/definition/weird}{\exPH{wɪəd}}

  \begin{itemize}
  \item\exEN{He always has \href{https://youtu.be/fcdUXnt87ng}{weird} dreams that \href{https://youtu.be/FikYhD7bXYE}{nobody} understands.}
  \item\exFR{Il fait toujours des rêves bizarres que personne ne
      comprend.}
  \end{itemize}

  \youglish{weird}
  
\item \exEN{\href{http://www.wordreference.com/enfr/beer}{beer}} qui s'écrit
  phonétiquement
  \href{https://en.oxforddictionaries.com/definition/beer}{\exPH{bɪə}}

  \begin{itemize}
  \item\exEN{\href{https://youtu.be/71du169Hn90}{Football} supporters usually drink \href{https://youtu.be/I1fsk4k-bOs}{beer}.}
  \item\exFR{Les supporters de foot boivent habituellement de la
      bière (attention à consommer avec modération).}
  \end{itemize}
  
\item \exEN{\href{http://www.wordreference.com/enfr/near}{near}} qui s'écrit
  phonétiquement
  \href{https://en.oxforddictionaries.com/definition/near}{\exPH{nɪə}}. Exemple
  d'utilisation du mot :

  \begin{itemize}
  \item\exEN{\href{https://youtu.be/0TZg_a6T-Cw}{UK} is \href{https://youtu.be/xIS9K-bNt3M}{near} from France.}
  \item\exFR{Le Royaume-Uni est proche de la France.}
  \end{itemize}

  \youglish{near}
  
\item \exEN{\href{http://www.wordreference.com/enfr/steer}{steer}} qui
  s'écrit phonétiquement
  \href{https://en.oxforddictionaries.com/definition/steer}{\exPH{stɪə}}

  \begin{itemize}
  \item\exEN{The politician \href{https://youtu.be/z\_vSRFODAxU}{steered} the conversation to a different
      \href{https://youtu.be/pjYVTe0mcGg}{topic}.}
  \item\exFR{L'homme politique a orienté la conversation vers un autre sujet.}
  \end{itemize}

  \youglish{steer}
  
\end{enumerate}

\newpage

\section{ \son{eə} qui devrait plutôt s'écrire le \son[~]{ɛə} à
  juste titre}\label{sec:eeteenv}

\diph{ɛ}{ə}{ɛə}{diphthong-7-7}{\son{ɛ} a été étudié
  page~\pageref{chap:sone} et pour le \son[~]{ə} voir
  page~\pageref{chap:sonenv}.}
  \uks{https://youtu.be/Ff-MqM6Zb4Q}

\begin{enumerate}
\item \exEN{\href{http://www.wordreference.com/enfr/bear}{bear}} qui s'écrit
  phonétiquement
  \href{https://dictionary.cambridge.org/dictionary/english/bear}{\exPH{bɛə}}

  \begin{itemize}
  \item\exEN{This \href{https://youtu.be/bo_GZ3q_Ays}{noise} is difficult to \href{https://youtu.be/NJ6jv\_lPBN8}{bear}.}
  \item\exFR{Ce bruit est difficile à supporter.}
  \end{itemize}

  \youglish{bear}
  
\item \exEN{\href{http://www.wordreference.com/enfr/rare}{rare}} qui s'écrit
  phonétiquement
  \href{https://dictionary.cambridge.org/dictionary/english/rare}{\exPH{rɛə}}

  \begin{itemize}
  \item\exEN{The \href{https://youtu.be/a7nrDN15NPE}{consultant} is an expert in \href{https://youtu.be/hPncU3924fU}{rare} illnesses.}
  \item\exFR{Le médecin spécialiste est expert en maladies rares.}
  \end{itemize}

  \youglish{rare}
  
\item \exEN{\href{http://www.wordreference.com/enfr/there}{there}} qui
  s'écrit phonétiquement
  \href{https://dictionary.cambridge.org/dictionary/english/there}{\exPH{ðeər}}

  \begin{itemize}
  \item\exEN{My friend is always \href{https://youtu.be/fg9pkAYvrSM}{there} for me when I \href{https://youtu.be/rdUeq09cGJ0}{need} her.}
  \item\exFR{Mon amie est toujours là pour moi quand j'ai besoin
      d'elle.}
  \end{itemize}

  \youglish{there}
  
\item \exEN{\href{http://www.wordreference.com/enfr/care}{care}} qui s'écrit
  phonétiquement
  \href{https://dictionary.cambridge.org/dictionary/english/care}{\exPH{keər}}

  \begin{itemize}
  \item\exEN{\href{https://youtu.be/y1KIVZw7Jxk}{Babies} need constant \href{https://youtu.be/ClrSEz\_tBZw}{care}.}
  \item\exFR{Les bébés ont besoin d'une attention constante.}
  \end{itemize}

  \youglish{care}
  
\end{enumerate}

\newpage
\minitoc
